\documentclass[12 pt]{article}
\usepackage[utf8]{inputenc}
\usepackage{graphicx}
\usepackage{amsmath}
\usepackage[version=4]{mhchem}
\usepackage{siunitx}
\usepackage{longtable,tabularx}
\usepackage{float}
\usepackage[left=1in, right=1in]{geometry}
\usepackage{pdfpages}

\title{MECH6066 HW\#4}
\date{11.02.25}
\author{Slade Brooks \\ M13801712}

\begin{document}
\maketitle

\section*{Problem 1}
\subsection*{(a)}
The mass law states that the transmission loss through a sufficiently thin wall is given by:
\begin{align*}
    T_L=20\log_{10}\left(\frac{\rho_2L\omega}{2Z_1}\right) \\
    10^{T_L/20}=\frac{\rho_2L\omega}{2Z_1} \\
    L=\frac{2Z_1}{\rho_2\omega}10^{(T_L/20)}
\end{align*}
The design requires a transmission loss of 30dB at 300Hz. The imepdance $Z_1$ will be the speed of sound of air times
the density, assuming standard air. We can plug these known values in to solve for the required
wall thickness based on the wall density to satisfy the requirement for each material.
\begin{center}
    \begin{tabular}{c|c|c|c}
        Material & Density (kg/m$^3$) & Speed of Sound (m/s) & L (mm) \\
        \hline
        air & 1.21 & 343 & --- \\
        steel & 7700 & 6100 & 1.81 \\
        glass & 2300 & 5600 & 6.05 \\
        concrete & 2600 & 3100 & 5.36\\
        pine & 450 & 3500 & 30.95
    \end{tabular}
\end{center}

\subsection*{(b)}
The exact transmission loss is given by:
\begin{align*}
    T_L=20\log_{10}\left(\frac{2+\left(Z_3/Z_1+Z_1/Z_3\right)\cos^2{(K_2L)}+
    \left(Z_2^2/Z_1Z_3+Z_1Z_3/Z_2^2\right)\sin^2{(K_2L)}}{4}\right) \\
    K_2 = \omega/c_2;\quad Z_1=Z_3 \\
    T_L=20\log_{10}\left(\frac{2+2\cos^2{(K_2L)}+
    \left(Z_2^2/Z_1^2+Z_1^2/Z_2^2\right)\sin^2{(K_2L)}}{4}\right)
\end{align*}
We can plug in the known values ($Z_1$, $Z_2$, $K_2$, $L$) to find the actual transmission loss through each material.
\begin{center}
    \begin{tabular}{c|c}
        Material & $T_L$ actual (dB) \\
        \hline
        steel & 30.01 \\
        glass & 30.0 \\
        concrete & 30.01 \\
        pine & 30.01
    \end{tabular}
\end{center}
We can see that the mass law is clearly valid for all materials. The lowest speed of sound material, concrete, has a
wavelength of 10~meters at 300Hz, and the largest wall thickness was pine at 31~mm. All of the materials clearly satisfy
the assumption that the thickness is very small compared to the wavelength.

\subsection*{(c)}
I would probably choose pine. Assuming the desire to reduce noise is primarily for a residential area, wooden fences are
a clear choice. The 31~mm required thickness is around 1.25~inches, which is only slightly smaller than the width of a
normal 2x4 wood board. A very simple pine fence made of 2x4s should effectively reduce the center frequency by the
desired amount, while looking nice and being a very cheap option in terms of parts and labor.

\section*{Problem 2}
\subsection*{(a)}
We know the vector $\vec{k}$ can be defined by the angle and its magnitude. When the switch to magnitude is made, the
magnitude is just the regular wave number based on the material. From the book, fresh water has a density of 998kg/m$^3$
and a speed of sound of 1481m/s.
\begin{align*}
    \vec{k}=||\vec{k}||(\cos{\theta}\hat{x} + \sin{\theta}\hat{y})=K(\cos{30}\hat{x}+\sin{30}\hat{y}) \\
    \vec{k}=\frac{\omega}{c}(0.866\hat{x}+0.5\hat{y})=\frac{2\pi(100000)}{1481}(0.866\hat{x}+0.5\hat{y}) \\
    \fbox{$\vec{k}=367.4\hat{x}+212.13\hat{y}$}
\end{align*}

\subsection*{(b)}
\begin{align*}
    K=\frac{2\pi}{\lambda}\rightarrow K_x=\frac{2\pi}{\lambda_x};\quad K_y=\frac{2\pi}{\lambda_y} \\
    \lambda_x=\frac{2\pi}{K_x}=\frac{2\pi}{367.4}=0.0171\text{m} \\
    \lambda_y=\frac{2\pi}{K_y}=\frac{2\pi}{212.13}=0.03\text{m} \\
    \fbox{$\lambda_x=17.1$mm; $\lambda_y=30$mm}
\end{align*}
\begin{align*}
    \lambda=\frac{c}{f}=\frac{1481}{100000}=0.01481\text{m} \\
    \fbox{$\lambda=14.8$mm}
\end{align*}
We can see that the wavelength in both directions is larger than the wavelength for 100kHz in water. This makes sense as
the wave front is moving slower with respect to both directions. Since it is close to the x axis, $\lambda_x$ is not
much different than the normal wavelength, but the y direction will stretch much more.

\subsection*{(c)}
\subsubsection*{i.}
Following the derivation we did in class for 2D instead of 3D:
\begin{align*}
    -\nabla p=\rho\frac{\partial u}{\partial t} \\
    -\nabla \left[\bar{p}e^{j(\omega t-k_xx-k_yy)}\right]=\rho\frac{\partial u}{\partial t} \\
    -\left(\frac{\partial}{\partial x}\hat{x}+\frac{\partial}{\partial y}\hat{y}\right)\bar{p}e^{j(\omega t-k_xx-k_yy)}
    =\rho\frac{\partial u}{\partial t} \\
    -\left(-jk_x\hat{x}-jk_y\hat{y}\right)\bar{p}e^{j(\omega t-k_xx-k_yy)}
    =\rho\frac{\partial u}{\partial t} \\
    j\vec{k}\bar{p}e^{j(\omega t-k_xx-k_yy)}=\rho\frac{\partial u}{\partial t} \\
    \frac{\vec{k}}{\omega\rho}\bar{p}e^{j(\omega t-k_xx-k_yy)}=\vec{u} \\
    \fbox{$\vec{u}(x,y,t)=\frac{\vec{k}}{\omega\rho}\bar{p}e^{j(\omega t-k_xx-k_yy)}$}
\end{align*}

\subsubsection*{ii.}
We know the magnitude will be everything except for the exponential term. We can take this and simplify it based on what
relationships we know.
\begin{align*}
    ||\vec{u}||=u=\frac{\vec{k}}{\omega\rho}\bar{p}=\frac{K\hat{e}_k}{\omega\rho}\bar{p}=\frac{K}{\omega\rho}\bar{p}
    =\frac{\omega}{c\omega\rho}\bar{p}=\frac{\bar{p}}{c\rho}=\frac{\bar{p}}{Z_0} \\
    \fbox{$u=\frac{\bar{p}}{Z_0}$}
\end{align*}

\subsubsection*{iii.}
First we can solve for the magnitude of the velocity vector since we know both parts:
\begin{align*}
    u=\frac{\bar{p}}{Z_0}=\frac{1000000}{998*1481}=0.6766 \\
    \fbox{$u=0.6766$m/s}
\end{align*}
We can plug this and some other knowns back into the particle velocity equation:
\begin{align*}
    \vec{u}=0.6766e^{j(\omega t-k_xx-k_yy)}=0.6766e^{j(2\pi(100000)t-367.4x-212.13y)} \\
    \fbox{$\vec{u}=0.6766e^{j(628318.5t-367.4x-212.13y)}$}
\end{align*}
It would also be given in vector form as:
\begin{align*}
    \fbox{$\vec{u}=0.6766\left(\cos{30}\hat{x}+\sin{30}\hat{y}\right)$}
\end{align*}

\subsection*{(d)}
See attached model pdf.

\subsection*{(e)}
Yes, there are waves at a 30$^{\circ}$ angle varying from -1 to 1 MPa.
\begin{figure}[H]
    \centering
    \includegraphics[width=0.5\linewidth]{figs/hw4fig1.png}
\end{figure}

\subsection*{(f)}
Yes, the waves propagate towards the northeast. To make them move southwest, we would switch the sign of both components
of $\hat{e}_k$.

\subsection*{(g)}
Based on the plot, one of the peaks goes from 15mm to 45mm, giving a wavelength of 30mm. This is exactly what was
calculated in part b.
\begin{figure}[H]
    \centering
    \includegraphics[width=0.5\linewidth]{figs/hw4fig2.png}
\end{figure}

\subsection*{(h)}
\begin{figure}[H]
    \centering
    \includegraphics[width=0.5\linewidth]{figs/hw4fig3.png}
\end{figure}
\subsubsection*{i.}
There are arrows going both northeast and southwest. This makes sense as the particles should be oscillating in both
directions along the angle of the plane wave and they are aligned with where wave fronts should be.

\subsubsection*{ii.}
The particles oscillate, flipping their velocity vectors between northeast and southwest.

\subsection*{(i)}
We can see the resultant plot is all one magnitude of 0.57735m/s. We determined the x direction should be
$0.6766\cos{30}=0.586$ m/s. This is very close to the comsol value. Any difference in the value is due to using a
different speed of sound and density for water. I used the book appendix values for part c, but used the simplified
values from the hint document for the comsol model.
\begin{figure}[H]
    \centering
    \includegraphics[width=0.5\linewidth]{figs/hw4fig4.png}
\end{figure}

\section*{Problem 3}
I thought this assignment was shorter and easier than previous ones. I enjoyed that. The study materials were very
adequate. I felt like my notes from class were more than enough to understand everything and the hint sheet helped
greatly with making comsol take less time.

\includepdf[pages=-]{HW4comsol.pdf}

\end{document}