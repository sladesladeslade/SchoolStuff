\documentclass[12 pt]{article}
\usepackage[utf8]{inputenc}
\usepackage{graphicx}
\usepackage{amsmath}
\usepackage[version=4]{mhchem}
\usepackage{siunitx}
\usepackage{longtable,tabularx}
\usepackage{float}
\usepackage[left=1in, right=1in]{geometry}
\usepackage{pdfpages}

\title{MECH6066 HW\#5}
\date{11.17.25}
\author{Slade Brooks \\ M13801712}

\begin{document}
\maketitle

\section*{Problem 1}
\subsection*{(a)}
\begin{figure}[H]
    \centering
    \includegraphics[width=0.5\linewidth]{figs/hw5fig1.png}
\end{figure}

\subsection*{(b)}
From the plot, it appears the critical angle is 60$^{\circ}$. We can verify this analytically from:
$\sin{\theta_c}=\frac{c_1}{c_2}$. We will do this in matlab to determine the critical angle and we get that
\fbox{$\theta_c=61.12^{\circ}$}.

\subsection*{(c)}
\begin{figure}[H]
    \centering
    \includegraphics[width=0.5\linewidth]{figs/hw5fig2.png}
\end{figure}

\subsection*{(d)}
At normal incidence ($\theta_i=0^{\circ})$, \fbox{$\bar{R}=0.398$ and $\bar{T}=1.398$}.

\subsection*{(e)}
\begin{figure}[H]
    \centering
    \includegraphics[width=0.5\linewidth]{figs/hw5fig3.png}
\end{figure}

\subsection*{(f)}
We can determine the Brewster angle based on the plot in part e. The brewster angle will be at 100\% transmission, so
our reflected intensity coefficient would be 0. We can see that since this does not happen, there is no brewster angle
for this combination of materials.

\section*{Problem 2}
\subsection*{(a)}
See attached model.

\subsection*{(b)}
Yes, there are downwards moving waves and a standing wave in the water. The transmitted wave is roughly 1.4~Pa, so it
matches very closely to previous calculations.

\subsection*{(c)}
When plotting just ps, we can see that the reflected wave is rougly 0.4~Pa in amplitude which is exactly what we
calculated previously.
\begin{figure}[H]
    \centering
    \includegraphics[width=0.5\linewidth]{figs/hw5fig4.png}
\end{figure}

\subsection*{(d)}
We can see that the amplitude does not vary since there is just a wave transmitted into the sand and no standing wave to
change the amplitude.
\begin{figure}[H]
    \centering
    \includegraphics[width=0.5\linewidth]{figs/hw5fig5.png}
\end{figure}

\subsection*{(e)}
We can see that the reflected pressure does match what is expected (roughly 0.5~Pa).
\begin{figure}[H]
    \centering
    \includegraphics[width=0.5\linewidth]{figs/hw5fig6.png}
\end{figure}
We can see that the amplitude still does not vary vertically, and shouldn't because it is still just transmitted
pressure into the quartz sand not a standing wave.
\begin{figure}[H]
    \centering
    \includegraphics[width=0.5\linewidth]{figs/hw5fig7.png}
\end{figure}

\subsection*{(f)}
The waves in the sand move only in the x direction. I believe these would be transverse waves. The amplitude does vary
in the vertical direction but I believe this is due to artifacts in the simulation, not a physical phenomenon.
\begin{figure}[H]
    \centering
    \includegraphics[width=0.5\linewidth]{figs/hw5fig8.png}
\end{figure}

\section*{Problem 3}
\subsection*{(a)}
See attached model.

\subsection*{(b)}
The generated waves can be approximated as spherical. The wavelength of the generated sound will be 0.343~meters and the
largest dimension of the speaker geometry is only 0.02~meters, which should be small enough compared to the wavelength
to approximate the generated wave as spherical.

\subsection*{(c)}
Evaluating at the far right side of the domain gave an amplitude of 0.007~Pa at 1~m, which means
\fbox{$A=0.007$ Pa$\cdot$m}

\subsection*{(d)}
\begin{align*}
    I=re\{0.5\cdot p\cdot u^*\} \\
    I=re\{\frac{p}{2}\left[\frac{1}{Z_0}\left(1 + \frac{1}{jkr}\right)p\hat{r}\right]^*\}
\end{align*}
The exponential terms will now cancel out since one is a conjugate, and because we are taking only the real any
remaining j terms disappear.
\begin{align*}
    I=\frac{|A|}{2r}e^{j(\omega t-kr)}\frac{1}{Z_0}\frac{|A|}{r}e^{-j(\omega t-kr)}\hat{r} \\
    I=\frac{|A|}{2r}\frac{1}{Z_0}\frac{|A|}{r}\hat{r}
\end{align*}
\begin{align*}
    \fbox{$I=\frac{|A|^2}{2Z_0r^2}\hat{r}$}
\end{align*}
We can see that the intensity goes down relative to the radius squared as we move farther away.

\subsection*{(e)}
\begin{align*}
    I=\frac{|A|^2}{2Z_0r^2}=\frac{0.007^2}{2(343\cdot1.15)}=6.2e-8W/m^2 \\
    SIL = 10\log_{10}{\left(I/I_{ref}\right)}=10\log_{10}{(6.2e-8/10^{-12})}=\fbox{$SIL=48dB$}
\end{align*}
I get an SPL of just under 48~dB from the comsol model, so the approximation is very accurate.

\subsection*{(f)}
\begin{align*}
    W = I4\pi a^2=\frac{|A|^2}{2Z_0r^2}4\pi a^2=\frac{2\pi|A|^2}{Z_0}=\fbox{$7.8e-7W$}
\end{align*}
The power does not depend on the distance. The intensity depends on the distance, but the power equation simplifies to
not have any radius term in it, so it depends only on the amplitude of the pressure and the impedance.

\subsection*{(g)}
The comsol model reports a power of $8.3e-7$~W, whereas I calculated a power of $7.8e-7$~W. These values do differ
slighlty, but are fairly close. I believe the difference in value comes from the spherical assumption. The spherical
assumption for the analytical value assumes a spherical source at 0, 0. Our comsol model is slightly offset from that
and has a linear moving source instead of a single breathing point, so it makes sense that comsol would calculate a
higher power. At some points, the sound generating surface is closer to the edge of the domain than it would be for the
spherical approximation, and the pressure wave generated is larger since it has some length to it.

\subsection*{(h)}
I would not assume the waves are spherical. The wavelength would go down to 0.03~m, which is only slightly bigger than
the radius of the speaker, so it is probably not a large enough difference to be considered spherical.

\section*{Problem 4}
My only change for this assignment would be less comsol work. I think this assignment is well-structured and helpful for
understanding some of the concepts. However, I sometimes have issues where comsol is a little clunky or confusing and I
waste time troubleshooting comsol instead of getting valuable insights.

\section*{Problem 1 Matlab}
\begin{verbatim}
clear variables; close all;

% define properties from appendix
rhoSW = 1026;
cSW = 1500;
ZSW = 1.54e6;
rhoQS = 2070;
cQS = 1730;
ZQS = 3.58e6;

%% Part A
% set range of angles to check
thetai = -90:1:90;

% calculate xmission angle for each
thetat = asind(cQS/cSW.*sind(thetai));

% plot results
figure
plot(thetai, thetat)
xlim([thetai(1) thetai(end)]); xlabel("Incident Angle (deg)")
ylim([-100 100]); ylabel("Transmission Angle (deg)")
grid on

%% Part B
% calc critical angle
thetac = asind(cSW/cQS);

%% Part C
% calc reflection coef (magnitude)
Rbar = abs((ZQS./cosd(thetat) - ZSW./cosd(thetai))./(ZQS./cosd(thetat) + ZSW./cosd(thetai)));

% calc xmission coef (magnitude)
Tbar = abs(2*(ZQS./cosd(thetat))./(ZQS./cosd(thetat) + ZSW./cosd(thetai)));

% plot results
figure
hold on
plot(thetai, Rbar); plot(thetai, Tbar)
hold off
xlim([thetai(1) thetai(end)]); xlabel("Incident Angle (deg)")
ylim([0 2.5]); ylabel("Pressure Coefficient")
legend(["Reflected", "Transmitted"])
grid on

%% Part D
% calculate reflection intensity coeff
RI = Rbar.^2;

% plot
figure
plot(thetai, RI)
xlim([thetai(1) thetai(end)]); xlabel("Incident Angle (deg)")
ylim([0 1.5]); ylabel("Reflected Intensity Coefficient")
grid on
\end{verbatim}

\includepdf[pages=-]{HW5.2comsol.pdf}
\includepdf[pages=-]{HW5.3comsol.pdf}

\end{document}