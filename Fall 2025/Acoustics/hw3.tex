\documentclass[12 pt]{article}
\usepackage[utf8]{inputenc}
\usepackage{graphicx}
\usepackage{amsmath}
\usepackage[version=4]{mhchem}
\usepackage{siunitx}
\usepackage{longtable,tabularx}
\usepackage{float}
\usepackage[left=1in, right=1in]{geometry}
\usepackage{pdfpages}

\title{MECH6066 HW\#3}
\date{10.19.25}
\author{Slade Brooks \\ M13801712}

\begin{document}
\maketitle

\section*{Problem 1}
We know that the sound pressure level is equal to the sound intensity level. We will assume the standard value of
$I_{ref}$ for air of $10^{-12}$W/m$^2$.
\begin{align*}
    SPL=SIL=10\log_{10}\left(\frac{I}{I_{ref}}\right) \\
    120=10\log_{10}\left(\frac{I}{10^{-12}}\right) \\
    I = 10^{-12}\cdot 10^{\frac{120}{10}}=1\text{W/m}^2
\end{align*}
Now we can add our efficiency and calculate the size required to generate 3W of power.
\begin{align*}
    P_{req}=\frac{3}{0.95}=3.158\text{W} \\
    A_{req}=3.158/1=3.158\text{m}^2 \\
    r_{req}=\sqrt{A_{req}/\pi}=1\text{m} \\
    \fbox{required diameter $D_{req}=2$m}
\end{align*}
Repeating the same calculations with 80dB instead of 120dB gives:
\begin{align*}
    \fbox{required diameter $D_{req}=200.5$m}
\end{align*}
Based on these results, harvesting power from acoustic energy is not very reasonable. It requires a very loud
environment to have a reasonably sized device, but the device is only able to slow charge 1 phone. There are not many
environments where it is consistently 120dB to generate power, and even needing a 2 meter circle to charge one phone is
fairly ridiculous. Clearly the problem is even worse when considering regular ambient noise that requires a 200 meter
device to charge one phone. These also assume a 95\% efficiency on the device, which is likely not possible to achieve.

\section*{Problem 2}
\subsection*{(a)}
From the textbook for water (1) and aluminum (2):
\begin{align*}
    Z_1 = 1.48E6; \quad Z_2 = 17E6
\end{align*}

The percentage reflected is given by the reflection coefficient:
\begin{align*}
    R_I=\left(\frac{Z_2-Z_1}{Z_2+Z_1}\right)^2=\left(\frac{17-1.48}{17+1.48}\right)^2=0.705 \\
    \fbox{amount of intensity reflected is 70.5\%}
\end{align*}

\subsection*{(b)}
For maximum transmission, we will do quarter-wavelength matching. The ideal material will have an impedance given by:
\begin{align*}
    Z_L=\sqrt{Z_1\cdot Z_2}=\sqrt{1.48E6\cdot 17E6}=5.016E6 \\
    \fbox{$Z_L=5.016E6$ Ry}
\end{align*}
The material will have a length equal to 1/4 of the wavelength at the desired frequency to transmit. We can determine
that the required speed of sound of the material is given by the same equation as the impedance.
\begin{align*}
    c_L = \sqrt{1481*6300}=3054.6\text{m/s} \\
    L=\frac{c}{4f}=\frac{3054.6}{4\cdot 2E6}=0.000382\text{m} \\
    \fbox{ideal thickness $L=0.382$mm}
\end{align*}

\subsection*{(c)}
There are no perfect materials in the appendix. The closest is quartz sand. It has an impedance of 3.58E6Ry and speed of
sound of 1730 m/s. Its required length would be:
\begin{align*}
    L=\frac{c}{4f}=\frac{1730}{4\cdot 2E6}=0.0002163\text{m} \\
    \fbox{required thickness $L=0.2163$mm}
\end{align*}

\subsection*{(d)}
\begin{figure}[H]
    \centering
    \includegraphics[width=0.75\linewidth]{figs/hw3fig1.png}
\end{figure}

\subsection*{(e)}
We can see without a matching layer, only around 30\% of the intensity is transmitted. For the ideal layer, this goes up
to 100\% at our design point of 2MHz with a steep falloff on either side. The layer using the quartz sand, the most
practical option from the book, transmits around 90\% at the design frequencies. Overall, the matching layer is fairly
good at transmitting despite having a different impedance. There is also good transmission at off-design frequencies.
This is because the quarter-wavelength matching that is done to set the length is valid for any odd integer multiples of
the design frequency. This makes sense as we can see transmission peaks at 6MHz (3x) and 10MHz (5x), with the layer
behaving as if it is not there at even multiples of the design frequency.

\section*{Problem 3}
\subsection*{(a)}
The results for the no transmission layer match what I expect. The pressure wave goes between the two mediums and
changes wavelength as expected for each domain. Since the plot is of total acoustic pressure, we are unable to see the
change in intensity in this plot. I would expect an animation of this plot would show a standing wave in the water.

\subsection*{(b)}
\subsubsection*{i}
The amplitude of the pressure wave is constant at the input 1Pa. The instantaneous value changes over time as it is a
pressure wave, so the value fluctuates between -1 and 1Pa at 2MHz as time goes on.

\subsubsection*{ii}
The pressure amplitude fluctuates in the water because a large amount is reflected and creates a standing wave in the
water. This wave leads to a fluctuation in the amplitude of the pressure. However, since aluminum has some intensity
transferred to it but nothing reflected within, it has constant amplitude.

\subsubsection*{iii}
The incident and scattered have different magnitudes in each medium. In the aluminum, there is no indicent so its
magnitude is 0. The incident wave in water has the magnitude we input. The scattered is also different. Water has a
lower ampltiude than aluminum for its scattered wave, and a fluctuation is clearly visible (since it is a standing
wave). Aluminum has a much higher amplitude for its scattered wave which makes sense since this setup has a positive
transmission coefficient.

\subsubsection*{iv}
From the plot, the scattered amplitude is around 0.84 Pa on average. This gives $\hat{R}=\frac{0.84}{1}=0.84$. Using
comsol parameters to calculate the analytical reflection coefficient, we can see that it is 0.84011, which is almost
identical to the measured value from the comsol sim.

\subsection*{(c)}
We can see that comsol is very close to the analytical solution for the transmitted intensity. It varies from 29.4\% to
32.2\% across the range of frequencies. The analytical solution is a constant value of 29.4\%. The variation shown in
comsol is very likely due to the meshing of the problem. I assume the mesh is generated for the first input frequency, so
it should be less accurate at other simulation frequencies. We can see for the second model with an ideal transmission
layer, comsole predicts a similar shape to the analytical solution but it overpredicts the transmission at the design
frequency (reporting slightly over 100\%). Running the same model with quartz sand instead of the ideal values shows the
same results---the same shape to the analytical solution but with an overprediction at the design frequency.

\section*{Problem 4}
I preferred having the hints separate. I thought having more detail given with them and having the option to reference
it when I felt stuck was a good way to work through the problems and very similar to how I would interact with
classmates or AI for support on an assignment. I thought this assignment was very good and probably my favorite in terms
of applying what we learned in class and understanding its real-world application.

\section*{Matlab Code}
\begin{verbatim}
%%
% Slade Brooks
% brooksl@mail.uc.edu
% 10.19.25
% MECH6066
% HW 03

clear variables
close all

%% set up vals
% range of frequencies
fs = 0.1e6:0.01e6:10e6;
f = 2e6;

% acoustic properties
Z1 = 1.48e6; c1 = 1481;
Z2 = 17e6; c2 = 6300;
ZL = sqrt(Z1*Z2); cL = sqrt(c1*c2); L = cL/(4*f);
Zs = 3.58e6; cs = 1730; Ls = cs/(4*f);

%% calculate transmission
% w/o matching layer
TIi = 4*Z1*Z2/(Z1 + Z2)^2;

% w/ ideal matching layer
w = (2*pi).*fs;
K = w./cL;
TIii = 4./(2 + (Z2/Z1 + Z1/Z2).*(cos(K*L)).^2 + ((ZL^2)/(Z1*Z2) + (Z1*Z2)/(ZL^2)).*(sin(K*L)).^2);

% w/ practical matching layer
w = (2*pi).*fs;
K = w./cs;
TIiii = 4./(2 + (Z2/Z1 + Z1/Z2).*(cos(K*Ls)).^2 + ((Zs^2)/(Z1*Z2) + (Z1*Z2)/(Zs^2)).*(sin(K*Ls)).^2);

%% plotting
figure
hold on
yline(TIi); plot(fs, TIii); plot(fs, TIiii);
hold off
xlabel("Frequency (Hz)"); xlim([fs(1) fs(end)]);
ylabel("Intensity Transmission Coefficient"); ylim([0 1]);
legend(["w/o matching layer" "w/ ideal matching layer" "w/ practical matching layer"], location="southeast");
grid on
\end{verbatim}

\includepdf[pages=-]{HW3comsol.pdf}
\includepdf[pages=-]{HW3.2comsol.pdf}

\end{document}