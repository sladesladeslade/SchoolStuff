\documentclass[12 pt]{article}
\usepackage[utf8]{inputenc}
\usepackage{graphicx}
\usepackage{amsmath}
\usepackage[version=4]{mhchem}
\usepackage{siunitx}
\usepackage{longtable,tabularx}
\usepackage{float}
\usepackage[left=1in, right=1in]{geometry}

\title{MECH6066 HW\#1}
\date{09.12.25}
\author{Slade Brooks \\ M13801712}

\begin{document}
\maketitle

\section*{Problem 1}

\subsection*{(a)}
The wavelength $\lambda=0.6860$ m.
\begin{figure}[H]
    \centering
    \includegraphics[width=1\linewidth]{figs/hw1fig1.png}
\end{figure} \par

\pagebreak
\subsection*{(b)}
If the frequency is doubled, the wavelength should become half of the original value: $\lambda=0.3430$ m. Wavelength is
inversely related to the frequency ($\lambda=\frac{c}{f}$), so a 2x increase in frequency will become a 2x reduction in
wavelength.
\begin{figure}[H]
    \centering
    \includegraphics[width=1\linewidth]{figs/hw1fig2.png}
\end{figure} \par

\pagebreak
\subsection*{(c)}
If the medium was changed to water, the speed of sound would increase so the wavelength would increase as well. From the
equation shown in Part b, wavelength and speed of sound are directly linearly related.
\begin{figure}[H]
    \centering
    \includegraphics[width=1\linewidth]{figs/hw1fig3.png}
\end{figure} \par

\pagebreak
\subsection*{(d)}
If the piston amplitude is doubled, the pressure amplitude should also double but the wavelength will remain constant.
Wavelength is only related to speed of sound and the frequency, so for a constant material and frequency the wavelength
will remain constant regardless of amplitude. The pressure is directly related to the amplitude of the piston.
\begin{figure}[H]
    \centering
    \includegraphics[width=1\linewidth]{figs/hw1fig4.png}
\end{figure} \par

\pagebreak
\section*{Problem 2}
\subsection*{(a)}
Based on the provided plot of the mic signal, one peak spans from 6.5--12.5 milliseconds. This would be a period of
0.006 seconds. Since $f=1/T$, the frequency would be 167 Hz. The angular frequency $\omega=2\pi f$ so the angular
frequency is 1050 Hz.

\subsection*{(b)}
Assuming the 0.9mV/Pa microphone sensitivity is accurate, the Mic 1 signal has an amplitude of roughly
$0.026*1000/0.9=29$Pa. We will assume that the pressure signal is a right-traveling sinusoid at constant frequency.
The RMS of a pure sine wave is its peak divided by $\sqrt{2}$. The RMS of the Mic 1 signal
$p_{rms}=\frac{29}{\sqrt{2}}=20.5$ Pa. Plugging that value into the equation for the sound pressure level in dB:
\begin{align*}
    SPL=20\log_{10}\left(\frac{p_{rms}}{p_{ref}}\right)=20\log_{10}\left(\frac{20.5}{20E-6}\right)=120\text{ dB}
\end{align*}
The sound measured at Mic 1 is 118 dB. This is roughly as loud as a police siren or rock concert. I would definitely not
be comfortable listening at the end of the tube. While it is still under the threshold for pain, it would probably not
be a pleasant experience to listen to.

\subsection*{(c)}
The total pressure will be the atmospheric plus the acoustic at the start. The acoustic pressure was $p=0.026*1000/0.9=29$
Pa. Assuming a standard atmosphere $p_{atm}=101325$ Pa, the total pressure $p_{tot}=101354$ Pa.

\subsection*{(d)}
Mic 2 is likely at a slight lower amplitude because it is further down the tube. Since it is farther away from the
source, the pressure amplitude is unlikely to be as large as it is in Mic 1 since some energy will be lost as the wave
travels and attenuate into the medium.

\subsection*{(e)}
First, we will measure the time between two peaks. Mic 1 appears to have a peak at $t=0$, so $\Delta t=0.75$ ms. We know
that the mics are spaced by 1 meter, so the wave speed $c=\frac{1}{0.75/1000}=1333$ m/s. I would guess that the gas in
the tube is Hydrogen, as a quick google search returns its speed of sound is 1330 m/s at 25$^{\circ}$C.

\subsection*{(f)}
A right-traveling 1-D wave is given by: $p(x, t)=\hat{p}e^{j(\omega t-kx)}$. We previously determined that
$\hat{p}=29$Pa. We can calculate $k=\omega/c=1050/1333=0.788$ 1/m. Our final equation for the wave measured by Mic 1
would be:
\begin{align*}
    p(x, t)=29e^{j(1050t-0.788x)}
\end{align*}

\subsection*{(g)}
We can see from our equation for the particle speed that $u(x, t)=\frac{\hat{p}}{\rho_0c}p(x, t)$ will be at a maximum
when $p(x, t)=1$ and our max speed will be $v_{max}=\frac{\hat{p}}{\rho_0c}=\frac{29}{0.0838*1330}=0.26$ m/s (assuming
hydrogen). Integrating $u(x, t)$ adds an $\omega$ back to the equation, meaning the max displacement
$\zeta_{max}=\frac{\hat{p}}{\rho_0c\omega}=\frac{29}{0.0838*1330*1050}=2.5E-4$ m.

\subsection*{(h)}
The resulting sound is a low-mid frequency that would be very annoying to listen to for an extended period of time. The
mic signal plot matches the one given very closely.
\begin{figure}[H]
    \centering
    \includegraphics[width=0.75\linewidth]{figs/hw1fig5.png}
\end{figure} \par

\section*{Problem 3}
A leftward traveling wave is given by:
\begin{align*}
    p(x, t)=\hat{p}e^{j(\omega t +kx)}
\end{align*}
Sub in to Euler's equation:
\begin{align*}
    \rho_0\frac{\partial u}{\partial t}=-\frac{\partial \left(\hat{p}e^{j(\omega t +kx)}\right)}{\partial x}
\end{align*}
Take the derivative with respect to x:
\begin{align*}
    \rho_0\frac{\partial u}{\partial t}=-jk\hat{p}e^{j(\omega t + kx)} \\
    \frac{\partial u}{\partial t}=-j\frac{k\hat{p}}{\rho_0}e^{j(\omega t + kx)}
\end{align*}
Now take the integral with respect to t:
\begin{align*}
    u = -j\frac{k\hat{p}}{j\rho_0\omega}e^{j(\omega t + kx)} \\
    u = -\frac{k}{\rho_0\omega}\hat{p}e^{j(\omega t + kx)} \\
    u(x, t)=-\frac{k}{\rho_0\omega}p(x, t)
\end{align*}
Lastly, we can sub in $k=\omega/c$:
\begin{align*}
    u(x, t)=-\frac{1}{\rho_0c}p(x, t) \\
    p(x, t)=-\rho_0cu(x, t)
\end{align*}
The final expression is the same as what was derived in the lecture with the key difference that the relationship
between pressure and velocity is negative instead of positive. This makes sense as the wave is traveling the opposite
direction. It also could be interpreted as the acoustic impedance being negative.

\section*{Problem 4}
I thought Problem 2 was well structured and was a nice way of following through a bunch of relevant calculations with
some tangible application. I thought Problem 1 was a bit too simple. I don't think anything needs to be changed.

\section*{Matlab Code}
\begin{verbatim}
clear variables; close all;
fs = 8192;
dt = 1/fs;
t = 0:dt:5;

Vs = 0.9*29*exp(j*1050*t);

figure()
plot(t, Vs);
xlabel("Time (ms)"); ylabel("Signal (V)");
grid on;

sound(real(Vs))
\end{verbatim}
\end{document}