\documentclass[12 pt]{article}
\usepackage[utf8]{inputenc}
\usepackage{graphicx}
\usepackage{amsmath}
\usepackage[version=4]{mhchem}
\usepackage{siunitx}
\usepackage{longtable,tabularx}
\usepackage{float}
\usepackage[left=1in, right=1in]{geometry}

\title{BME6013C HW\#6}
\date{10.30.25}
\author{Slade Brooks \\ M13801712}

\begin{document}
\maketitle

\section*{Part 1}
We know from the convolution theorem that:
\begin{align*}
    g(t)*h(t)=\int_{-\infty}^{\infty}g(\tau)h(t-\tau)d\tau
\end{align*}
Which means that for our given identity we can determine an equation for $u(t)$ and then use the definition of
convolution to get an expression for the Fourier Transform.
\begin{align*}
    u(t)=u(t)*v(t) \\
    U(f) = U(f)\cdot V(f)
\end{align*}
This shows that clearly the F.T. of $v(t)$ must be equal to 1 in order for this definition to be valid. The most obvious
option for a function for $V(f)$ that could only be 1 and has a known F.T. would be the Heaviside function.
\begin{align*}
    V(f) = \mathcal{H}(f_0/2 - |f|)
\end{align*}
The Heaviside will be 1 when the term inside is greater than or equal to 0. This means we want the period $f_0$ to
always be larger than the frequency variable $f$. This could be achieved by forcing $f_0$ to be infinity, which in
practice would be taking the limit as $f_0$ goes to infinity. Then, we can take the inverse F.T. (which we know from the
F.T. table) to get an expression for $v(t)$.
\begin{align*}
    V(f) = \lim_{f_0\rightarrow\infty}\mathcal{H}(f_0/2-|f|) \\
    v(t) = \lim_{f_0\rightarrow\infty}\frac{\sin{(\pi f_0 t)}}{\pi t}
\end{align*}

\section*{Part 2}
First we will try to take the F.T. of $r_{uv}(t)$ using the F.T. definition:
\begin{align*}
    \mathcal{F}[r_{uv}(t)]=\int_{-\infty}^{\infty}e^{-j2\pi ft}r_{uv}(t)dt=\int_{-\infty}^{\infty}e^{-j2\pi ft}
    \int_{-\infty}^{\infty}[u(\tau)]^*v(t+\tau)d\tau dt
\end{align*}
We can reorganize for convenience:
\begin{align*}
    \mathcal{F}[r_{uv}(t)]=\int_{-\infty}^{\infty}[u(\tau)]^*\left(\int_{-\infty}^{\infty}e^{-j2\pi ft}v(t+\tau)dt\right)d\tau 
\end{align*}
Now we will create a subsitution for $t+\tau$ to simplify the inner integral, with $\tau$ now being a constant since it
is inside of the other integral:
\begin{align*}
    t+\tau = \alpha \\
    d\alpha = dt
\end{align*}
\begin{align*}
    \mathcal{F}[r_{uv}(t)]=\int_{-\infty}^{\infty}[u(\tau)]^*\left(\int_{-\infty}^{\infty}e^{-j2\pi f(\alpha-\tau)}v(\alpha)d\alpha\right)d\tau \\
    \mathcal{F}[r_{uv}(t)]=\int_{-\infty}^{\infty}[u(\tau)]^*\left(\int_{-\infty}^{\infty}e^{-j2\pi f\alpha}e^{j2\pi f\tau}
    v(\alpha)d\alpha\right)d\tau \\
    \mathcal{F}[r_{uv}(t)]=\int_{-\infty}^{\infty}e^{j2\pi f\tau}[u(\tau)]^*\left(\int_{-\infty}^{\infty}e^{-j2\pi f\alpha}
    v(\alpha)d\alpha\right)d\tau
\end{align*}
Now, the integral inside is clearly the F.T. of a function $v(\alpha)$. This can be subbed in and then moved outside of
the integral since it does not depend on $t$.
\begin{align*}
    \mathcal{F}[r_{uv}(t)]=V(f)\int_{-\infty}^{\infty}e^{j2\pi f\tau}[u(\tau)]^*d\tau
\end{align*}
Now we can do some manipulation to the remaining exponential term in order to move the conjugate. We know from the
definition of the conjugate that we can write it as the conjugate of $e^{-j}$, then move the conjugate outside of the
integral since both terms have it. This will allows us to see that the integral part is just the definition of the F.T.
again and we can substitute for that expression:
\begin{align*}
    \mathcal{F}[r_{uv}(t)]=V(f)\int_{-\infty}^{\infty}[e^{-j2\pi f\tau}]^*[u(\tau)]^*d\tau \\
    \mathcal{F}[r_{uv}(t)]=V(f)\left[\int_{-\infty}^{\infty}e^{-j2\pi f\tau}u(\tau)d\tau\right]^* \\
    \mathcal{F}[r_{uv}(t)]=V(f)[U(f)]^*
\end{align*}
Therefore:
\begin{align*}
    \mathcal{F}[r_{uv}(t)]=\mathcal{F}[u(t)]^*\cdot\mathcal{F}[v(t)]
\end{align*}

\section*{Part 3}
From the definition for the DFT:
\begin{align*}
    G[k]=\sum_{n=0}^{N-1}g[n]e^{-j\frac{2\pi kn}{N}} \\
    G[-k] = \sum_{n=0}^{N-1}g[n]e^{-j\frac{2\pi (-k)n}{N}}=\sum_{n=0}^{N-1}g[n]e^{j\frac{2\pi kn}{N}}
\end{align*}
If we are to take the conjugate of $G[-k]$, we will simply flip the sign of the complex part, which in this case will
just put the negative back in front of $j$. This is clearly the same as the definition of the DFT, giving
$G[-k]^*=G[k]$.
\begin{align*}
    G[-k]^*=\sum_{n=0}^{N-1}g[n]e^{-j\frac{2\pi kn}{N}}=G[k]
\end{align*}
We will once again start from the DFT definition to prove the second statement.
\begin{align*}
    G[k]=\sum_{n=0}^{N-1}g[n]e^{-j\frac{2\pi kn}{N}} \\
    G[N-k] =\sum_{n=0}^{N-1}g[n]e^{-j\frac{2\pi (N-k)n}{N}}=\sum_{n=0}^{N-1}g[n]e^{-j\frac{2\pi Nn}{N}}e^{-j\frac{2\pi (-k)n}{N}}
\end{align*}
Now we can simplify and take the conjugate:
\begin{align*}
    G[N-k] = \sum_{n=0}^{N-1}g[n]e^{-j2\pi n}e^{j\frac{2\pi kn}{N}} \\
    G[N-k]^*=\sum_{n=0}^{N-1}g[n]e^{j2\pi n}e^{-j\frac{2\pi kn}{N}}
\end{align*}
Since $n$ is only integers, the exponent of the term $e^{j2\pi n}$ will always be an integer multiple of $2\pi$, which
means the magnitude will always be $e^0=1$. Therefore, that term can be ignored. We again see that the final expression
is the definition of the DFT.
\begin{align*}
    G[N-k]^*=\sum_{n=0}^{N-1}g[n]e^{-j\frac{2\pi kn}{N}}=G[k]
\end{align*}

\end{document}