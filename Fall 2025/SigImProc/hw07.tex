\documentclass[12 pt]{article}
\usepackage[utf8]{inputenc}
\usepackage{graphicx}
\usepackage{amsmath}
\usepackage[version=4]{mhchem}
\usepackage{siunitx}
\usepackage{longtable,tabularx}
\usepackage{float}
\usepackage[left=1in, right=1in]{geometry}

\title{BME6013C HW\#7}
\date{11.20.25}
\author{Slade Brooks \\ M13801712}

\begin{document}
\maketitle

\section*{Part 1}
Since the 2D convolution was defined in the manner given, we can use the convolution theorem similarly to how we would
for a 1D convolution.
\begin{align*}
    u(x,y)*v(x,y)=U(x', y')\cdot V(x', y')
\end{align*}
From our notes, we know the F.T. of a 2D gaussian. We can multiply to make them a Gaussian pulse then use the known F.T.:
\begin{align*}
    g(x,y)=\frac{2\pi\sigma_1^2}{2\pi\sigma_1^2}e^{-(x^2+y^2)/(2\sigma_1^2)}*\frac{2\pi\sigma_2^2}{2\pi\sigma_2^2}e^{-(x^2+y^2)/(2\sigma_2^2)} \\
    U(x', y')\cdot V(x', y')=2\pi\sigma_1^2e^{-2\pi^2\sigma_1^2(x'^2+y'^2)}\cdot 2\pi\sigma_2^2e^{-2\pi^2\sigma_2^2(x'^2+y'^2)} \\
    U(x', y')\cdot V(x', y')=(2\pi)^2\sigma_1^2\sigma_2^2e^{-2\pi^2(\sigma_1^2+\sigma_2^2)(x'^2+y'^2)}
\end{align*}
Now to get $g(x,y)$ we can take the inverse 2D F.T. of this result. We can see that it resembles the F.T. of a guassian
pulse with some scaling, so we can separate the scale factor and take the inverse F.T. with the known definition for a
2D gaussian pulse.
\begin{align*}
    U(x', y')\cdot V(x', y')=(2\pi)^2\sigma_1^2\sigma_2^2e^{-2\pi^2(\sigma_1^2+\sigma_2^2)x'^2}e^{-2\pi^2(\sigma_1^2+\sigma_2^2)y'^2} \\
    g(x, y)=\mathcal{F}^{-1}[U(x', y')\cdot V(x', y')]=(2\pi)^2\sigma_1^2\sigma_2^2\frac{e^{-(x^2+y^2)/2(\sigma_1^2+\sigma_2^2)}}{2\pi(\sigma_1^2 + \sigma_2^2)}
\end{align*}
Therefore:
\begin{align*}
    \fbox{$g(x,y)=\frac{2\pi\sigma_1^2\sigma_2^2}{\sigma_1^2+\sigma_2^2}e^{\frac{-(x^2+y^2)}{2(\sigma_1^2+\sigma_2^2)}}$}
\end{align*}

\section*{Part 2}
First, we will start with the definition for $g[m,n]$ and sub in $m+M$ and $n+N$.
\begin{align*}
    g[m+M,n+N]=\frac{1}{MN}\sum_{p=0}^{M-1}\sum_{q=0}^{N-1}e^{j\pi(m+M)p/M}e^{j\pi(n+N)q/N}G[p,q]H[p,q] \\
    g[m+M,n+N]=\frac{1}{MN}\sum_{p=0}^{M-1}\sum_{q=0}^{N-1}e^{j\pi mp/M}e^{j\pi p}e^{j\pi nq/N}e^{j\pi q}G[p,q]H[p,q]
\end{align*}
We know that $p$ and $q$ are integers, which means the two terms $e^{j\pi p}$ and $e^{j\pi q}$ will always be integer
multiple of $e^{j\pi}=-1$. We can subsitute in $-1$ for both of those terms, giving:
\begin{align*}
    g[m+M,n+N]=\frac{1}{MN}\sum_{p=0}^{M-1}\sum_{q=0}^{N-1}e^{j\pi mp/M}(-1)e^{j\pi nq/N}(-1)G[p,q]H[p,q]
\end{align*}
We can see that this simplifies to $1$, and therefore $g[m+M, n+N]$ has the same expression as our original definition
for $g[m, n]$.
\begin{align*}
    g[m+M,n+N]=\frac{1}{MN}\sum_{p=0}^{M-1}\sum_{q=0}^{N-1}e^{j\pi mp/M}e^{j\pi nq/N}G[p,q]H[p,q]=g[m, n]
\end{align*}

\section*{Part 3}
The power spectrum is simply the F.T. of the autocorrelation function. We will use the integral definition of the F.T.:
\begin{align*}
    \mathcal{F}[r_{gg}(\tau)]=\int_{-\infty}^{\infty}e^{-j2\pi f\tau}(1-|\tau|)d\tau
\end{align*}
We know that the integral is only defined for $\tau=-1\rightarrow1$, but since it is the absolute value of $\tau$, we
can say the integral is:
\begin{align*}
    \mathcal{F}[r_{gg}(\tau)]=2\int_{0}^{1}e^{-j2\pi f\tau}(1-\tau)d\tau \\
\end{align*}
We know that $e^{-j2\pi f\tau}$ will only have real numbers:
\begin{align*}
    \mathcal{F}[r_{gg}(\tau)]=2\int_{0}^{1}\cos{(2\pi f\tau)}(1-\tau)d\tau=
    2\int_{0}^{1}\cos{(2\pi f\tau)}d\tau-2\int_{0}^{1}\tau\cos{(2\pi f\tau)}d\tau \\
    =2\left[\frac{\sin{(2\pi f\tau)}}{2\pi f}\right]_0^1-2\left[\frac{\tau\sin{(2\pi f\tau)}}{2\pi f}|_0^1
    -\frac{1}{2\pi f}\int_{0}^{1}\sin{(2\pi f\tau)}d\tau\right] \\
    =2\left(\frac{\sin{(2\pi f)}}{2\pi f}\right)-2\left[\frac{\sin{(2\pi f)}}{2\pi f}
    -\frac{1}{2\pi f}\frac{-\cos{(2\pi f\tau)}}{2\pi f}|_0^1\right] \\
    =2\left(\frac{\sin{(2\pi f)}}{2\pi f}\right)-2\left[\frac{\sin{(2\pi f)}}{2\pi f}-
    \left(\frac{-\cos{(2\pi f)}}{(2\pi f)^2}-\frac{-1}{(2\pi f)^2}\right)\right] \\
    =-2\left(\frac{\cos{(2\pi f)}-1}{(2\pi f)^2}\right)
\end{align*}
We can use the trig half-angle identities to get a sinc function:
\begin{align*}
    \sin^2{(\pi f)}=\frac{1-\cos{2\pi f}}{2} \\
    s_{gg}(f)=4\left(\frac{1-\cos{(2\pi f)}}{2(2\pi f)^2}\right)=4\frac{\sin^2{(\pi f)}}{(2\pi f)^2}
    =\frac{4\sin^2{(\pi f)}}{4(\pi f)^2} \\
    s_{gg}(f)=\left(\frac{\sin{(\pi f)}}{\pi f}\right)^2
\end{align*}
We can see the result is a sinc function squared. Gaussian
white noise has a flat power spectra, and its autocorrelation function is a sinc. This signal has a power
spectrum that is a sinc squared. Gaussian white noise is an incredibly random signal with lots of high frequency content
since it has power content across all frequencies evenly. However, the signal we measured should be much smoother. Since
the power spectrum is a sinc and not a flat line, it is essentially like a low-passed version of the gaussian white
noise. It should be a random signal in the time-domain, but appear much smoother since the high-frequency is attenuated.

\end{document}