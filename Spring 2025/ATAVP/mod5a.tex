\documentclass[journal]{new-aiaa}
%\documentclass[conf]{new-aiaa} for conference papers
\usepackage[utf8]{inputenc}
\usepackage{textcomp}

\usepackage{graphicx}
\usepackage{amsmath}
\usepackage[version=4]{mhchem}
\usepackage{siunitx}
\usepackage{longtable,tabularx}
\setlength\LTleft{0pt} 

\title{Conceptual Design of Hybrid-Electric Aircraft}

\author{Slade Brooks}

\begin{document}

\maketitle

\section{Introduction}
The advent of electric propulsion in the aviation industry has led to the emergence of new types of aircraft propulsion
systems and power architectures. While many fully-electric aircraft concepts exist, hybrid-electric systems are also
becoming popular. Hybrid-electric propulsion systems seek to solve some of the initial problems that electric aircraft
face. Namely, the low energy density of current batteries makes electric aircraft severely range limited. Many different
hybrid-electric architectures exist to try to combat this issues and deliver long range performance with reduced
emissions and while retaining the benefits of electric propulsion systems. Selecting an appropriate hybrid-electric
system for a given aircraft design is vital to the performance of an aircraft system. This selection must happen as part
of conceptual design. Electric aircraft design and initial sizing is tied closely to the power generation and propulsion
system, so it must be selected as early as possible. Since the industry is so new, methods for designing hybrid and
electric aircraft systems are still being developed.

\section{Types of Hybrid Systems}
While most hybrid architectures aim to employ the benefits of electric propulsion in some way while still using the high
energy density of hydrocarbon fuels, there are multiple methods for achieving this balance. The following sections
detail the different types of hybrid-electric systems, their drawbacks and benefits, and some concepts employing them.
Each type has a unique application for aviation propulsion, and selecting an appropriate type for a given aircraft
concept will be vitally important to its success.

\subsection{Series Hybrid}
The first type of system is a series hyrbid architecture. In a series hybrid system, electric motors for propulsion are
driven by both batteries and a generator. An engine like a turboshaft burns fuel and turns a generator to generate some
electricity, and supplements the energy stored by dedicated battery packs. Electra's technical demonstrator, the EL-2
``Goldfinch'', has a series hybrid propulsion system. Figure~\ref{fig:el2} shows an image of the EL-2. It uses
distributed propellers driven by electric motors for propulsion. It has both a battery pack in the floor of the aircraft
and a turbogenerator in the nose to increase range.
\begin{figure}
    \centering
    \includegraphics[height=3in]{el2.jpg}
    \caption{Image of the Electra EL-2 ``Goldfinch'' demonstrating short takeoff capabilities.}
    \label{fig:el2}
\end{figure}
The main benefit of a series hybrid system is that it essentially increases the energy density of the batteries. By
carrying a generator, an electric aircraft can greatly increase its range while still having reduced emissions. However,
bringing a generator is more weight, complexity, and failure points. While the range is likely to be increased greatly,
the aircraft would have to have additional space and weight allowance for fuel tanks, the generator, auxiliary systems,
gearboxes, and more.

\subsection{Parallel Hybrid}
A parallel hybrid architecture employs a fuel-burning engine as the main propulsion source. It supplements this
propulsion in some way with a battery pack driving an electric motor coupled with the drive shaft. Astro Mechanica is a
company that is trying to employ a parallel hybrid system on a jet engine. By using two electric motors, one for the
compressor and one for the turbine, they hope to increase the efficiency of jet engines across a range of flight
conditions by being able to operate the turbomachinery separately at their ideal operating conditions. While parallel
architecture may have large efficiency benefits for a hyrdocarbon burning engine, its emissions reductions are
relatively low. An electric propulsion system will typically have much more emissions savings potential than a parallel
hybrid system that relies on fuel being burned for propulsion, not just energy. Furthermore, it requires some battery
pack to drive the electric powertrain, which is additional weight that may be useless during some phases of flight.

\subsection{Turboelectric}
A turboelectric system is one where electric propulsion is used, but no battery packs exist. A turboshaft or other
similar engine is used as a generator, which directly feeds the electric motors. While few concepts employ fully
turboelectric systems, partially turboelectric concepts do exist. These aircraft have some propulsion generated via
fuel-burning engines, and some generated with electric motors that are powered by the fuel-burning engines. These
concepts typically use electric propulsion during one phase of flight, but do not need to carry heavy battery packs due
to the selection of a turboelectric architecture. While turboelectric benefits from not requiring auxiliary systems like
battery packs, it is not as fault tolerant as other hybrid archictectures. With no backup battery system or other
sources of power, losing the generator could lead to total loss of propulsion almost instantly.

\subsection{Selecting a Hybrid-Electric Architecture}
\citeauthor{zamboni2019} describes the advantages and disadvantages of each type of hybrid-electric system in detail. It
is described that while some systems like parallel architecture are the lightest and easiest to retrofit, integrating a
series system early in the design could allow for more benefits to be taken advantage of \cite{zamboni2019}. This is
also true of other architectures that may be overlooked in literature since there are few clean sheet hybrid aircraft
designs with enough public data to provide valuable research. Furthermore, technology maturity has a large impact on the
viability of different types of systems. The actual benefit of a hybrid system and the degree of which hybridization is
useful changes as technology assumptions change \cite{zamboni2019}. For this reason, new conceptual design techniques are
being developed to assist designers with evaluating the merits of various hybrid-electric systems on conceptual designs.

\section{Hybrid-Electric Aircraft Conceptual Design Methods}

\subsection{Traditional Aircraft Design Methods}
Currently, conceptual design methods specific to hybrid and electric aircraft are still being developed. While they are
being created and validated, traditional aircraft design methods have been employed in the meantime.
\citeauthor{sziroczak2020} explains that analysis of radical new technologies is most often evaluated with ``fundamental
aspects of aircraft design'', allowing for exploration of the design space. Slight modifications to existing design
tools may be made to allow for analysis of unconventional aircraft systems. \citeauthor{finger2020} describes an
existing iterative solver for aircraft design that simulates a mission, calculates relevant system masses, and
determines a converged solution. This system was modified to function with hybrid electric aircraft by defining some
energy to be consumable and some to be non-consumable. This method was compared to a reference aircraft to determine its
accuracy. Figure~\ref{fig:diagram} shows an example of a standard aircraft design method that has been adapted to
electric and hybrid-electric aircraft configurations.
\begin{figure}
    \centering
    \includegraphics[height=3in]{diagram.png}
    \caption{The conventional conceptual design methodology adapted to the electric and hybrid aircraft configurations \cite{sziroczak2020}.}
    \label{fig:diagram}
\end{figure}

\subsection{Custom Design Methods}
Custom conceptual design methods are still being developed for hybrid and electric aircraft. Both
\citeauthor{sziroczak2020} and \citeauthor{finger2020} also describe custom design methodologies. One common practice is
highly customizing existing software. Changing the methods used for calculating mass fractions, setting new constraints
for energy burn, integrating hybridization power splits, and fully integrating electric propulsion and battery modeling
are all techniques being used to turn existing software into new conceptual design methodologies \cite{sziroczak2020}.
Some fully custom methods are also being developed. \citeauthor{finger2020} describes a custom methodology created to
analyze novel propulsion systems and hybrid architecture. It follows the structure of an existing design tool, but uses
custom methods for designing the aircraft. Aero-propulsive interactions and electric powertrain components are both
featured in early sizing to ensure system integration as the concept matures \cite{finger2020}. This and similar custom
methods are being developed currently to serve the design market. As the industry evolves, new methods will show their
potential to aid in the design process.

\subsection{Multidisciplinary Design Optimization}
Multidisciplinary design optimization (MDO) employs optimization methods and multiple disciplines within aircraft design to
solve an aircraft design problem. MDO is commonly applied in aircraft design to balance various design variables to
determine the ``best'' possible design given the constraints. \citeauthor{silva2021} describes how a custom MDO
framework was created to analyze hybrid-electric aircraft. An iterative design process and genetic algorithm is used to
solve for the ideal aircraft design. A series of constraints on the aircraft are created. Then the geometry,
aerodynamics, static stability, trim conditions, structures, and mission are all determined and analyzed within the
framework, shown in Fig.~\ref{fig:diag2}.
\begin{figure}
    \centering
    \includegraphics[height=3in]{diag2.png}
    \caption{MDO design structure matrix \cite{silva2021}.}
    \label{fig:diag2}
\end{figure}
Each portion of the aircraft is sized in the framework and how they are
all connected together for analysis. These methods were validated with existing aerodynamics and structures solvers. The
custom MDO method was shown to be accurate when compared to traditional methods \cite{silva2021}. MDO has proved to be a
robust and effective tool for conventional aircraft design, and will likely continue to be a hallmark of aircraft design
software for hybrid-electric systems.

\section{Conclusion}
The emergence of hybrid-electric propulsion systems represents a critical evolution in the aviation industry, bridging
the gap between conventional fuel-based propulsion and fully electric flight. These systems offer promising solutions to
challenges such as limited battery energy density and emissions reduction. With multiple architectures each offering
unique trade-offs, selecting the appropriate system during conceptual design is
essential for optimizing aircraft performance. As the field evolves, traditional design tools are being adapted and new
methodologies developed to accurately evaluate hybrid-electric configurations. Continued research and innovation will be
key in refining these approaches and advancing sustainable aviation technology.

\bibliography{sample}

\end{document}