\documentclass{article}
\usepackage{graphicx}
\usepackage{amsmath}
\usepackage{multirow}
\graphicspath{{./images/hw4/}}
\usepackage[legalpaper, portrait, margin=1in]{geometry}

\title{AEEM4058 - Homework 4}
\author{Slade Brooks}
\date{10.05.2023}

\begin{document}
\maketitle

\section*{Problem 1}
Figure 1 shows a truss structure with two uniform members made of same material.
The truss structure is constrained at two ends. The cross-sectional area of all the
truss members is $0.01$ m$^2$, and the Young’s modulus of
the material is $2.0E10$ N/m$^2$. Using the finite element method, calculate \\
(a) all the nodal displacements; \\
(b) the internal forces in all the truss members; and \\
(c) the reaction forces at the supports.
\begin{center}
    \includegraphics[scale=0.75]{fig1} \includegraphics*[scale=0.5]{fig11} \\
    Figure 1
\end{center}
\subsection*{Part A}
\begin{tabular}{|c|c|c|c|c|c|c|}
    \hline
    & \multicolumn{2}{|c|}{Global node corresponding to} &
    \multicolumn{2}{|c|}{Coordinates in global} & 
    \multicolumn{2}{|c|}{Direction cosines} \\
    \hline
    Element \# & local node 1 & local node 2 & $X_i, Y_i$ & $X_j, Y_j$ & $l_{ij}$ & $m_{ij}$ \\
    \hline
    1 & 1 & 2 & 0, 0 & 1, 0 & 1 & 0 \\
    2 & 2 & 3 & 1, 0 & 2, -1 & $\frac{1}{\sqrt{2}}$ & $\frac{-1}{\sqrt{2}}$ \\
    \hline
\end{tabular}
\subsubsection*{Build Element Matrices}
$K_e=\frac{AE}{l_e}
\begin{bmatrix}
    l_{ij}^2 & l_{ij}m_{ij} & -l_{ij}^2 & -l_{ij}m_{ij} \\
    & m_{ij}^2 & -l_{ij}m_{ij} & -m_{ij}^2 \\
    & & l_{ij}^2 & l_{ij}m_{ij} \\
    sym. & & & m_{ij}^2
\end{bmatrix}$ \\\\
$K_{e1}=\frac{0.01(2E10)}{1}\begin{bmatrix}
    1 & 0 & -1 & 0 \\
    & 0 & 0 & 0 \\
    & & 1 & 0 \\
    sym. & & & 0
\end{bmatrix}=\begin{bmatrix}
    2 & 0 & -2 & 0 \\
    & 0 & 0 & 0 \\
    & & 2 & 0 \\
    sym. & & & 0
\end{bmatrix}*10^8$ N/m \\\\
$K_{e2}=\frac{0.01(2E10)}{\sqrt{2}}\begin{bmatrix}
    0.5 & -0.5 & -0.5 & 0.5 \\
    & 0.5 & 0.5 & -0.5 \\
    & & 0.5 & -0.5 \\
    sym. & & & 0.5
\end{bmatrix}=\begin{bmatrix}
    1/\sqrt{2} & -1/\sqrt{2} & -1/\sqrt{2} & 1/\sqrt{2} \\
    & 1/\sqrt{2} & 1/\sqrt{2} & -1/\sqrt{2} \\
    & & 1/\sqrt{2} & -1/\sqrt{2} \\
    sym. & & & 1/\sqrt{2}
\end{bmatrix}*10^8$ N/m

\subsubsection*{Build Global Matrices}
combine $K_{e1}$ and $K_{e2}$: \\\\
$K=10^8*\begin{bmatrix}
    2 & 0 & -2 & 0 & 0 & 0 \\
    0 & 0 & 0 & 0 & 0 & 0 \\
    -2 & 0 & 2 & 0 & 0 & 0 \\
    0 & 0 & 0 & 0 & 0 & 0 \\
    0 & 0 & 0 & 0 & 0 & 0 \\
    0 & 0 & 0 & 0 & 0 & 0
\end{bmatrix}\rightarrow K=10^8*\begin{bmatrix}
    2 & 0 & -2 & 0 & 0 & 0 \\
    0 & 0 & 0 & 0 & 0 & 0 \\
    -2 & 0 & 2+1/\sqrt{2} & -1/\sqrt{2} & -1/\sqrt{2} & 1/\sqrt{2} \\
    0 & 0 & -1/\sqrt{2} & 1/\sqrt{2} & 1/\sqrt{2} & -1/\sqrt{2} \\
    0 & 0 & -1/\sqrt{2} & 1/\sqrt{2} & 1/\sqrt{2} & -1/\sqrt{2} \\
    0 & 0 & 1/\sqrt{2} & -1/\sqrt{2} & -1/\sqrt{2} & 1/\sqrt{2}
\end{bmatrix}$ \\\\
set up F and D matrices from boundary conditions: \\
$F=\begin{bmatrix}
    R_{D1} \\ R_{D2} \\ 0 \\ -100 \\ R_{D5} \\ R_{D6}
\end{bmatrix}$, \quad
$D=\begin{bmatrix}
    0 \\ 0 \\ D3 \\ D4 \\ 0 \\ 0
\end{bmatrix}$

\subsubsection*{Solve for displacements}
$KD=F$ \\
$10^8*\begin{bmatrix}
    2 & 0 & -2 & 0 & 0 & 0 \\
    0 & 0 & 0 & 0 & 0 & 0 \\
    -2 & 0 & 2+1/\sqrt{2} & -1/\sqrt{2} & -1/\sqrt{2} & 1/\sqrt{2} \\
    0 & 0 & -1/\sqrt{2} & 1/\sqrt{2} & 1/\sqrt{2} & -1/\sqrt{2} \\
    0 & 0 & -1/\sqrt{2} & 1/\sqrt{2} & 1/\sqrt{2} & -1/\sqrt{2} \\
    0 & 0 & 1/\sqrt{2} & -1/\sqrt{2} & -1/\sqrt{2} & 1/\sqrt{2}
\end{bmatrix}\begin{bmatrix}
    0 \\ 0 \\ D3 \\ D4 \\ 0 \\ 0
\end{bmatrix}=\begin{bmatrix}
    R_{D1} \\ R_{D2} \\ 0 \\ -100 \\ R_{D5} \\ R_{D6}
\end{bmatrix}$ \\\\
use Matlab to solve for displacements at node 2 (D3 and D4): \\
$\begin{bmatrix}
    -2 & 0 & 2+1/\sqrt{2} & -1/\sqrt{2} & -1/\sqrt{2} & 1/\sqrt{2} \\
    0 & 0 & -1/\sqrt{2} & 1/\sqrt{2} & 1/\sqrt{2} & -1/\sqrt{2}
\end{bmatrix}\begin{bmatrix}
    D3 \\ D4
\end{bmatrix}=\begin{bmatrix}
    0 \\ -100
\end{bmatrix}$ \\\\
$\fbox{
$\begin{bmatrix}
    0 \\ 0 \\ -5*10^{-7} \\ -1.9142*10^{-6} \\ 0 \\ 0
\end{bmatrix}$ m}$

\subsection*{Part B}
$F_e=\frac{AE}{l_e}\begin{bmatrix}
    -l_{ij} & -m_{ij} & l_{ij} & m_{ij}
\end{bmatrix}\begin{bmatrix}
    D1 \\ D2 \\ D3 \\ D4
\end{bmatrix}$ \\\\\\
$F_1=\frac{0.01(2E10)}{1}\begin{bmatrix}
    -1 & 0 & 1 & 0
\end{bmatrix}\begin{bmatrix}
    0 \\ 0 \\ -5*10^{-7} \\ -1.9142*10^{-6}
\end{bmatrix}$ \\
solve with matlab for F: \\
$\fbox{$F_1=-100$ N}$ \\\\\\
$F_2=\frac{0.01(2E10)}{\sqrt{2}}\begin{bmatrix}
    -\frac{1}{\sqrt{2}} & \frac{1}{\sqrt{2}} & \frac{1}{\sqrt{2}} & -\frac{1}{\sqrt{2}}
\end{bmatrix}\begin{bmatrix}
    -5*10^{-7} \\ -1.9142*10^{-6} \\ 0 \\ 0
\end{bmatrix}$ \\
solve with matlab for F: \\
$\fbox{$F_2=-141.42$ N}$

\subsection*{Part C}
plug in known values for displacements into KD=F: \\
$10^8*\begin{bmatrix}
    2 & 0 & -2 & 0 & 0 & 0 \\
    0 & 0 & 0 & 0 & 0 & 0 \\
    -2 & 0 & 2+1/\sqrt{2} & -1/\sqrt{2} & -1/\sqrt{2} & 1/\sqrt{2} \\
    0 & 0 & -1/\sqrt{2} & 1/\sqrt{2} & 1/\sqrt{2} & -1/\sqrt{2} \\
    0 & 0 & -1/\sqrt{2} & 1/\sqrt{2} & 1/\sqrt{2} & -1/\sqrt{2} \\
    0 & 0 & 1/\sqrt{2} & -1/\sqrt{2} & -1/\sqrt{2} & 1/\sqrt{2}
\end{bmatrix}\begin{bmatrix}
    0 \\ 0 \\ -5*10^{-7} \\ -1.9142*10^{-6} \\ 0 \\ 0
\end{bmatrix}=\begin{bmatrix}
    R_{D1} \\ R_{D2} \\ 0 \\ -100 \\ R_{D5} \\ R_{D6}
\end{bmatrix}$ \\\\\\
solve for each missing force in F with matlab:
$\fbox{$\begin{bmatrix}
    R_{D1} \\ R_{D2} \\ R_{D5} \\ R_{D6}
\end{bmatrix}=\begin{bmatrix}
    100 \\ 0 \\ -100 \\ 100
\end{bmatrix}$ N}$

\section*{Problem 2}
Figure 2 shows a three-node truss element of length $L$ and a constant cross-section area
$A$. It is made of a material of Young’s modulus $E$ and density $\rho$. The truss is
subjected to a uniformly distributed force $b$. \\
(a) Derive the stiffness matrix for the element. \\
(b) Write down the expression for the element mass matrix, and obtain $m_{11}$ in terms of
$L$, $E$, $\rho$, and $A$. \\
(c) Derive the external force vector.
\begin{center}
    \includegraphics[]{fig2} \includegraphics[scale=0.5]{fig22} \\
    Figure 2
\end{center}
\subsection*{Part A}
$N_0(\xi)=-\frac{1}{2}\xi(1-\xi)=-\frac{1}{2}\xi+\frac{1}{2}\xi^2$ \\
$N_1(\xi)=(1+\xi)(1-\xi)=1-\xi^2$ \\
$N_2(\xi)=\frac{1}{2}\xi(1+\xi)=\frac{1}{2}\xi+\frac{1}{2}\xi^2$ \\\\
$a\xi=x \rightarrow d\xi=\frac{1}{a}dx \rightarrow \frac{d\xi}{dx}=\frac{1}{a}$ \\\\
$K=\int_{V}B^TcBdV=A\int_{-L/2}^{L/2}B^TEBdx$ \\
$B=\frac{dN}{dx}=\frac{dN}{d\xi}\frac{d\xi}{dx}=\frac{dN}{d\xi}\frac{1}{a}$ \\
$B=\frac{1}{dx}\begin{bmatrix}
    dN_1 & dN_2 & dN_3
\end{bmatrix}$ \\
$\frac{dN_1}{d\xi}=-\frac{1}{2}+\xi$, \quad $\frac{dN_2}{d\xi}=-2\xi$, \quad $\frac{dN_3}{d\xi}
=\frac{1}{2}+\xi$ \\
$B=\frac{1}{a}\begin{bmatrix}
    -\frac{1}{2}+\xi & -2\xi & \frac{1}{2}+\xi
\end{bmatrix}$ \\\\
$K=EA\int_{-1}^{1}B^TB ad\xi$ \\
$B^TB=\frac{1}{a^2}\begin{bmatrix}
    -\frac{1}{2}+\xi & -2\xi & \frac{1}{2}+\xi
\end{bmatrix}\begin{bmatrix}
    -\frac{1}{2}+\xi \\ -2\xi \\ \frac{1}{2}+\xi
\end{bmatrix}=\frac{1}{a^2}\begin{bmatrix}
    \frac{1}{4}-\xi+\xi^2 & \xi-2\xi^2 & -\frac{1}{4}+\xi^2 \\
    & 4\xi^2 & -\xi-2\xi^2 \\
    sym. & & \frac{1}{4}+\xi+\xi^2
\end{bmatrix}$ \\\\\\
integrate and ignore the odd power terms of $\xi$ since they will cancel: \\\\
$K=\frac{EA}{a}\int_{-1}^{1}\begin{bmatrix}
    \frac{1}{4}-\xi+\xi^2 & \xi-2\xi^2 & -\frac{1}{4}+\xi^2 \\
    & 4\xi^2 & -\xi-2\xi^2 \\
    sym. & & \frac{1}{4}+\xi+\xi^2
\end{bmatrix} d\xi=\frac{EA}{a}\begin{bmatrix}
    \frac{1}{4}\xi+\frac{1}{3}\xi^3 & -\frac{2}{3}\xi^3 & -\frac{1}{4}\xi+\frac{1}{3}\xi^3 \\
    & \frac{4}{3}\xi^3 & -\frac{2}{3}\xi^3 \\
    sym. & & \frac{1}{4}\xi+\frac{1}{3}\xi^3
\end{bmatrix}_{-1}^1$ \\
then evaluate from 0 to 1 and multiply by 2 since those terms are gone: \\\\
$K=\frac{2EA}{a}\begin{bmatrix}
    \frac{1}{4}\xi+\frac{1}{3}\xi^3 & -\frac{2}{3}\xi^3 & -\frac{1}{4}\xi+\frac{1}{3}\xi^3 \\
    & \frac{4}{3}\xi^3 & -\frac{2}{3}\xi^3 \\
    sym. & & \frac{1}{4}\xi+\frac{1}{3}\xi^3
\end{bmatrix}_{0}^1=\frac{2EA}{a}\begin{bmatrix}
    \frac{1}{4}+\frac{1}{3} & -\frac{2}{3} & -\frac{1}{4}+\frac{1}{3} \\
    & \frac{4}{3} & -\frac{2}{3} \\
    sym. & & \frac{1}{4}+\frac{1}{3}
\end{bmatrix}$ \\\\
then plug in $a=L/2$: \\
$\fbox{$K=\frac{EA}{L}\begin{bmatrix}
    \frac{7}{3} & -\frac{8}{3} & \frac{1}{3} \\
    & \frac{16}{3} & -\frac{8}{3} \\
    sym. & & \frac{7}{3}
\end{bmatrix}$}$

\subsection*{Part B}
$m_e=\int_{V_e}\rho N^TNdV=A\rho a\int_{-1}^{1}N^TNd\xi$ \\
$N^TN=\begin{bmatrix}
    -\frac{1}{2}\xi+\frac{1}{2}\xi^2 \\ 1-\xi^2 \\ \frac{1}{2}\xi+\frac{1}{2}\xi^2
\end{bmatrix}\begin{bmatrix}
    -\frac{1}{2}\xi+\frac{1}{2}\xi^2 & 1-\xi^2 & \frac{1}{2}\xi+\frac{1}{2}\xi^2
\end{bmatrix}=\\ \begin{bmatrix}
    \frac{1}{4}\xi^2-\frac{1}{2}\xi^3+\frac{1}{4}\xi^4 &
    -\frac{1}{2}\xi+\frac{1}{2}\xi^2+\frac{1}{3}\xi^3-\frac{1}{2}\xi^4 &
    -\frac{1}{4}\xi^2+\frac{1}{4}\xi^4 \\
    & 1-2\xi^2+\xi^4 & \frac{1}{2}\xi+\frac{1}{2}\xi^2-\frac{1}{2}\xi^3-\frac{1}{2}\xi^4 \\
    sym. & & \frac{1}{4}\xi^2+\frac{1}{2}\xi^3+\frac{1}{4}\xi^4
\end{bmatrix}$ \\\\\\
integrate and ignore the odd power terms of $\xi$ since they will cancel: \\\\
$m_e=A\rho a\int_{-1}^{1}\begin{bmatrix}
    \frac{1}{4}\xi^2-\frac{1}{2}\xi^3+\frac{1}{4}\xi^4 &
    -\frac{1}{2}\xi+\frac{1}{2}\xi^2+\frac{1}{3}\xi^3-\frac{1}{2}\xi^4 &
    -\frac{1}{4}\xi^2+\frac{1}{4}\xi^4 \\
    & 1-2\xi^2+\xi^4 & \frac{1}{2}\xi+\frac{1}{2}\xi^2-\frac{1}{2}\xi^3-\frac{1}{2}\xi^4 \\
    sym. & & \frac{1}{4}\xi^2+\frac{1}{2}\xi^3+\frac{1}{4}\xi^4
\end{bmatrix}d\xi= \\\\
A\rho a\begin{bmatrix}
    \frac{1}{12}\xi^3+\frac{1}{20}\xi^5 & \frac{1}{6}\xi^3-\frac{1}{10}\xi^5 &
    -\frac{1}{12}\xi^3+\frac{1}{20}\xi^5 \\
    & \xi-\frac{2}{3}\xi^3+\frac{1}{5}\xi^5 & \frac{1}{6}\xi^3-\frac{1}{10}\xi^5 \\
    sym. & & \frac{1}{12}\xi^3+\frac{1}{20}\xi^5
\end{bmatrix}_{-1}^1$ \\
then evaluate from 0 to 1 and multiply by 2 since those terms are gone: \\\\
$m_e=2A\rho a\begin{bmatrix}
    \frac{1}{12}\xi^3+\frac{1}{20}\xi^5 & \frac{1}{6}\xi^3-\frac{1}{10}\xi^5 &
    -\frac{1}{12}\xi^3+\frac{1}{20}\xi^5 \\
    & \xi-\frac{2}{3}\xi^3+\frac{1}{5}\xi^5 & \frac{1}{6}\xi^3-\frac{1}{10}\xi^5 \\
    sym. & & \frac{1}{12}\xi^3+\frac{1}{20}\xi^5
\end{bmatrix}_{0}^1=2A\rho a\begin{bmatrix}
    \frac{1}{12}+\frac{1}{20} & \frac{1}{6}-\frac{1}{10} & -\frac{1}{12}+\frac{1}{20} \\
    & 1-\frac{2}{3}+\frac{1}{5} & \frac{1}{6}-\frac{1}{10} \\
    sym. & & \frac{1}{12}+\frac{1}{20}
\end{bmatrix}$ \\\\
then plug in $a=L/2$: \\
$\fbox{$m_e=A\rho L\begin{bmatrix}
    \frac{2}{15} & \frac{1}{15} & -\frac{1}{30} \\
    & \frac{8}{15} & \frac{1}{15} \\
    sym. & & \frac{2}{15}
\end{bmatrix}$}$ \\\\
find $m_{11}$: \\
$\fbox{$m_{11}=\frac{2A\rho L}{15}$}$

\subsection*{Part C}
$f_e=\int_{V_e}N^Tf_bdV+\int_{S_e}N^Tf_sdS=A\int_{-1}^{1}N^Tf_bad\xi+L\int_{-1}^{1}N^Tf_sad\xi$ \\
$f_e=(Aaf_b+Laf_s)\int_{-1}^{1}N^Td\xi=(Aaf_b+Laf_s)\int_{-1}^{1}\begin{bmatrix}
    -\frac{1}{2}\xi+\frac{1}{2}\xi^2 \\ 1-\xi^2 \\ \frac{1}{2}\xi+\frac{1}{2}\xi^2
\end{bmatrix}d\xi=\\
(Aaf_b+Laf_s)\begin{bmatrix}
    -\frac{1}{4}\xi^2+\frac{1}{6}\xi^3 \\
    \xi+\frac{1}{3}\xi^3 \\
    \frac{1}{4}\xi^2+\frac{1}{6}\xi^3
\end{bmatrix}_{-1}^1=(Aaf_b+Laf_s)\begin{bmatrix}
    -\frac{1}{12}+\frac{5}{12} \\
    \frac{2}{3}+\frac{2}{3} \\
    \frac{5}{12}-\frac{1}{12}
\end{bmatrix}$ \\\\
$\fbox{$f_e=(Aaf_b+Laf_s)\begin{bmatrix}
    \frac{1}{3} \\\\ \frac{4}{3} \\\\ \frac{1}{3}
\end{bmatrix}$}$

\end{document}