\documentclass{article}
\usepackage{graphicx}
\usepackage{amsmath}
\usepackage{multirow}
\graphicspath{{./images/hw4/}}
\usepackage[legalpaper, portrait, margin=1in]{geometry}

\title{AEEM4058 - Homework 4}
\author{Slade Brooks}
\date{10.05.2023}

\begin{document}
\maketitle

\section*{Problem 1}
Figure 1 shows a truss structure with two uniform members made of same material.
The truss structure is constrained at two ends. The cross-sectional area of all the
truss members is $0.01$ m$^2$, and the Young’s modulus of
the material is $2.0E10$ N/m$^2$. Using the finite element method, calculate \\
(a) all the nodal displacements; \\
(b) the internal forces in all the truss members; and \\
(c) the reaction forces at the supports.
\begin{center}
    \includegraphics[scale=0.75]{fig1} \includegraphics*[scale=0.5]{fig11} \\
    Figure 1
\end{center}
\subsection*{Part A}
\begin{tabular}{|c|c|c|c|c|c|c|}
    \hline
    & \multicolumn{2}{|c|}{Global node corresponding to} &
    \multicolumn{2}{|c|}{Coordinates in global} & 
    \multicolumn{2}{|c|}{Direction cosines} \\
    \hline
    Element \# & local node 1 & local node 2 & $X_i, Y_i$ & $X_j, Y_j$ & $l_{ij}$ & $m_{ij}$ \\
    \hline
    1 & 1 & 2 & 0, 0 & 1, 0 & 1 & 0 \\
    2 & 2 & 3 & 1, 0 & 2, -1 & $\frac{1}{\sqrt{2}}$ & $\frac{-1}{\sqrt{2}}$ \\
    \hline
\end{tabular}
\subsubsection*{Build Element Matrices}
$K_e=\frac{AE}{l_e}
\begin{bmatrix}
    l_{ij}^2 & l_{ij}m_{ij} & -l_{ij}^2 & -l_{ij}m_{ij} \\
    & m_{ij}^2 & -l_{ij}m_{ij} & -m_{ij}^2 \\
    & & l_{ij}^2 & l_{ij}m_{ij} \\
    sym. & & & m_{ij}^2
\end{bmatrix}$ \\\\
$K_{e1}=\frac{0.01(2E10)}{1}\begin{bmatrix}
    1 & 0 & -1 & 0 \\
    & 0 & 0 & 0 \\
    & & 1 & 0 \\
    sym. & & & 0
\end{bmatrix}=\begin{bmatrix}
    2 & 0 & -2 & 0 \\
    & 0 & 0 & 0 \\
    & & 2 & 0 \\
    sym. & & & 0
\end{bmatrix}*10^8$ N/m \\\\
$K_{e2}=\frac{0.01(2E10)}{\sqrt{2}}\begin{bmatrix}
    0.5 & -0.5 & -0.5 & 0.5 \\
    & 0.5 & 0.5 & -0.5 \\
    & & 0.5 & -0.5 \\
    sym. & & & 0.5
\end{bmatrix}=\begin{bmatrix}
    1/\sqrt{2} & -1/\sqrt{2} & -1/\sqrt{2} & 1/\sqrt{2} \\
    & 1/\sqrt{2} & 1/\sqrt{2} & -1/\sqrt{2} \\
    & & 1/\sqrt{2} & -1/\sqrt{2} \\
    sym. & & & 1/\sqrt{2}
\end{bmatrix}*10^8$ N/m

\subsubsection*{Build Global Matrices}
combine $K_{e1}$ and $K_{e2}$: \\\\
$K=10^8*\begin{bmatrix}
    2 & 0 & -2 & 0 & 0 & 0 \\
    0 & 0 & 0 & 0 & 0 & 0 \\
    -2 & 0 & 2 & 0 & 0 & 0 \\
    0 & 0 & 0 & 0 & 0 & 0 \\
    0 & 0 & 0 & 0 & 0 & 0 \\
    0 & 0 & 0 & 0 & 0 & 0
\end{bmatrix}\rightarrow K=10^8*\begin{bmatrix}
    2 & 0 & -2 & 0 & 0 & 0 \\
    0 & 0 & 0 & 0 & 0 & 0 \\
    -2 & 0 & 2+1/\sqrt{2} & -1/\sqrt{2} & -1/\sqrt{2} & 1/\sqrt{2} \\
    0 & 0 & -1/\sqrt{2} & 1/\sqrt{2} & 1/\sqrt{2} & -1/\sqrt{2} \\
    0 & 0 & -1/\sqrt{2} & 1/\sqrt{2} & 1/\sqrt{2} & -1/\sqrt{2} \\
    0 & 0 & 1/\sqrt{2} & -1/\sqrt{2} & -1/\sqrt{2} & 1/\sqrt{2}
\end{bmatrix}$ \\\\
set up F and D matrices from boundary conditions: \\
$F=\begin{bmatrix}
    R_{D1} \\ R_{D2} \\ 0 \\ -100 \\ R_{D5} \\ R_{D6}
\end{bmatrix}$, \quad
$D=\begin{bmatrix}
    0 \\ 0 \\ D3 \\ D4 \\ 0 \\ 0
\end{bmatrix}$

\subsubsection*{Solve for displacements}
$KD=F$ \\
$10^8*\begin{bmatrix}
    2 & 0 & -2 & 0 & 0 & 0 \\
    0 & 0 & 0 & 0 & 0 & 0 \\
    -2 & 0 & 2+1/\sqrt{2} & -1/\sqrt{2} & -1/\sqrt{2} & 1/\sqrt{2} \\
    0 & 0 & -1/\sqrt{2} & 1/\sqrt{2} & 1/\sqrt{2} & -1/\sqrt{2} \\
    0 & 0 & -1/\sqrt{2} & 1/\sqrt{2} & 1/\sqrt{2} & -1/\sqrt{2} \\
    0 & 0 & 1/\sqrt{2} & -1/\sqrt{2} & -1/\sqrt{2} & 1/\sqrt{2}
\end{bmatrix}\begin{bmatrix}
    0 \\ 0 \\ D3 \\ D4 \\ 0 \\ 0
\end{bmatrix}=\begin{bmatrix}
    R_{D1} \\ R_{D2} \\ 0 \\ -100 \\ R_{D5} \\ R_{D6}
\end{bmatrix}$ \\\\
use Matlab to solve for displacements at node 2 (D3 and D4): \\
$\begin{bmatrix}
    -2 & 0 & 2+1/\sqrt{2} & -1/\sqrt{2} & -1/\sqrt{2} & 1/\sqrt{2} \\
    0 & 0 & -1/\sqrt{2} & 1/\sqrt{2} & 1/\sqrt{2} & -1/\sqrt{2}
\end{bmatrix}\begin{bmatrix}
    D3 \\ D4
\end{bmatrix}=\begin{bmatrix}
    0 \\ -100
\end{bmatrix}$ \\\\
$\fbox{
$\begin{bmatrix}
    0 \\ 0 \\ -5*10^{-7} \\ -1.9142*10^{-6} \\ 0 \\ 0
\end{bmatrix}$ m}$

\subsection*{Part B}
$F_e=\frac{AE}{l_e}\begin{bmatrix}
    -l_{ij} & -m_{ij} & l_{ij} & m_{ij}
\end{bmatrix}\begin{bmatrix}
    D1 \\ D2 \\ D3 \\ D4
\end{bmatrix}$ \\\\\\
$F_1=\frac{0.01(2E10)}{1}\begin{bmatrix}
    -1 & 0 & 1 & 0
\end{bmatrix}\begin{bmatrix}
    0 \\ 0 \\ -5*10^{-7} \\ -1.9142*10^{-6}
\end{bmatrix}$ \\
solve with matlab for F: \\
$\fbox{$F_1=-100$ N}$ \\\\\\
$F_2=\frac{0.01(2E10)}{\sqrt{2}}\begin{bmatrix}
    -\frac{1}{\sqrt{2}} & \frac{1}{\sqrt{2}} & \frac{1}{\sqrt{2}} & -\frac{1}{\sqrt{2}}
\end{bmatrix}\begin{bmatrix}
    -5*10^{-7} \\ -1.9142*10^{-6} \\ 0 \\ 0
\end{bmatrix}$ \\
solve with matlab for F: \\
$\fbox{$F_2=-141.42$ N}$

\subsection*{Part C}
plug in known values for displacements into KD=F: \\
$10^8*\begin{bmatrix}
    2 & 0 & -2 & 0 & 0 & 0 \\
    0 & 0 & 0 & 0 & 0 & 0 \\
    -2 & 0 & 2+1/\sqrt{2} & -1/\sqrt{2} & -1/\sqrt{2} & 1/\sqrt{2} \\
    0 & 0 & -1/\sqrt{2} & 1/\sqrt{2} & 1/\sqrt{2} & -1/\sqrt{2} \\
    0 & 0 & -1/\sqrt{2} & 1/\sqrt{2} & 1/\sqrt{2} & -1/\sqrt{2} \\
    0 & 0 & 1/\sqrt{2} & -1/\sqrt{2} & -1/\sqrt{2} & 1/\sqrt{2}
\end{bmatrix}\begin{bmatrix}
    0 \\ 0 \\ -5*10^{-7} \\ -1.9142*10^{-6} \\ 0 \\ 0
\end{bmatrix}=\begin{bmatrix}
    R_{D1} \\ R_{D2} \\ 0 \\ -100 \\ R_{D5} \\ R_{D6}
\end{bmatrix}$ \\\\\\
solve for each missing force in F with matlab:
$\fbox{$\begin{bmatrix}
    R_{D1} \\ R_{D2} \\ R_{D5} \\ R_{D6}
\end{bmatrix}=\begin{bmatrix}
    100 \\ 0 \\ -100 \\ 100
\end{bmatrix}$ N}$

\section*{Problem 2}
Figure 2 shows a three-node truss element of length $L$ and a constant cross-section area
$A$. It is made of a material of Young’s modulus $E$ and density $\rho$. The truss is
subjected to a uniformly distributed force $b$.
(a) Derive the stiffness matrix for the element. \\
(b) Write down the expression for the element mass matrix, and obtain $m_{11}$ in terms of
$L$, $E$, $\rho$, and $A$. \\
(c) Derive the external force vector.
\begin{center}
    \includegraphics[]{fig2} \\
    Figure 2
\end{center}
\subsection*{Part A}
\subsection*{Part B}
\subsection*{Part C}

\end{document}