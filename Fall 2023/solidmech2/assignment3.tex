\documentclass{article}
\usepackage{graphicx}
\graphicspath{{./images/hw3/}}
\usepackage[legalpaper, portrait, margin=1in]{geometry}

\title{AEEM4058 - Homework 3}
\author{Slade Brooks}
\date{09.26.2023}

\begin{document}
\maketitle

\section*{Problem 1}
Consider a 2 DOF system with two rigid blocks that can move only in the horizontal direction, as shown in
the following figure.
\begin{center}
    \includegraphics{fig1}
\end{center}

\subsection*{Part 1}
Give the expressions for the strain (potential) energy ($\Pi$), work done by external forces ($W_f$), and the
total potential energy ($\Pi_t=\Pi-W_f$). \\\\
$\fbox{$\Pi=\frac{1}{2}k_1D_1^2 + \frac{1}{2}k_2(D_2-D_1)^2$}$ \\
$\fbox{$W_f=D_1f_1 + D_2f_2$}$ \\
$\Pi_t=\Pi-W_f=\frac{1}{2}k_1D_1^2 + \frac{1}{2}k_2(D_2-D_1)^2-(D_1f_1 + D_2f_2)$ \\
$\fbox{$\Pi_t=\frac{1}{2}k_1D_1^2 + \frac{1}{2}k_2(D_2-D_1)^2-D_1f_1 - D_2f_2$}$

\subsection*{Part 2}
Derive the equilibrium equations for the system using the minimum potential energy principle. \\\\
$\Pi_t=\frac{1}{2}k_1D_1^2 + \frac{1}{2}k_2(D_2-D_1)^2-D_1f_1 - D_2f_2$ \\
$\delta\Pi_t=0=\delta[\frac{1}{2}k_1D_1^2 + \frac{1}{2}k_2(D_2-D_1)^2-D_1f_1 - D_2f_2]=
\delta[\frac{1}{2}k_1D_1^2 + \frac{1}{2}k_2(D_2^2-2D_1D_2+D_1^2)-D_1f_1 - D_2f_2]=
\delta[\frac{1}{2}k_1D_1^2 + \frac{1}{2}k_2D_2^2 -k_2D_1D_2 + \frac{1}{2}k_2D_1^2 -D_1f_1 - D_2f_2]$ \\\\
vary $D_1$: $\delta D_1[k_1D_1-k_2D_2+k_2D_1-f_1]=0 \rightarrow D_1=\frac{f_1+k_2D_2}{k_1+k_2}$ \\\\
vary $D_2$: $\delta D_2[k_2D_2-k_2D_1-f_2]=0 \rightarrow D_2=\frac{f_2+k_2D_1}{k_2}$ \\\\
$k_1D_1-k_2D_2+k_2D_1-f_1=0=k_1D_1-k_2(\frac{f_2+k_2D_1}{k_2})+k_2D_1-f_1=k_1D_1-(f_2+k_2D_1)+k_2D_1-f_1=0$ \\
$k_1D_1=f_1+f_2$ \\
$\fbox{$D_1=\frac{f_1+f_2}{k_1}$}$ \\\\
$D_2=\frac{f_2+k_2D_1}{k_2}=\frac{f_2+k_2(\frac{f_1+f_2}{k_1})}{k_2}=\frac{f_2}{k_2}+\frac{f_1+f_2}{k_1}$ \\
$\fbox{$D_2=\frac{f_2}{k_2}+\frac{f_1+f_2}{k_1}$}$

\section*{Problem 2}
Consider again the simplest problem of a continuum 1D bar of uniform cross-section studied in Q1-HW1.
The bar is fixed at the left-end and is of length $l$ and section area $A$. It is subjected to a uniform body
force $f_x$ and a concentrated force $F_s$ at the right end. The young’s modulus of the material is $E$. Using
the method of minimum potential energy, obtain the distribution of the displacement for the following cases.
\begin{center}
    \includegraphics{fig}
\end{center}

\subsection*{Part 1}
$f_x=0$ and $F_s=$ constant \\\\
$\Pi=\frac{1}{2}\int_V \varepsilon^T\sigma dV$, $\sigma=E\varepsilon$,
$\Pi=\frac{E}{2}A\int_{0}^{l}\varepsilon^2 dx$ \\\\
assume $u(x)=C_0+C_1x \rightarrow \varepsilon=\frac{du}{dx}=C_1$ \\\\
$\Pi=\frac{EA}{2}\int_{0}^{l}C_1^2 dx=\frac{EA}{2}lC_1^2$ \\
$W_f=u(l)F=C_1lF_s$ \\
$\Pi_t=\Pi-W_f=\frac{EA}{2}lC_1^2-lF_sC_1$ \\
$\delta\Pi_t=0=\delta C_1[EAlC_1-lF_s]=\delta C_1[EAC_1-F_s]=0$ \\
$C_1=\frac{F_s}{EA}$ \\
enforce boundary condition: $u(0)=0=C_0$ \\
$\fbox{$u(x)=\frac{F}{EA}x$}$

\subsection*{Part 2}
$f_x=$ constant and $F_s=$ constant \\\\
$\Pi=\frac{E}{2}A\int_{0}^{l}\varepsilon^2 dx$ \\\\
assume $u(x)=C_0+C_1x+C_2x^2 \rightarrow \varepsilon=\frac{du}{dx}=C_1+2C_2x$ \\\\
$\Pi=\frac{EA}{2}\int_{0}^{l}(C_1+2C_2x)^2 dx=\frac{EA}{2}\int_{0}^{l}(4C_2^2x^2+4C_1C_2x+C_1^2)dx
=\frac{EA}{2}(\frac{4C_2}{3}l^3+2C_1C_2l^2+C_1^2l)$\\
enforce boundary condition: $u(0)=0=C_0$ \\
$W_f=u(l)F+\int_{V}u^Tf_bdV=(C_1l+C_2l^2)F_s+A\int_{0}^{l}(C_1x+C_2x^2)f_xdx=(C_1l+C_2l^2)F_s+
A(\frac{C_1}{2}l^2+\frac{C_2}{3}l^3)f_x$ \\\\
$\Pi_t=\Pi-W_f=\frac{EA}{2}(\frac{4C_2^2}{3}l^3+2C_1C_2l^2+C_1^2l)-(C_1l+C_2l^2)F_s-
A(\frac{C_1}{2}l^2+\frac{C_2}{3}l^3)f_x$ \\
$\delta\Pi_t=0=\delta C_1(\frac{EA}{2}(2C_2l^2+2C_1l)-(l)F_s-A(\frac{l^2}{2})f_x)$ \\
$\delta\Pi_t=0=\delta C_2(\frac{EA}{2}(\frac{8C_2}{3}l^3 + 2C_1l^2)-(l^2)F_s-A(\frac{l^3}{3})f_x)$ \\\\
find $C_1$: $\frac{EA}{2}(2C_2l^2+2C_1l)-(l)F_s-A(\frac{l^2}{2})f_x=0$ \\
$(2C_2l^2+2C_1l)=\frac{A\frac{l^2}{2}f_x+lF_s}{\frac{EA}{2}}$ \\
$C_1=\frac{\frac{A}{2}lf_x+F_s}{EA}-C_2l$ \\
$C_1=\frac{lf_x}{2E}+\frac{F_s}{EA}-C_2l$ \\\\
find $C_2$: $\frac{EA}{2}(\frac{8C_2}{3}l^3 + 2C_1l^2)-(l^2)F_s-A(\frac{l^3}{3})f_x$ \\
$(\frac{8}{3}C_2l^3+2C_1l^2)=\frac{A\frac{l^3}{3}f_x+l^2F_s}{\frac{EA}{2}}=\frac{2l^3f_x}{3E}+
\frac{2l^2F_s}{EA}$ \\
$\frac{8}{3}C_2l^3=\frac{2l^3f_x}{3E}+\frac{2l^2F_s}{EA}-2C_1l^2$ \\
$C_2=\frac{f_x}{4E}+\frac{3F_s}{4EAl}-\frac{3}{4l}C_1$ \\\\
$C_2=\frac{f_x}{4E}+\frac{3F_s}{4EAl}-\frac{3}{4l}(\frac{lf_x}{2E}+\frac{F_s}{EA}-C_2l)$ \\
$C_2=\frac{f_x}{4E}+\frac{3F_s}{4EAl}-\frac{3f_x}{8E}-\frac{3F_s}{4EAl}+\frac{3}{4}C_2$ \\
$\frac{1}{4}C_2=\frac{-f_x}{8E}$ \\
$C_2=\frac{-f_x}{2E}$ \\\\
$C_1=\frac{lf_x}{2E}+\frac{F_s}{EA}-l(\frac{-f_x}{2E})$ \\
$C_1=\frac{lf_x}{E}+\frac{F_s}{EA}$ \\
$C_1=\frac{Alf_x+F_s}{EA}$ \\\\
$\fbox{$u(x)=\frac{Alf_x+F_s}{EA}x-\frac{f_x}{2E}x^2$}$

\end{document}