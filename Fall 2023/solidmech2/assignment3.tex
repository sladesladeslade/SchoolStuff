\documentclass{article}
\usepackage{graphicx}
\graphicspath{{./images/hw3/}}
\usepackage[legalpaper, portrait, margin=1in]{geometry}

\title{AEEM4058 - Homework 3}
\author{Slade Brooks}
\date{09.26.2023}

\begin{document}
\maketitle

\section*{Problem 1}
Consider a 2 DOF system with two rigid blocks that can move only in the horizontal direction, as shown in
the following figure.
\begin{center}
    \includegraphics{fig1}
\end{center}

\subsection*{Part 1}
Give the expressions for the strain (potential) energy ($\Pi$), work done by external forces ($W_f$), and the
total potential energy ($\Pi_t=\Pi-W_f$). \\\\
$\fbox{$\Pi=\frac{1}{2}k_1D_1^2 + \frac{1}{2}k_2(D_2-D_1)^2$}$ \\
$\fbox{$W_f=D_1f_1 + D_2f_2$}$ \\
$\Pi_t=\Pi-W_f=\frac{1}{2}k_1D_1^2 + \frac{1}{2}k_2(D_2-D_1)^2-(D_1f_1 + D_2f_2)$ \\
$\fbox{$\Pi_t=\frac{1}{2}k_1D_1^2 + \frac{1}{2}k_2(D_2-D_1)^2-D_1f_1 - D_2f_2$}$

\subsection*{Part 2}
Derive the equilibrium equations for the system using the minimum potential energy principle. \\\\
$\Pi_t=\frac{1}{2}k_1D_1^2 + \frac{1}{2}k_2(D_2-D_1)^2-D_1f_1 - D_2f_2$ \\
$\delta\Pi_t=0=\delta[\frac{1}{2}k_1D_1^2 + \frac{1}{2}k_2(D_2-D_1)^2-D_1f_1 - D_2f_2]=
\delta[\frac{1}{2}k_1D_1^2 + \frac{1}{2}k_2(D_2^2-2D_1D_2+D_1^2)-D_1f_1 - D_2f_2]=
\delta[\frac{1}{2}k_1D_1^2 + \frac{1}{2}k_2D_2^2 -k_2D_1D_2 + \frac{1}{2}k_2D_1^2 -D_1f_1 - D_2f_2]$ \\\\
vary $D_1$: $\delta D_1[k_1D_1-k_2D_2+k_2D_1-f_1]=0 \rightarrow \fbox{$D_1=\frac{f_1+k_2D_2}{k_1+k_2}$}$ \\\\
vary $D_2$: $\delta D_2[k_2D_2-k_2D_1-f_2]=0 \rightarrow \fbox{$D_2=\frac{f_2+k_2D_1}{k_2}$}$

\section*{Problem 2}
Consider again the simplest problem of a continuum 1D bar of uniform cross-section studied in Q1-HW1.
The bar is fixed at the left-end and is of length $l$ and section area $A$. It is subjected to a uniform body
force $f_x$ and a concentrated force $F_s$ at the right end. The young’s modulus of the material is $E$. Using
the method of minimum potential energy, obtain the distribution of the displacement for the following cases.

\subsection*{Part 1}
$f_x=0$ and $F_s=$ constant

\subsection*{Part 2}
$f_x=$ constant and $F_s=$ constant

\end{document}