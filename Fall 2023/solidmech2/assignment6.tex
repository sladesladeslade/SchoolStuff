\documentclass{article}
\usepackage{graphicx}
\usepackage{amsmath}
\usepackage{multirow}
\setcounter{MaxMatrixCols}{11}
\graphicspath{{./images/hw6/}}
\usepackage[legalpaper, portrait, margin=1in]{geometry}

\title{AEEM4058 - Homework 6}
\author{Slade Brooks}
\date{11.07.2023}

\begin{document}
\maketitle

\section*{Problem 1}
Show that the stiffness matrix of an isotropic linear triangular element whose thickness varies
linearly in the element is: \\
$k_e=\bar{h}A_eB^TcB$ \\
where B is the strain matrix, c is matrix of material constants, $A_e$ is the area of the
triangle and $\bar{h}$ the average thickness $(h_1+h_2+h_3)/3$, where $h_1$, $h_2$, and $h_3$ are
the nodal thickness at the node. \\\\\\
$N=\begin{bmatrix}
    N_1 & N_2 & N_3
\end{bmatrix}=\begin{bmatrix}
    L_1 & L_2 & L_3
\end{bmatrix}$, \quad
$h=\begin{bmatrix}
    h_1 \\ h_2 \\ h_3
\end{bmatrix}$ \\\\
$h(x, y)=Nh=L_1h_1 + L_2h_2 + L_3h_3$ \\\\
$k_e=B^TcB\int_{A_e}h(x,y)dA=B^TcB\int_{A_e}(L_1h_1+L_2h_2+L_3h_3) dA$ \\\\
from eisenberg and malvern: $\int_{A_e}L_1^mL_2^nL_3^p dA=\frac{m!n!p!}{(m+n+p+2)!}$ \\
$h_1\int_{A_e}L_1=h_1\frac{1!0!0!}{(1+0+0+2)!}2A_e=\frac{2h_1}{6}A_e=\frac{h_1A_e}{3}$ \\
$h_2\int_{A_e}L_2=h_2\frac{0!1!0!}{(0+1+0+2)!}2A_e=\frac{2h_2}{6}A_e=\frac{h_2A_e}{3}$ \\
$h_3\int_{A_e}L_3=h_3\frac{0!0!1!}{(0+0+1+2)!}2A_e=\frac{2h_3}{6}A_e=\frac{h_3A_e}{3}$ \\\\
$k_e=B^TcB\frac{h_1+h_2+h_3}{3}A_e \rightarrow \fbox{$k_e=\bar{h}A_eB^TcB$}$

\section*{Problem 2}
Show that the mass matrix of a linear triangular element whose thickness varies linearly within
the plane of the element is: \\\\
$m_e=\frac{\rho\bar{h}A_e}{60}\begin{bmatrix}
    6+4\alpha_1 & 0 & 6-\alpha_3 & 0 & 6-\alpha_2 & 0 \\
    & 6+4\alpha_1 & 0 & 6-\alpha_3 & 0 & 6-\alpha_2 \\
    & & 6+4\alpha_2 & 0 & 6-\alpha_1 & 0 \\
    & & & 6+4\alpha_2 & 0 & 6-\alpha_1 \\
    & & & & 6+4\alpha_3 & 0 \\
    sym. & & & & & 6+4\alpha_3
\end{bmatrix}$ \\\\
where $\rho$ is the density, $A_e$ is the area, $\bar{h}$ is the mean thickness and $\alpha_i
=\frac{h_i}{\bar{h}}$ with i=1, 2, 3 for the three nodes. \\\\\\
$m_e=\rho\int_{A_e}h(x,y)N^TNdA$ \\\\
$m_e=\rho\int_{A_e}\begin{bmatrix}
    hN_1^2 & 0 & hN_1N_2 & 0 & hN_1N_3 & 0 \\
    & hN_1^2 & 0 & hN_1N_2 & 0 & hN_1N_3 \\
    & & hN_2^2 & 0 & hN_2N_3 & 0 \\
    & & & hN_2^2 & 0 & hN_2N_3 \\
    & & & & hN_3^2 & 0 \\
    sym. & & & & & hN_3^2
\end{bmatrix}dA$ \\\\
$\int_{A_e}hN_1^2=\int_{A_e}(h_1N_1^3+h_2N_2N_1^2+h_3N_3N_1^2)dA=\frac{12}{120}h_1A_e+
\frac{4}{120}h_2A_e+\frac{4}{120}h_3A_e=\frac{60}{60}A_e(\frac{1}{10}h_1+\frac{1}{30}(h_2+h_3))$ \\
$\frac{A_e}{60}(4h_1+2(h_1+h_2+h_3))=\frac{A_e}{60}(4h_1+6\bar{h})=
\frac{A_e\bar{h}}{60}(4\alpha_1+6)$, \quad so \quad $\int_{A_e}hN_i^2dA=
\frac{A_e\bar{h}}{60}(6+4\alpha_i)$ \\\\
$\int_{A_e}hN_1N_2dA=\int_{A_e}(h_1N_1^2N_2+h_2N_1N_2^2+h_3N_1N_2N_3)dA=\frac{4}{120}h_1A_e+
\frac{4}{120}h_2A_e+\frac{2}{120}h_3A_e$ \\
$\frac{60}{60}A_e(\frac{1}{30}h_1+\frac{1}{30}h_2+ \frac{1}{60}h_3)=\frac{A_e}{60}(2h_1+2h_2+h_3)
=\frac{A_e}{60}(2(h_1+h_2+h_3)-h_3)=\frac{A_e}{60}(6\bar{h}-h_3)=
\frac{A_e\bar{h}}{60}(6-\alpha_3)$, \quad so \quad $\int_{A_e}hN_iN_jdA=\frac{A_e\bar{h}}{60}
(6-\alpha_{\textrm{not i or j}})$ \\\\
then apply those definitions to the original $m_e$ matrix to get the desired matrix: \\
$\fbox{$m_e=\frac{\rho\bar{h}A_e}{60}\begin{bmatrix}
    6+4\alpha_1 & 0 & 6-\alpha_3 & 0 & 6-\alpha_2 & 0 \\
    & 6+4\alpha_1 & 0 & 6-\alpha_3 & 0 & 6-\alpha_2 \\
    & & 6+4\alpha_2 & 0 & 6-\alpha_1 & 0 \\
    & & & 6+4\alpha_2 & 0 & 6-\alpha_1 \\
    & & & & 6+4\alpha_3 & 0 \\
    sym. & & & & & 6+4\alpha_3
\end{bmatrix}$}$

\section*{Problem 3}
Figure 7-39 shows a plane strain problem to be solved using only one rectangular element.
Determine the nodal displacements and element stresses. \\
\begin{center}
    \includegraphics[scale=0.4]{fig1} \\ Figure 7-39
\end{center}
$\xi=\frac{x-\frac{x_2+x_1}{2}}{a}$, \quad $\eta=\frac{y-\frac{y_2+y_1}{2}}{b}$ \\\\
$k_e=\begin{bmatrix}
    K_{11} & K_{12} & K_{13} & K_{14} \\
    & K_{22} & K_{23} & K_{24} \\
    & & K_{33} & K_{34} \\
    sym. & & & K_{44}
\end{bmatrix}$, \quad $k_e=\frac{abh}{16}\int_{-1}^{1}\int_{-1}^{1}B^TcBd\xi d\eta$ \\\\
$B=\begin{bmatrix}
    \textrm{Node 1} & & & \textrm{Node 2} & & & \textrm{Node 3} & & & \textrm{Node 4}\\
    \hline
    -\frac{1-\eta}{a} & 0 & | & \frac{1-\eta}{a} & 0 & | & \frac{1+\eta}{a} & 0 & |
    & -\frac{1+\eta}{a} & 0 \\
    0 & -\frac{1-\xi}{b} & | & 0 & -\frac{1+\xi}{b} & | & 0 & \frac{1+\xi}{b} & | & 0 &
    \frac{1-\xi}{b} \\
    -\frac{1-\xi}{b} & -\frac{1-\eta}{a} & | & -\frac{1+\xi}{b} & \frac{1-\eta}{a} & | & 
    \frac{1+\xi}{b} & \frac{1+\eta}{a} & | & \frac{1-\xi}{b} & -\frac{1+\eta}{a}
\end{bmatrix}$ \\\\
$C=\begin{bmatrix}
    C_{11} & C_{12} & 0 \\
    & C_{22} & 0 \\
    sym. & & C_{33}
\end{bmatrix}=\frac{E(1-v)}{(1+v)(1-v)}\begin{bmatrix}
    1 & \frac{v}{1-v} & 0 \\
    & 1 & 0 \\
    sym. & & \frac{1-2v}{2(1-v)}
\end{bmatrix}$ \\\\
$k_ed_e=f_e=\begin{bmatrix}
    0 \\ 0 \\ 0 \\ -P \\ 0 \\ 0 \\ 0 \\ 0
\end{bmatrix}$ \\\\\\
matlab output: \\
\begin{tabular}{|c|}
\hline
X disp: $(72*P*a^2*b^2*(v - 1)*(v + 1))/(E*h*(64*a^4*v^2 - 96*a^4*v + 32*a^4 + 128*a^2*b^2*v^2$ \\$ -
192*a^2*b^2*v + 71*a^2*b^2 + 64*b^4*v^2 - 96*b^4*v + 32*b^4))$ \\ \hline
Y disp:
$-(96*P*a*b*(v - 1)*(v + 1)*(2*a^2*v + 2*b^2*v - a^2 - 2*b^2))/ $\\$ (E*h*(64*a^4*v^2 -
96*a^4*v + 32*a^4 + 128*a^2*b^2*v^2 - 192*a^2*b^2*v + 71*a^2*b^2 + $\\$ 64*b^4*v^2
- 96*b^4*v + 32*b^4))$
\\ \hline
\end{tabular}
\\\\ all other nodes have 0 disp b/c clamped \\\\
$\sigma_{xx}=E\varepsilon_{xx}=E\frac{dx}{l} \rightarrow \sigma_{e1}=\sigma_{xx}=\frac{Edx}{2a}$ \\
$\sigma_{yy}=E\varepsilon_{yy}=E\frac{dy}{l} \rightarrow \sigma_{e2}=\sigma_{yy}=\frac{Edy}{2b}$ \\\\
$\fbox{$\sigma_{e1}=\frac{E}{2a}\textrm{X disp}$}$ \\
$\fbox{$\sigma_{e2}=\frac{E}{2b}\textrm{Y disp}$}$ \\\\
no forces/disp in other elements so no stress in those

\section*{Matlab Code}
\begin{verbatim}
clear; clc; close all;

syms n x a b v E C11 C12 C22 C33 P h

% B matrix
B = [-(1-n)/a 0 (1-n)/a 0 (1+n)/a 0 -(1+n)/a 0;
    0 -(1-x)/b 0 -(1+x)/b 0 (1+x)/b 0 (1-x)/b;
    -(1-x)/b -(1-n)/a -(1+x)/b (1-n)/a (1+x)/b (1+n)/a (1-x)/b -(1+n)/a];

% C matrix
C = (E*(1-v))/((1+v)*(1-v))*[1 v/(1-v) 0;
                            v/(1-v) 1 0;
                            0 0 (1-2*v)/(2*(1-v))];

% ke matrix
ke = (a*b*h/16)*int(int(transpose(B)*C*B, x, [-1, 1]), n, [-1, 1]);

% fe matrix
fe = [0; 0; 0; -P; 0; 0; 0; 0];

% calc disp at only node 2
disp2 = inv(ke(3:4, 3:4))*fe(3:4, 1);
fprintf("x disp @ node 2:")
simplify(disp2(1))
fprintf("y disp @ node 2:")
simplify(disp2(2))
\end{verbatim}

\end{document}