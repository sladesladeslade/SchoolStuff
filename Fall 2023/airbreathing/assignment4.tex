\documentclass{article}
\usepackage{graphicx}
\graphicspath{{./images/A4}}
\usepackage[legalpaper, portrait, margin=1in]{geometry}

\title{AEEM4063 - Assignment 4}
\author{Slade Brooks}
\date{11.07.2023}

\begin{document}
\maketitle

\section*{Problem 1}
\subsection*{Air}
Air = $O_2 + 3.76N_2$ \\
$MW_{air}=2MW_O+7.52MW_N=2(15.999)+7.52(14.007)=137.33$ g/mol \\
$\fbox{$MW_{air}=137.33$ g/mol}$ 

\subsection*{Dodecane}
Dodecane = $C_{12}H_{26}$ \\
$MW_{C_{12}H_{26}}=12MW_C+26MW_H=12(12.011)+26(1.008)=170.34$ g/mol \\
$\fbox{$MW_{C_{12}H_{26}}=170.34$ g/mol}$

\subsection*{Fuel-Air Ratio}
$Q_{f}=44147$ kJ/kg (from book p.286) \\
$\dot{m}fQ_f=c_p\dot{m}(\Delta T_0) \rightarrow f=\frac{c_p \Delta T_0}{Q_f}=\frac{1.08*1150}{44147}$ \\
$\fbox{$f=0.0281$}$

\subsection*{Moles of Air Required}
$f=\frac{MW_{C_{12}H_{26}}N_{C_{12}H_{26}}}{MW_{air}N_{air}} \rightarrow
N_{air}=\frac{MW_{C_{12}H_{26}}N_{C_{12}H_{26}}}{fMW_{air}}=\frac{170.34(1)}{0.0281(137.33)}=
44.14$ mol \\
$\fbox{$N_{air}=44.14$ mol/mol of fuel}$

\section*{Problem 2}
\subsection*{Part A}
\subsubsection*{$\phi=1, C_3H_8$}
for $C_nH_{2n+2}$ fuels with $\phi=1$, \\
$C_nH_{2n+2}+(\frac{3n+1}{2})(O_2+3.76N_2) \rightarrow nCO_2 + (n+1)H_2O +3.76(\frac{3n+1}{2})N_2$ \\
$\fbox{$C_3H_8+5(O_2+3.76N_2)\rightarrow 3CO_2+4H_2O+18.8N_2$}$ \\
$\fbox{1 mol $C_3H_8$, 5 mol air, 3 mol $CO_2$, 4 mol water, 18.8 mol $N_2$}$\\\\
$f_{stoic}=\frac{7n+1}{34.32(3n+1)}$ \\
$\fbox{$f=0.0641$}$ \\\\
$\frac{A}{F}=\frac{1}{f}=15.6$ \\
$\fbox{$\frac{A}{F}=15.6$}$

\subsubsection*{$\phi=0.5, C_3H_8$}
$\phi=\frac{1}{X}$, $X=2$ \\
for $C_nH_{2n+2}$ fuels with $\phi<1$ and $X$, \\
$C_nH_{2n+2}+X(\frac{3n+1}{2})(O_2+3.76N_2) \rightarrow nCO_2 + (n+1)H_2O +3.76X(\frac{3n+1}{2})N_2 +
(X-1)(\frac{3n+1}{2})O_2$ \\
$\fbox{$C_3H_8+10(O_2+3.76N_2)\rightarrow 3CO_2+4H_2O+37.6N_2+5O_2$}$ \\
$\fbox{1 mol $C_3H_8$, 10 mol air, 3 mol $CO_2$, 4 mol water, 37.6 mol $N_2$, 5 mol $O_2$}$\\\\
$f=\phi f_{stoic}=0.5(0.0641)=0.03205$ \\
$\fbox{$f=0.03205$}$ \\\\
$\frac{A}{F}=\frac{1}{f}=31.2$ \\
$\fbox{$\frac{A}{F}=31.2$}$

\subsection*{Part B}
\subsubsection*{$\phi=1, C_{10}H_{22}$}
for $C_nH_{2n+2}$ fuels with $\phi=1$, \\
$C_nH_{2n+2}+(\frac{3n+1}{2})(O_2+3.76N_2) \rightarrow nCO_2 + (n+1)H_2O +3.76(\frac{3n+1}{2})N_2$ \\
$\fbox{$C_{10}H_{22}+15.5(O_2+3.76N_2) \rightarrow 10CO_2+11H_2O+58.28N_2$}$ \\
$\fbox{1 mol $C_{10}H_{22}$, 15.5 mol air, 10 mol $CO_2$, 11 mol water, 58.28 mol $N_2$}$\\\\
$f_{stoic}=\frac{7n+1}{34.32(3n+1)}$ \\
$\fbox{$f=0.06673$}$ \\\\
$\frac{A}{F}=\frac{1}{f}=14.985$ \\
$\fbox{$\frac{A}{F}=14.985$}$

\subsubsection*{$\phi=0.5, C_{10}H_{22}$}
$\phi=\frac{1}{X}$, $X=2$ \\
for $C_nH_{2n+2}$ fuels with $\phi<1$ and $X$, \\
$C_nH_{2n+2}+X(\frac{3n+1}{2})(O_2+3.76N_2) \rightarrow nCO_2 + (n+1)H_2O +3.76X(\frac{3n+1}{2})N_2 +
(X-1)(\frac{3n+1}{2})O_2$ \\
$\fbox{$C_{10}H_{22}+31(O_2+3.76N_2) \rightarrow 10CO_2+11H_2O+116.56N_2+15.5O_2$}$ \\
$\fbox{1 mol $C_{10}H_{22}$, 31 mol air, 10 mol $CO_2$, 11 mol water, 116.56 mol $N_2$,
15.5 mol $O_2$}$\\\\
$f=\phi f_{stoic}=0.5(0.06673)=0.03337$ \\
$\fbox{$f=0.03337$}$ \\\\
$\frac{A}{F}=\frac{1}{f}=29.97$ \\
$\fbox{$\frac{A}{F}=29.97$}$

\section*{Problem 3}
\begin{tabular}{ccc}
    $A_i$ & 0.0389 & m$^2$ \\
    $A_m$ & 0.0975 & m$^2$ \\
    $K_1$ & 19 \\
    $\dot{m}$ & 9 & kg/s \\
    $T_{01}$ & 475 & K \\
    $T_{02}$ & 1023 & K \\
    $P_1$ & 4.47 & bar \\
    $\Delta P_0$ & 0.27 & bar
\end{tabular} \\\\\\
$\frac{\Delta P_0}{\dot{m}^2/2\rho_1A_m^2}=K_1+K_2(\frac{T_{02}}{T_{01}}-1)$ \\\\
$\dot{m}=\rho_1V_1A_i \rightarrow V_1=\frac{\dot{m}}{\rho_1A_i}$ \\
$T_{01}=T_1+\frac{V_1^2}{2c_p} \rightarrow T_{01}=\frac{P_1}{R\rho_1}+
\frac{\dot{m}^2}{2c_p\rho_1^2A_i^2} \rightarrow \rho_1^2T_{01}=\frac{P_1\rho_1}{R}+\frac{\dot{m}^2}
{2c_pA_i^2}$ \\
$T_{01}\rho_1^2-\frac{P_1}{R}\rho_1-\frac{\dot{m}^2}{2c_pA_i^2}=0$ \\
$475\rho_1^2-\frac{4.47*10^5}{287}\rho_1-\frac{9^2}{2(1005)(0.0389)^2}=0$, \quad
$\rho_1=3.296$ kg/m$^3$ \\\\
$K_2=(\frac{\Delta P_0}{\dot{m}^2/2\rho_1A_m^2}-K_1)/(\frac{T_{02}}{T_{01}}-1)=
(\frac{0.27*10^5}{9/2(3.296)(0.0975)^2}-19)/(\frac{1023}{475}-1)=1.637$ \\\\\\
\begin{tabular}{ccc}
    $A_m$ & 0.0975 & m$^2$ \\
    $K_1$ & 19 \\
    $K_2$ & 1.637 \\
    $\dot{m}$ & 7.4 & kg/s \\
    $T_{01}$ & 439 & K \\
    $T_{02}$ & 900 & K \\
    $P_1$ & 3.52 & bar
\end{tabular} \\\\\\
$\Delta P_0=\frac{\dot{m}^2}{2\rho_1A_m^2}(K_1 + K_2(\frac{T_{02}}{T_{01}}-1))$ \\\\
$T_{01}\rho_1^2-\frac{P_1}{R}\rho_1-\frac{\dot{m}^2}{2c_pA_i^2}=0$ \\
$439\rho_1^2-\frac{3.52*10^5}{287}\rho_1-\frac{7.4^2}{2(1005)(0.0389)^2}$, \quad
$\rho_1=2.808$ kg/m$^3$ \\\\
$\Delta P_0=\frac{7.4^2}{2(2.808)(0.0975)^2}(19+1.637(\frac{900}{439}-1))=21251.85$ Pa \\
$\fbox{$\Delta P_0=0.213$ bar}$

\subsection*{Part A}
\subsubsection*{Design}
$V_1=\frac{\dot{m}}{\rho_1A_i}=\frac{9}{3.296(0.0389)}=70.195$ m/s \\
$\fbox{$V_1=70.2$ m/s}$

\subsubsection*{Partial}
$V_{1}=\frac{\dot{m}}{\rho_1A_i}=\frac{7.4}{2.808(0.0389)}=67.75$ m/s \\
$\fbox{$V_{1}=67.75$ m/s}$

\subsection*{Part B}
\subsubsection*{Design}
$P_{01}=P_1(\frac{T_{01}}{T_1})^\frac{\gamma}{\gamma-1}=P_1(\frac{T_{01}R\rho_1}{P_1})
^\frac{\gamma}{\gamma-1}=4.47(\frac{475(287)(3.296)}{4.47*10^5})^\frac{1.4}{0.4}=4.552$ bar \\\\
$\frac{\Delta P_0}{P_{01}}=\frac{0.27}{4.552}=0.05931$ \\
$\fbox{$\frac{\Delta P_0}{P_{01}}=0.05931$}$

\subsubsection*{Partial}
$P_{01}=P_1(\frac{T_{01}R\rho_1}{P_1})^\frac{\gamma}{\gamma-1}
=3.52(\frac{439(287)(2.808)}{3.52*10^5})^\frac{1.4}{0.4}=3.583$ bar \\
$\frac{\Delta P_0}{P_{01}}=\frac{0.213}{3.583}=0.05945$ \\
$\fbox{$\frac{\Delta P_0}{P_{01}}=0.0595$}$

\subsubsection*{Discussion}
We can see that despite the different operating conditions, the ratio of pressure loss to delivery
pressure stays relatively the same. This would imply that the combustor will have similar performance
in terms of pressure loss across a range of operating conditions that would be required for engine
operation. We also see that the velocity is similar in both cases, which is also a good sign for stable
operation across a range of conditions.

\end{document}