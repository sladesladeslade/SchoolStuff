\documentclass{article}
\usepackage{graphicx}
\graphicspath{{./images/A2B}}
\usepackage[legalpaper, portrait, margin=1in]{geometry}

\title{AEEM4063 - Assignment 2B}
\author{Slade Brooks}
\date{09.29.2023}

\begin{document}
\maketitle

\section*{Problem 1}
Simple turbojet is operating with a compressor pressure ratio of 8.0, a turbine
inlet temperature of 1200 K, and a mass flow of 15 kg/s, when the aircraft is flying at 260
m/s at an altitude of 7,000 m. Assuming the following component efficiencies, and ISA
conditions, calculate the propelling nozzle area required, the net thrust developed, and
the SFC.
\begin{center}
\begin{tabular}{cc}
    r & 8 \\
    $T_{03}$ (K) & 1200 \\
    $\dot{m}$ (kg/s) & 15 \\
    $C_a$ (m/s) & 260 \\
    alt (m) & 7000 \\
    $\eta_{\infty c}$ & 0.87 \\
    $\eta_{\infty t}$ & 0.87 \\
    $\eta_i$ & 0.95 \\
    $\eta_j$ & 0.95 \\
    $\eta_m$ & 0.99 \\
    $\Delta P_b/P_{02}$ (\%) & 6 \\
    $\eta_b$ & 0.97
\end{tabular}
\end{center}
\subsubsection*{Ambient}
find conditions at altitude: \\
$\theta=0.8423$, $\delta=0.4057$ (from table A.2) \\
$P_a=0.4057*1.01325=0.411$ bar, $T_a=0.8423*288.15=242.71$ K, $a=340.3*\sqrt{0.8423}=312.317$ m/s \\
$M=\frac{V}{a}=\frac{260}{312.317}=0.8325$

\subsubsection*{Inlet}
find $P_{01}$: \\
$P_{01}/P_a=(1+\eta_i\frac{\gamma_c-1}{2}M^2)^\frac{\gamma_c}{\gamma_c-1}=
(1+0.95\frac{0.4}{2}(0.8325)^2)^\frac{1.4}{0.4}=1.542$ \\
$P_{01}=0.6338$ bar \\\\
intake is adiabatic so: $T_{01}=T_{0a}=T_a*(1+\frac{\gamma_c-1}{2}M^2)=
288.15(1+\frac{0.4}{2}(0.8325)^2)$ \\
$T_{01}=276.35$ K

\subsubsection*{Compressor}
$P_{02}=rP_{01}=8(0.6338)=5.0704$ bar \\\\
$\frac{n-1}{n}=\frac{\gamma_c-1}{\gamma_c\eta_{\infty c}}=\frac{0.4}{1.4(0.87)}=0.3284$ \\
$T_{02}/T_{01}=(r)^\frac{n-1}{n}=8^{0.3284}=1.98$ \\
$T_{02}=547.17$ K

\subsubsection*{Combustor}
account for pressure loss in combustor: $P_{03}=P_{02}(1-\frac{\Delta P_b}{P_{02}})=
5.0704(1-0.06)=4.766$ bar \\
$P_{03}=4.766$ bar

\subsubsection*{Turbine}
$W_c=\frac{c_{pc}}{\eta_m}(T_{02}-T_{01})=\frac{1.005}{0.99}(547.17-276.35)=275$ kJ/kg \\
$W_t=W_c=275$ kJ/kg $=c_{pg}(T_{03}-T_{04})=1.148(1200-T_{04})=275$ \\
$T_{04}=960.45$ K \\\\
$\frac{m-1}{m}=\frac{\eta_{\infty t}(\gamma_g-1)}{\gamma_g}=\frac{0.87(0.333)}{1.333}=0.2173$ \\
$\frac{T_{04}}{T_{03}}=(\frac{P_{04}}{P_{03}})^\frac{m-1}{m} \rightarrow \frac{960.45}{1200}=
(\frac{P_{04}}{P_{03}})^{0.2173} \rightarrow \frac{P_{04}}{P_{03}}=0.359$ \\
$P_{04}=1.71$ bar

\subsubsection*{Nozzle}
$NPR=\frac{P_{04}}{P_a}=\frac{1.71}{0.411}=4.16$ \\
$PR_{crit}=\frac{1}{[1-\frac{1}{\eta_j}(\frac{\gamma_g-1}{\gamma_g+1})]
^\frac{\gamma_g}{\gamma_g-1}}=\frac{1}{[1-\frac{1}{0.95}(\frac{0.333}{2.333})]
^\frac{1.333}{0.333}}=1.92$ \\
$NPR>PR_{crit} \rightarrow$ nozzle is choked \\
$P_5=\frac{P_{04}}{PR_{crit}}=\frac{1.71}{1.92}=0.89$ bar \\\\
$TR_{crit}=\frac{\gamma_g+1}{2}=\frac{2.333}{2}=1.167$ \\
$T_5=\frac{T_{04}}{TR_{crit}}=\frac{960.45}{1.167}=823$ K \\\\
$C_5=1\sqrt{\gamma_gRT_5}=\sqrt{1.333*287*823}=561.12$ m/s \\
$\rho_5=\frac{P_5}{RT_5}=\frac{0.89}{287*823}*100000=0.377$ kg/m$^3$ \\
$\dot{m}=\rho_5C_5A_5 \rightarrow A_5=\frac{\dot{m}}{\rho_5C_5}=\frac{15}{0.377*561.12}
=0.0709$ m$^2$ \\
$\fbox{$A_5=0.0709$ m$^2$}$ \\\\
$F=\dot{m}(C_5-C_a)+A_5(P_5-P_a)=15(561.12-260)+0.0709(0.89-0.411)*100000=7913$ N \\
$\fbox{$F=7913$ N}$ \\\\
$f_a=\frac{c_{pg}T_{03}-c_{pc}T_{02}}{\eta_b(Q_f-c_{pg}T_{03})}=
\frac{1.148(1200)-1.005(547.17)}{0.99(43100-1.148(1200))}=0.02$ \\
$F_s=\frac{F}{\dot{m}}=\frac{7913}{15}=527.53$ Ns/kg \\
$TSFC=\frac{f}{F_s}=\frac{0.02}{527.53}*3600=0.1365$ kg/N-hr \\
$\fbox{$TSFC=0.1365$ kg/N-hr}$

\section*{Problem 2}
Under take-off conditions when the ambient pressure and temperature are 1.01
bar and 288 K, the stagnation pressure and temperature in the jet pipe of a turbojet
engine are 3.8 bar and 1000 K, and the mass flow is 23 kg/s. Calculate:

a. The exit area required and the thrust produced, assuming an isentropic
convergent nozzle. (Use stations 7 \& 9 for inlet and exit of nozzle)

b. The nozzle exit area and thrust produced, assuming an isentropic convergent-
divergent nozzle that is fully expanded. (Use stations 7, 8, \& 9 for inlet, throat, and
exit of nozzle)

c. Comment on which nozzle you would choose if you were the design engineer
responsible for selecting the type of nozzle to utilize on an aircraft. \\\\
$P_a=1.01$ bar, $T_a=288$ K, $P_{07}=3.8$ bar, $T_{07}=1000$ K, $\dot{m}=23$ kg/s

\subsection*{Part A}
$NPR=\frac{P_{07}}{P_a}=\frac{3.8}{1.01}=3.76$ \\
$PR_{crit}=\frac{1}{[1-(\frac{\gamma_g-1}{\gamma_g+1})]
^\frac{\gamma_g}{\gamma_g-1}}=\frac{1}{[1-(\frac{0.333}{2.333})]
^\frac{1.333}{0.333}}=1.85$ \\
$NPR>PR_{crit} \rightarrow$ nozzle is choked \\
$P_9=\frac{P_{07}}{PR_{crit}}=\frac{3.8}{1.85}=2.054$ bar \\\\
$TR_{crit}=1.167$ \\
$T_9=\frac{T_{07}}{TR_{crit}}=\frac{1000}{1.167}=856.9$ K \\\\
$C_9=1\sqrt{\gamma_gRT_9}=\sqrt{1.333(287)(856.9)}=572.56$ m/s \\
$\rho_9=\frac{P_9}{RT_9}=\frac{2.054}{287(856.9)}*100000=0.8352$ kg/m$^3$ \\
$A_9=\frac{\dot{m}}{\rho_9C_9}=\frac{23}{0.8352(572.56)}=0.0481$ m$^2$ \\
$\fbox{$A_9=0.0481$ m$^2$}$ \\\\
$F=\dot{m}(C_9-C_a)+A_9(P_9-P_a)=23(572.56-0)+0.0481(2.054-1.01)*100000=18191$ N \\
$\fbox{$F=18191$ N}$

\subsection*{Part B}
fully expanded: $P_9=P_a$, $T_9=T_a$
$P_{09}=P_{07}$ because isentropic nodzle \\
$\frac{P_{09}}{Pa}=[1+\frac{\gamma_g-1}{2}M^2]^\frac{\gamma_g}{\gamma_g-1} \rightarrow
\frac{3.8}{1.01}=[1+\frac{0.333}{2}M^2]^\frac{1.333}{0.333}$ \\
solve for $M=1.535$ \\\\
$C_9=M\sqrt{\gamma_gRT_9}=1.535\sqrt{1.333(287)(288)}=509.52$ m/s \\
$\rho_9=\frac{P_9}{RT_9}=\frac{1.01}{287(288)}*100000=1.222$ kg/m$^3$ \\
$A_9=\frac{\dot{m}}{\rho_9C_9}=\frac{23}{1.222(509.52)}=0.03694$ m$^2$ \\
$\fbox{$A_9=0.03694$ m$^2$}$ \\\\
$F=\dot{m}(C_9-C_a)=23(509.52)=11719$ N \\
$\fbox{$F=11719$ N}$

\subsection*{Part C}
For this case I would choose the convergent nozzle because it provides more thrust
at the same conditions. If the aircraft were to have an afterburner, I would want the C-D
nozzle to be able to operate efficiently across all flight regimes. Otherwise, the reduced
complexity and increased performance of the converging nozzle is preferred.

\section*{Problem 3}
A high bypass ratio turbofan is designed for a cruise condition of M=0.85, at
an altitude of 11,000 m. Assume a simple two-spool configuration with separate
converging-only nozzles. The following cycle data apply:
\begin{center}
\begin{tabular}{cc}
    $\eta_{\infty f/c/t}$ & 0.90 \\
    $\eta_i$ & 0.95 \\
    $BPR$ & 6.2 \\
    $FPR=\frac{P_{02}}{P_{01}}$ & 1.55 \\
    $OPR=\frac{P_{03}}{P_{01}}$ & 34 \\
    $T_{04}$ (K) & 1350 \\
    $\dot{m}$ (kg/s) & 220 \\
    $\Delta P_b/P_{02}$ (\%) & 6 \\
    $r=\frac{P_{03}}{P_{02}}=\frac{OPR}{FPR}$ & 21.94
\end{tabular} \\
$\dot{m}_c=\frac{220*6.2}{6.2 + 1}=189.44$ kg/s \\
$\dot{m}_h=\frac{220}{6.2 + 1}=30.56$ kg/s
\end{center}

\subsection*{Part A}
\subsubsection*{Ambient}
find conditions at altitude: \\
$\theta=0.7523$, $\delta=0.224$ (from table A.2) \\
$P_a=0.224(1.01325)=0.227$ bar, $T_a=0.7523(288.15)=216.76$ K,
$a=340.3\sqrt{0.7523}=295.16$ m/s \\
$C_a=295.16*0.85=250.9$ m/s

\subsubsection*{Inlet}
find $P_{01}$: \\
$P_{01}/P_a=(1+\eta_i\frac{\gamma_c-1}{2}M^2)^\frac{\gamma_c}{\gamma_c-1}=
(1+0.95\frac{0.4}{2}(0.85)^2)^\frac{1.4}{0.4}=1.569$ \\
$P_{01}=0.3561$ bar \\\\
inlet is adiabatic: \\
$T_{01}=T_1(1+\frac{\gamma_c-1}{2}M^2)=216.76(1+\frac{0.4}{2}(0.85)^2)=248.08$ K

\subsubsection*{Fan}
$\frac{n-1}{n}=\frac{\gamma_c-1}{\gamma_c\eta_{\infty c}}=\frac{0.4}{1.4(0.9)}=0.3175$ \\
$\frac{T_{02}}{T_{01}}=FPR^{(\frac{n-1}{n})}=1.55^{0.3175}=1.149$ \\
$T_{02}=285.04$ K \\\\
$P_{02}=0.3561*1.55=0.552$ bar

\subsubsection*{Compressor}
$P_{03}=rP_{02}=21.94(0.552)=12.11$ bar \\\\
$\frac{n-1}{n}=\frac{\gamma_c-1}{\gamma_c \eta_{\infty c}}=\frac{0.4}{1.4(0.9)}=
0.3175$ \\
$T_{03}/T_{02}=(r)^\frac{n-1}{n}=21.94^{0.3175}=2.666$ \\
$T_{03}=759.92$ K

\subsubsection*{Combustor}
account for pressure loss in combustor: $P_{04}=P_{03}(0.94)=12.11(0.94)=11.38$ \\
$P_{04}=11.38$ bar

\subsubsection*{High Pressure Turbine}
$\eta_m \dot{m}_h c_{pg}(T_{04}-T_{05})=\dot{m}_h c_{pc}(T_{03}-T_{02})$ \\
$T_{04}-T_{05}=\frac{c_{pc}}{\eta_m c_{pg}}(T_{03}-T_{02})=\frac{1.005}{0.99(1.148)}
(759.92-285.04)=419.93$ \\
$T_{05}=930.07$ K \\\\
$\frac{m-1}{m}=\frac{\eta_{\infty t}(\gamma_g-1)}{\gamma_g}=\frac{0.9(0.333)}{1.333}=
0.2248$ \\
$\frac{P_{05}}{P_{04}}=\frac{T_{05}}{T_{04}}^\frac{m}{m-1}=\frac{930.07}{1350}^{1/0.2248}
=0.1912$ \\
$P_{05}=2.176$ bar

\subsubsection*{Low Pressure Turbine}
$\eta_m \dot{m}_h c_{pg}(T_{05}-T_{06})=\dot{m} c_{pc}(T_{02}-T_{01})$ \\
$T_{05}-T_{06}=\frac{\dot{m}}{\dot{m}_h}\frac{c_{pc}}{\eta_m c_{pg}}(T_{02}-T_{01})=
\frac{220}{30.56}\frac{1.005}{0.99(1.148)}(285.04-248.08)=235.28$ \\
$T_{06}=694.8$ K \\\\
$\frac{P_{06}}{P_{05}}=(\frac{T_{06}}{T_{05}})^\frac{m}{m-1}=(\frac{694.8}{930.07})^
{1/0.2248}=0.2733$ \\
$P_{06}=0.595$ bar

\subsubsection*{Nozzle}
$NPR=\frac{P_{06}}{P_a}=\frac{0.595}{0.227}=2.621$ \\
$PR_{crit}=\frac{1}{[1-\frac{\gamma_g-1}{\gamma_g+1}]^\frac{\gamma_g}{\gamma_g-1}}=
1.852$ \\
$NPR>PR_{crit} \rightarrow$ nozzle is choked \\
$P_7=\frac{P_{06}}{PR_{crit}}=\frac{0.595}{1.852}=0.321$ bar \\\\
$TR_{crit}=1.167$ \\
$T_7=\frac{T_{06}}{TR_{crit}}=\frac{694.8}{1.167}=595.37$ K \\\\
$C_7=1\sqrt{\gamma_g RT_{7}}=\sqrt{1.148(287)(595.37)}=442.9$ m/s \\
$\rho_7=\frac{P_7}{RT_7}=\frac{0.321}{287(595.37)}*100000=0.1879$ kg/m$^3$ \\
$A_7=\frac{\dot{m}_h}{\rho_7 C_7}=\frac{30.56}{0.1879(442.9)}=0.3672$ m$^2$ \\\\
$F_c=\dot{m}_h(C_7-C_a)+A_7(P_7-P_a)*100000=\\
30.56(442.9-250.9)+0.3672(0.321-0.227)*100000=9319.2$ N

\subsubsection*{Fan Nozzle}
$PR_{crit}=\frac{1}{[1-(\frac{\gamma_c-1}{\gamma_c+1})]
^\frac{\gamma_c}{\gamma_c-1}}=\frac{1}{[1-(\frac{0.4}{2.4})]
^\frac{1.4}{0.4}}=1.893$ \\
$\frac{P_{02}}{P_a}=\frac{0.552}{0.227}=2.432$ \\
$NPR>PR_{crit} \rightarrow$ nozzle is choked \\
$P_8=\frac{P_{02}}{PR_{crit}}=\frac{0.552}{1.893}=0.2916$ bar \\\\
$TR_{crit}=1.167$ \\
$T_8=\frac{T_{02}}{TR_{crit}}=\frac{285.04}{1.167}=244.25$ K \\\\
$C_8=1\sqrt{\gamma_c RT_8}=\sqrt{1.4(287)(244.25)}=313.272$ m/s \\
$\rho_8=\frac{P_8}{RT_8}=\frac{0.2916}{287(244.25)}*100000=0.416$ kg/m$^3$ \\
$A_8=\frac{\dot{m}_c}{\rho_8C_8}=\frac{189.44}{0.416(313.272)}=1.454$ m$^2$ \\\\
$F_f=\dot{m}_c(C_8-C_a)+A_8(P_8-P_a)*100000=\\
189.44(313.272-250.9)+1.454(0.2916-0.227)*100000=21208.6$ N

\subsubsection*{Thrust and TSFC}
$F_t=21208.6+9319.2=30527.8$ \\
$\fbox{$F_{total}=30527.8$ N}$ \\\\
$f_a=\frac{c_{pg}T_{04}-c_{pc}T_{03}}{\eta_b(Q_f-c_{pg}T_{04})}=\frac{1.148(1350)
-1.005(759.92)}{0.99(43100-1.148(1350))}=0.0191$ \\
$TSFC=\frac{f\dot{m}}{(1+B)F}=\frac{0.0191(220)}{(1+6.2)30527.8}*3600=0.0688$ \\
$\fbox{$TSFC=0.0688$ kg/N-hr}$

\subsection*{Part B}
I would recommend an increase in bypass ratio as that would increase the thrust and lower
the SFC of the engine. This also would allow for the core to stay very similar to the
original so it would be easier to make new engines as they would not need to be
done from scratch.

\end{document}