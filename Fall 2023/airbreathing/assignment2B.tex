\documentclass{article}
\usepackage{graphicx}
\graphicspath{{./images/A2B}}
\usepackage[legalpaper, portrait, margin=1in]{geometry}

\title{AEEM4063 - Assignment 2B}
\author{Slade Brooks}
\date{09.29.2023}

\begin{document}
\maketitle

\section*{Problem 1}
Simple turbojet is operating with a compressor pressure ratio of 8.0, a turbine
inlet temperature of 1200 K, and a mass flow of 15 kg/s, when the aircraft is flying at 260
m/s at an altitude of 7,000 m. Assuming the following component efficiencies, and ISA
conditions, calculate the propelling nozzle area required, the net thrust developed, and
the SFC.
\begin{center}
\begin{tabular}{cc}
    r & 8 \\
    $T_{03}$ (K) & 1200 \\
    $\dot{m}$ (kg/s) & 15 \\
    $C_a$ (m/s) & 260 \\
    alt (m) & 7000 \\
    $\eta_{\infty c}$ & 0.87 \\
    $\eta_{\infty t}$ & 0.87 \\
    $\eta_i$ & 0.95 \\
    $\eta_j$ & 0.95 \\
    $\eta_m$ & 0.99 \\
    $\Delta P_b/P_{02}$ (\%) & 6 \\
    $\eta_b$ & 0.97
\end{tabular}
\end{center}
\subsubsection*{Ambient}
find conditions at altitude: \\
$\theta=0.8423$, $\delta=0.4057$ (from table A.2) \\
$P_a=0.4057*1.01325=0.411$ bar, $T_a=0.8423*288.15=242.71$ K, $a=340.3*\sqrt{0.8423}=312.317$ m/s \\
$M=\frac{V}{a}=\frac{260}{312.317}=0.8325$

\subsubsection*{Inlet}
find $P_{01}$: \\
$P_{01}/P_a=(1+\eta_i\frac{\gamma_c-1}{2}M^2)^\frac{\gamma_c}{\gamma_c-1}=
(1+0.95\frac{0.4}{2}(0.8325)^2)^\frac{1.4}{0.4}=1.542$ \\
$P_{01}=0.6338$ bar \\\\
intake is adiabatic so: $T_{01}=T_{0a}=T_a*(1+\frac{\gamma_c-1}{2}M^2)=
288.15(1+\frac{0.4}{2}(0.8325)^2)$ \\
$T_{01}=276.35$ K

\subsubsection*{Compressor}
$P_{02}=rP_{01}=8(0.6338)=5.0704$ bar \\\\
$\frac{n-1}{n}=\frac{\gamma_c-1}{\gamma_c\eta_{\infty c}}=\frac{0.4}{1.4(0.87)}=0.3284$ \\
$T_{02}/T_{01}=(r)^\frac{n-1}{n}=8^{0.3284}=1.98$ \\
$T_{02}=547.17$ K

\subsubsection*{Combustor}
account for pressure loss in combustor: $P_{03}=P_{02}(1-\frac{\Delta P_b}{P_{02}})=
5.0704(1-0.06)=4.766$ bar \\
$P_{03}=4.766$ bar

\subsubsection*{Turbine}
$W_c=\frac{c_{pc}}{\eta_m}(T_{02}-T_{01})=\frac{1.005}{0.99}(547.17-276.35)=275$ kJ/kg \\
$W_t=W_c=275$ kJ/kg $=c_{pg}(T_{03}-T_{04})=1.148(1200-T_{04})=275$ \\
$T_{04}=960.45$ K \\\\
$\frac{m-1}{m}=\frac{\eta_{\infty t}(\gamma_g-1)}{\gamma_g}=\frac{0.87(0.333)}{1.333}=0.2173$ \\
$\frac{T_{04}}{T_{03}}=(\frac{P_{04}}{P_{03}})^\frac{m-1}{m} \rightarrow \frac{960.45}{1200}=
(\frac{P_{04}}{P_{03}})^{0.2173} \rightarrow \frac{P_{04}}{P_{03}}=0.359$ \\
$P_{04}=1.71$ bar

\subsubsection*{Nozzle}
$NPR=\frac{P_{04}}{P_a}=\frac{1.71}{0.411}=4.16$ \\
$PR_{crit}=\frac{1}{[1-\frac{1}{\eta_j}(\frac{\gamma_g-1}{\gamma_g+1})]
^\frac{\gamma_g}{\gamma_g-1}}=\frac{1}{[1-\frac{1}{0.95}(\frac{0.333}{2.333})]
^\frac{1.333}{0.333}}=1.92$ \\
$NPR>PR_{crit} \rightarrow$ nozzle is choked \\
$P_5=\frac{P_{04}}{PR_{crit}}=\frac{1.71}{1.92}=0.89$ bar \\\\
$TR_{crit}=\frac{\gamma_g+1}{2}=\frac{2.333}{2}=1.167$ \\
$T_5=\frac{T_{04}}{TR_{crit}}=\frac{960.45}{1.167}=823$ K \\\\
$C_5=1\sqrt{\gamma_gRT_5}=\sqrt{1.333*287*823}=561.12$ m/s \\
$\rho_5=\frac{P_5}{RT_5}=\frac{0.89}{287*823}*100000=0.377$ kg/m$^3$ \\
$\dot{m}=\rho_5C_5A_5 \rightarrow A_5=\frac{\dot{m}}{\rho_5C_5}=\frac{15}{0.377*561.12}
=0.0709$ m$^2$ \\
$\fbox{$A_5=0.0709$ m$^2$}$ \\\\
$F=\dot{m}(C_5-C_a)+A_5(P_5-P_a)=15(561.12-260)+0.0709(0.89-0.411)*100000=7913$ N \\
$\fbox{$F=7913$ N}$ \\\\
$f_a=\frac{c_{pg}T_{03}-c_{pc}T_{02}}{\eta_b(Q_f-c_{pg}T_{03})}=
\frac{1.148(1200)-1.005(547.17)}{0.99(43100-1.148(1200))}=0.02$ \\
$F_s=\frac{F}{\dot{m}}=\frac{7913}{15}=527.53$ Ns/kg \\
$TSFC=\frac{f}{F_s}=\frac{0.02}{527.53}*3600=0.1365$ kg/hrN \\
$\fbox{$TSFC=0.1365$ kg/hrN}$

\section*{Problem 2}
Under take-off conditions when the ambient pressure and temperature are 1.01
bar and 288 K, the stagnation pressure and temperature in the jet pipe of a turbojet
engine are 3.8 bar and 1000 K, and the mass flow is 23 kg/s. Calculate:

a. The exit area required and the thrust produced, assuming an isentropic
convergent nozzle. (Use stations 7 \& 9 for inlet and exit of nozzle)

b. The nozzle exit area and thrust produced, assuming an isentropic convergent-
divergent nozzle that is fully expanded. (Use stations 7, 8, \& 9 for inlet, throat, and
exit of nozzle)

c. Comment on which nozzle you would choose if you were the design engineer
responsible for selecting the type of nozzle to utilize on an aircraft. \\\\
$P_a=1.01$ bar, $T_a=288$ K, $P_{07}=3.8$ bar, $T_{07}=1000$ K, $\dot{m}=23$ kg/s

\subsection*{Part A}
$NPR=\frac{P_{07}}{P_a}=\frac{3.8}{1.01}=3.76$ \\
$PR_{crit}=\frac{1}{[1-(\frac{\gamma_g-1}{\gamma_g+1})]
^\frac{\gamma_g}{\gamma_g-1}}=\frac{1}{[1-(\frac{0.333}{2.333})]
^\frac{1.333}{0.333}}=1.85$ \\
$NPR>PR_{crit} \rightarrow$ nozzle is choked \\
$P_9=\frac{P_{07}}{PR_{crit}}=\frac{3.8}{1.85}=2.054$ bar \\\\
$TR_{crit}=1.167$ \\
$T_9=\frac{T_{07}}{TR_{crit}}=\frac{1000}{1.167}=856.9$ K \\\\
$C_9=1\sqrt{\gamma_gRT_9}=\sqrt{1.148(287)(856.9)}=531.35$ m/s \\
$\rho_9=\frac{P_9}{RT_9}=\frac{2.054}{287(856.9)}*100000=0.8352$ kg/m$^3$ \\
$A_9=\frac{\dot{m}}{\rho_9C_9}=\frac{23}{0.8352(531.35)}=0.0518$ m$^2$ \\
$\fbox{$A_9=0.0518$ m$^2$}$ \\\\
$F=\dot{m}(C_9-C_a)+A_9(P_9-P_a)=23(531.35-0)+0.0518(2.054-1.01)*100000=17629$ N \\
$\fbox{$F=17629$ N}$

\subsection*{Part B}

\subsection*{Part C}

\end{document}