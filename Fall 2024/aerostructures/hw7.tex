\documentclass[12 pt]{article}
\usepackage[utf8]{inputenc}
\usepackage{graphicx}
\usepackage{amsmath}
\usepackage[version=4]{mhchem}
\usepackage{siunitx}
\usepackage{longtable,tabularx}
\usepackage{paracol}
\usepackage{float}
\usepackage[left=1in, right=1in]{geometry}

\title{AEEM5058 HW\#7}
\date{11.12.24}
\author{Slade Brooks \\ M13801712}

\begin{document}
\maketitle

\section*{Problem 1}
The structure shown in Fig. 1 consists of two horizontal flanges ABC and FED, three vertical
stiffeners, AF, BE, and CD, and two thin-walled panels ABEF and BCDE, the length scales are $a
=1000$ mm and $b =2000$ mm. The structure is simply supported in F and E and is loaded at corner
C by a force $F =6$ kN and forming the angle $\theta = 30^{\circ}$ relative to flange ABF. Assuming that the
normal stresses are resisted by the flanges and stiffeners while the panels are effective only in
shear, calculate and sketch the distribution of the normal force, N, in all flanges and stiffeners and
the shear flows in the panels.
\begin{figure}[H]
    \centering
    \includegraphics[width=0.7\linewidth]{figs/hw7fig1.png}
\end{figure}

\pagebreak
\subsection*{}
\begin{figure}[H]
    \centering
    \includegraphics[width=0.7\linewidth]{figs/hw7fig3.png}
\end{figure}
First do moment balance around point E.
\begin{align*}
    bF_y=aR_{Fy}+aF_x \rightarrow R_{Fy}=(bF_y-aF_x)/a \\
    F_y=F\sin{\theta}; \quad F_x=F\cos{\theta} \\
    R_{Fy}=0.804\textrm{kN}
\end{align*}

\subsection*{Panel BCDE}
Balance forces within panel BCDE:
\begin{align*}
    aq_1=F_x=F\sin{\theta} \\
    q_1=0.003 \textrm{kN/mm}
\end{align*}

\subsection*{Panel ABEF}
Balance forces within panel ABEF:
\begin{align*}
    aq_2=-R_{fy} \\
    q_2=-8.04E-4 \textrm{kN/mm}
\end{align*}

\subsection*{Member BE}
Balance forces in the member BE:
\begin{align*}
    a(q_2-q_1)=R_{ey} \\
    R_{Ey}=-3.8\textrm{kN}
\end{align*} \\
We know that $R_{Ex}$ must be equal and opposite to the horizontal force as it is the only horizontal reaction force.
\begin{align*}
    R_{Ex}=-F\cos{\theta}=-5.2\textrm{kN}
\end{align*}
\begin{figure}[H]
    \centering
    \includegraphics[width=0.7\linewidth]{figs/hw7fig4.png}
\end{figure}
Lastly, do moment balance with at both ends of each panel to determine the components:
\subsection*{Panel ABEF}
\begin{align*}
    \textrm{@E:} \quad 0=aR_{AB}+aR_{Fy} \rightarrow R_{AB}=-804\textrm{N} \\
    \textrm{@B:} \quad aR_{FE}=aR_{Fy} \rightarrow R_{FE}=804\textrm{N}
\end{align*}
\subsection*{Panel BCDE}
\begin{align*}
    \textrm{@B:} \quad bF\sin{\theta}=aR_{ED} \rightarrow R_{ED}=6000\textrm{N} \\
    \textrm{@E:} \quad aR_{BC}+bF\sin{\theta}=aF\cos{\theta} \rightarrow R_{BC}-804\textrm{N}
\end{align*}
\begin{figure}[H]
    \centering
    \includegraphics[width=0.95\linewidth]{figs/hw7fig5.png}
\end{figure}
\begin{figure}[H]
    \centering
    \includegraphics[width=0.7\linewidth]{figs/hw7fig6.png}
\end{figure}

\pagebreak
\section*{Problem 2}
A thin-walled column of length 4 m has the cross section shown in Fig. 2. If the ends of the column
are pinned and free to warp, calculate the buckling load and specify the mode of failure. Use $E =
70$GPa and $G=30$GPa.
\begin{figure}[H]
    \centering
    \includegraphics[width=0.7\linewidth]{figs/hw7fig2.png}
\end{figure}

\subsection*{}
Find area and centroid of the cross section.
\begin{align*}
    A=bt+w(h-t)=275\textrm{mm}^2 \\
    y_c=0\textrm{mm} \\
    z_c=\frac{1}{A}\left[bt(h-\frac{t}{2}+w(h-t)\frac{h-t}{2})\right]=49.03\textrm{mm}
\end{align*} \\
Then determine the moments of inertia.
\begin{align*}
    I_y=\frac{t^3b}{12}+bt(h-t+\frac{t}{2}-z_c)^2+\frac{(h-t)^3w}{12}+(h-t)w\left(\frac{h-t}{2}-z_c\right)^2=166532\textrm{mm}^4 \\
    I_z=\frac{b^3t}{12}+\frac{w^3(h-t)}{12}=11081\textrm{mm}^4
\end{align*} \\
The shear center can be determined via symmetry and based on the intersection of horizontal and vertical beams w.r.t the
centroid:
\begin{align*}
    y_{sc}=0\textrm{mm} \\
    z_{sc}=h-\frac{t}{2}-z_c=24.72\textrm{mm}
\end{align*} \\
Calculate polar moment, warping modulus, and polar moment of inertia:
\begin{align*}
    I_o=Ay_{sc}^2+Az_{sc}^2+I_y+I_z=345404\textrm{mm}^4 \\
    I_w=\frac{b^3t^3}{144}+\frac{h^3w^3}{36}=188828\textrm{mm}^4 \\
    J=\frac{1}{3}\int_0^st(s)ds=\frac{bt^3}{3}+\frac{(h-t)w^3}{3}=573\textrm{mm}^4
\end{align*}
Then determine the critical P from equations for a simply supported beam:
\begin{align*}
    P_{cry}=\frac{\pi^2EI_y}{L^2}=7.2\textrm{kN} \\
    P_{crz}=\frac{\pi^2EI_z}{L^2}=0.48\textrm{kN} \\
    P_{cr\theta}=\frac{A}{I_o}\left(GJ+\frac{\pi^2EI_w}{L^2}\right)=13.7\textrm{kN}
\end{align*}
Lastly, set up and solve the flexural-torsional load for P in matlab:
\begin{align*}
    \begin{bmatrix}
        P-P_{cry} & 0 & -Py_{sc} \\
        0 & P-P_{crz} & Pz_{sc} \\
        -Py_{sc} & Pz_{sc} & \frac{I_o}{A}(P-P_{cr\theta})
    \end{bmatrix}\begin{bmatrix}
        A_1 \\ A_2 \\ A_3
    \end{bmatrix}=\begin{bmatrix}
        0 \\ 0 \\ 0
    \end{bmatrix} \rightarrow P=0.47\textrm{kN}
\end{align*}
Since P is less than all of the other critical P values, $\fbox{buckling load$=0.47$ kN}$ and the failure mode is a
combination of bending and torsion.

\end{document}