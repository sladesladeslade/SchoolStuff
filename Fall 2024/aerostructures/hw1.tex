\documentclass[12 pt]{article}
\usepackage[utf8]{inputenc}
\usepackage{graphicx}
\usepackage{amsmath}
\usepackage[version=4]{mhchem}
\usepackage{siunitx}
\usepackage{longtable,tabularx}
\usepackage{float}
\usepackage[left=1in, right=1in]{geometry}

\title{AEEM5058 HW\#1}
\date{09.10.24}
\author{Slade Brooks \\ M13801712}

\begin{document}
\maketitle

\section*{Problem 1}
The thin-walled cross section shown in Fig.~1 has uniform wall thickness t=1.7 mm. It consists of three of circular arcs
of radii R, R/2 and 2R, and a straight segment of length R-the length R being 200~mm. The dimensions shown in Fig.~1 can
be assumed to correspond to the centerline of the cross section.
\begin{figure}[!hbtp]
    \centering
    \includegraphics[width=0.5\linewidth]{figs/hw1fig1.png}
\end{figure} \par

\subsection*{a)}
Find the coordinates of the centroid relative to the frame (u, v) with origin coincident with the center of the arc of
radius R.
\begin{figure}[!h]
    \centering
    \includegraphics[width=0.6\linewidth]{figs/hw1fig2.png}
\end{figure} \\ \\
First define the different s directions to analyze and their limits. The limits for $s_2$, $s_3$, and $s_4$ can be
determined by the circumference of the circle each one is a part of, then dividing by the amount of circle there is.
\begin{align*}
    C = \pi D \\
    s_2: \quad C = 4\pi R \rightarrow s_{2_{lim}}=\pi R \\
    s_3: \quad C = \pi R \rightarrow s_{3_{lim}}=\frac{\pi R}{2} \\
    s_4: \quad C = 2\pi R \rightarrow s_{4_{lim}}=\frac{3\pi R}{2}
\end{align*} \\
\begin{tabular}{|c|c|c|c|}
    \hline
    s & u(s) & v(s) & bounds \\ \hline
    $s_1$ & 0 & $R+s_1$ & $0 \rightarrow R$ \\
    $s_2$ & $-2R\sin{\frac{s_2}{2R}}$ & $2R\cos{\frac{s_2}{2R}}$ & $0 \rightarrow \pi R$ \\
    $s_3$ & $-\frac{3R}{2} - \frac{R}{2}\cos{\frac{2s_3}{R}}$ & $\frac{R}{2}\sin{\frac{2s_3}{R}}$ & $0 \rightarrow
    \frac{\pi R}{2}$
    \\
    $s_4$ & $-R\cos{\frac{s_4}{R}}$ & $-R\sin{\frac{s_4}{R}}$ & $0 \rightarrow \frac{3\pi R}{2}$ \\
    \hline
\end{tabular} \\ \\
Create the integrals to solve for the coordinates of the centroid with respect to u and v. Simplify, solve, and plug in
the known values to determine the coordinates of the centroid.
\begin{align*}
    u_c=\frac{\int_{l}{u(s)t(s) ds}}{\int_{l}{t(s) ds}}=
    \frac{\int_{s_1}{0} + \int_{s_2}{-2R\sin{\frac{s}{2R}}tds} + \int_{s_3}{(-\frac{3R}{2}-\frac{R}{2}\cos{\frac{2s}{R}})tds}
    + \int_{s_4}{(-R\cos{\frac{s}{R}})tds}}{\int_{s_1}{tds} + \int_{s_2}{tds} + \int_{s_3}{tds} + \int_{s_4}{tds}} \\
    =\frac{-2Rt\int_{0}^{\pi R}{\sin{\frac{s}{2R}ds}} + t\int_{0}^{\frac{\pi R}{2}}{(-\frac{3R}{2}-\frac{R}{2}\cos{\frac{2s}{R}})ds}
    -Rt\int_{0}^{\frac{3\pi R}{2}}{\cos{\frac{s}{R}}ds}}{t\int_{0}^{R}{ds} + t\int_{0}^{\pi R}{ds} + t\int_{0}^{\frac{\pi R}{2}}{ds}
    + t\int_{0}^{\frac{3\pi R}{2}}{ds}} \\
    = \frac{-4R^2t-\frac{3\pi R^2t}{4}+R^2t}{tR + \pi Rt + \frac{\pi Rt}{2} + \frac{3\pi Rt}{2}} = -\frac{3(\pi+4)R}{4(3\pi+1)}
    = -102.759 \textrm{mm}
\end{align*}
\begin{align*}
    v_c=\frac{\int_{l}{v(s)t(s) ds}}{\int_{l}{t(s) ds}} =
    \frac{\int_{s_1}{(R+s)tds} + \int_{s_2}{2R\cos{\frac{s}{2R}}tds} + \int_{s_3}{\frac{R}{2}\sin{\frac{2s}{R}}tds} +
    \int_{s_4}{-R\sin{\frac{s}{R}}tds}}{\int_{s_1}{tds} + \int_{s_2}{tds} + \int_{s_3}{tds} + \int_{s_4}{tds}} \\
    =\frac{t\int_{0}^{R}{(R+s)ds} + 2Rt\int_{0}^{\pi R}{\cos{\frac{s}{2R}}ds} + \frac{Rt}{2}\int_{0}^{\frac{\pi R}{2}}
    {\sin{\frac{2s}{R}}ds} - Rt\int_{0}^{\frac{3\pi R}{2}}{\sin{\frac{s}{R}}ds}}{t\int_{0}^{R}{ds} + t\int_{0}^{\pi R}{ds}
    + t\int_{0}^{\frac{\pi R}{2}}{ds} + t\int_{0}^{\frac{3\pi R}{2}}{ds}} \\
    =\frac{\frac{3tR^2}{2} + 4R^2t + \frac{R^2t}{2} - R^2t}{tR + \pi Rt + \frac{\pi Rt}{2} + \frac{3\pi Rt}{2}}
    = \frac{5R}{1+3\pi} = 95.925\textrm{mm}
\end{align*} \\ \\
\begin{center} $\fbox{$u_c$, $v_c$ = (-102.759, 95.925) mm}$ \end{center}
\begin{figure}[H]
    \centering
    \includegraphics[width=0.6\linewidth]{figs/hw1fig3.png}
\end{figure}

\subsection*{b)}
Determine the three second moment of area relative to a reference frame with axes parallel to the frame (u,v) and with
origin coincident with the centroid. \\ \\
We first define the reference frame $(\eta, \zeta)$ through the centroid determined in part a and parallel to u and v.
Then, we will set up the integrals to determine the 3 second moments of area relative to u and v, using the s, u(s), and
v(s) defined in part a. We solve the integral, simplify, then plug in and solve.
\begin{align*}
    I_{uu}=\int_{l}{v(s)^2tds}=\int_{s_1}{(R+s)^2tds} + \int_{s_2}{(2R\cos{\frac{s}{2R}})^2tds}
    + \int_{s_3}{(\frac{R}{2}\sin{\frac{2s}{R}})^2tds} + \int_{s_4}{(-R\sin{\frac{s}{R}})^2tds} \\
    =t\int_{0}^{R}{(R+s)^2ds} + t\int_{0}^{\pi R}{(2R\cos{\frac{s}{2R}})^2ds}
    + t\int_{0}^{\frac{\pi R}{2}}{(\frac{R}{2}\sin{\frac{2s}{R}})^2ds} + t\int_{0}^{\frac{3\pi R}{2}}{(-R\sin{\frac{s}{R}})^2ds}
    \\ = \frac{7R^3t}{3} + 2\pi R^3t + \frac{\pi R^3t}{16} + \frac{3\pi R^3t}{4} = \frac{112R^3t + 135\pi R^3t}{48}
    = 151899252.3 \textrm{mm}^4
\end{align*}
\begin{align*}
    I_{vv}=\int_{l}{u(s)^2tds}=\int_{s_1}{(0)^2tds} + \int_{s_2}{(-2R\sin{\frac{s}{2R}})^2tds}
    + \int_{s_3}{(-\frac{3R}{2}-\frac{R}{2}\cos{\frac{2s}{R}})^2tds} + \int_{s_4}{(-R\cos{\frac{s}{R}})^2tds} \\
    = t\int_{0}^{\pi R}{(-2R\sin{\frac{s}{2R}})^2ds} + t\int_{0}^{\frac{\pi R}{2}}{(-\frac{3R}{2}-\frac{R}{2}\cos{\frac{2s}{R}})^2ds}
    + t\int_{0}^{\frac{3\pi R}{2}}{(-R\cos{\frac{s}{R}})^2ds} \\
    = 2\pi R^3t + \frac{19\pi R^3t}{16} + \frac{3\pi R^3t}{4} = \frac{63\pi R^3t}{16} = 168232286.6 \textrm{mm}^4
\end{align*}
\begin{align*}
    I_{uv}=\int_{l}{u(s)v(s)tds}=\int_{s_1}{(0)(R+s)tds} + \int_{s_2}{(2R\cos{\frac{s}{2R}})(-2R\sin{\frac{s}{2R}})tds} \\
    + \int_{s_3}{(\frac{R}{2}\sin{\frac{2s}{R}})(-\frac{3R}{2}-\frac{R}{2}\cos{\frac{2s}{R}})tds}
    + \int_{s_4}{(-R\sin{\frac{s}{R}})(-R\cos{\frac{s}{R}})tds} \\
    = -4R^3t - \frac{3R^3t}{4} + \frac{R^3t}{2} = -\frac{17R^3t}{4} = -57800000 \textrm{mm}^4
\end{align*} \\ \\
Then, we will use transport theorem to determine the moments of area about the desired axes. The minimum distance will
be the coordinates of the centroid. We will determine the total cross sectional area of the shape by multiplying the
total length, the sum of all the s ranges, by the thickness.
\begin{align*}
    A = t(s_1+s_2+s_3+s_4)=t(R+\pi R+\frac{\pi R}{2}+\frac{3\pi R}{2}) = 3544.425 \textrm{mm}^2
\end{align*} \\ \\ 
Then we can solve for the moment of area using the transport theorem.
\begin{align*}
    I_{x'} = I_x + Ah^2 \rightarrow \textrm{solve for $I_x$ because we have $I_{x'}$, the one not through the centroid}\\
    I_\eta = I_u - Ah^2 = 151899252.3 - 3544.425(95.925)^2 = \fbox{$I_\eta = 119284851.3 \textrm{mm}^4$} \\
    I_\zeta = I_z - Ah^2 = 168232286.6 - 3544.425(-102.759)^2 = \fbox{$I_\zeta = 130805242.4 \textrm{mm}^4$} \\
    I_{\eta \zeta} = I_{uv} - Auv = -57800000 - 3544.425(95.925)(-102.759) = \fbox{$I_{\eta \zeta} = -22862046.03 \textrm{mm}^4$}
\end{align*}

\end{document}