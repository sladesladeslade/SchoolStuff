\documentclass[12 pt]{article}
\usepackage[utf8]{inputenc}
\usepackage{graphicx}
\usepackage{amsmath}
\usepackage[version=4]{mhchem}
\usepackage{siunitx}
\usepackage{longtable,tabularx}
\usepackage{paracol}
\usepackage{float}
\usepackage[left=1in, right=1in]{geometry}

\title{AEEM5058 HW\#6}
\date{11.03.24}
\author{Slade Brooks \\ M13801712}

\begin{document}
\maketitle

\section*{Problem 1}
A wing of length L =2 m has a semi-monocoque structure and undergoes a laboratory test in which
it is clamped at one end and subject to a concentrated load F= 18 kN applied at the wing tip as
shown in Fig. 1.
\begin{figure}[!hbtp]
    \centering
    \includegraphics[width=0.5\linewidth]{figs/hw6fig1.png}
\end{figure}
The wing is rectangular and has the cross section shown in Fig. 2 which consists of a uniform
external skin of thickness $t_s=0.75$ mm and three vertical webs of thickness $t_w = 1.5$ mm. The
section has stiffeners at nodes \#1 to \#6 with the cross sectional areas indicated in Fig. 2. All the
structural elements are made of the same aluminum alloy that has shear modulus G = 26.9 GPa.
Moreover, for a preliminary stress analysis, it is assumed that the skin and webs resist to shear
stresses only while the normal stresses are absorbed by the stiffeners that can therefore be
considered as booms according to the structural idealization theory.
\begin{figure}[H]
    \centering
    \includegraphics[width=0.9\linewidth]{figs/hw6fig2.png}
\end{figure}

\pagebreak
\subsection*{(a)}
Determine the largest tensile and compressive stresses in the section. \\ \\
First, determine the centroid location and the moments of inertia based on only the booms (because the skin does not
contribute to the axial load) w/ origin of u-v set at boom 6. \\
\begin{align*}
    u_c=\frac{\Sigma_{i=1}^{N}u_iB_i}{\Sigma_{i=1}^{N}B_i}=\frac{(0*600)+(400*300)+(760*300)+(760*300)+(400*400)+(0*400)}
    {600+300+300+300+400+400} \\
    u_c=320\textrm{mm} \\
    v_c=\frac{\Sigma_{i=1}^{N}v_iB_i}{\Sigma_{i=1}^{N}B_i}=\frac{(250*600)+(250*300)+(175*300)+(0*300)+(0*400)+(0*400)}
    {600+300+300+300+400+400} \\
    v_c=120.65\textrm{mm}
\end{align*}
\begin{align*}
    I_{\eta}=\Sigma_{i=1}^{N}(v_i-v_c)^2B_i=(250-120.65)^2(600)+(250-120.65)^2(300)+(175-120.65)^2(300) \\
    +(0-120.65)^2(300)
    +(0-120.65)^2(400)+(0-120.65)^2(400) \\
    I_{\eta}=3.2E7 \textrm{mm}^4 \\ \\
    I_{\zeta}=\Sigma_{i=1}^{N}(u_i-u_c)^2B_i=(0-320)^2(600)+(400-320)^2(300)+(760-320)^2(300) \\
    +(760-320)^2(300)+(400-320)^2(400)+(0-320)^2(400) \\
    I_{\zeta}= 22.3E7 \textrm{mm}^4 \\ \\
    I_{\eta\zeta}=\Sigma_{i=1}^{N}(u_i-u_c)(v_i-v_c)B_i=(0-320)(250-120.65)(600)+(400-320)(250-120.65)(300) \\
    +(760-320)(175-120.65)(300)+(760-320)(0-120.65)(300) \\
    +(400-320)(0-120.65)(400)+(0-320)(0-120.65)(400) \\
    I_{\eta\zeta}=-1.89E7 \textrm{mm}^4
\end{align*} \\
Then determine shear in each direction based on M and right hand rule (see diagram below). M is simply the length
multiplied by the force applied at the free end since it is not distributed. \\
\begin{align*}
    M_{\zeta}=0 \\
    M_{\eta}=M=FL=36000\textrm{kNmm}
\end{align*} \\
Then calculate beta:
\begin{align*}
    \tan{\beta}=\frac{M_{\eta}I_{\eta\zeta}-M_{\zeta}I_{\eta}}{M_{\zeta}I_{\eta\zeta}-M_{\eta}I_{\zeta}}
    =\frac{36000(-1.89E7)-0(3.2E7)}{0(-1.89E7)-36000(22.3E7)} \\ \\
    \beta=4.84^{\circ}
\end{align*} \\
Now we can determine the booms with the maximums based on the diagram are 2 and 6.
\begin{figure}[H]
    \centering
    \includegraphics[width=0.92\linewidth]{figs/hw6fig3.jpg}
\end{figure}
Then we can find the axial force at the two booms from:
\begin{align*}
    \sigma_{xx}=-\frac{M_{\eta}(I_{\zeta}\zeta-I_{\eta \zeta}\eta)}{I_{\eta}I_{\zeta}-I_{\eta \zeta}^2}
    -\frac{M_{\zeta}(I_{\eta}\eta - I_{\eta \zeta}\zeta)}{I_{\eta}I_{\zeta}-I_{\eta \zeta}^2}
\end{align*}
\begin{align*}
    \sigma_2=-\frac{36000(22.3E7*(250-120.65)-(-1.89E7)(400-320))}{3.2E7*22.3E7-(-1.89E7)^2}
    =-0.1612 \textrm{kN/mm$^2$} \\
    \sigma_6=-\frac{36000(22.3E7*(0-120.65)-(-1.89E7)(0-320))}{3.2E7*22.3E7-(-1.89E7)^2}
    =0.175 \textrm{kN/mm$^2$}
\end{align*}
\begin{align*}
    \fbox{$\sigma_2=-161.2$ MPa (compression)} \\
    \fbox{$\sigma_6=175$ MPa (tension)}
\end{align*}

\subsection*{(b)}
Determine the distribution of the shear stress along the skin and webs and indicate in the
cross section the corresponding direction. \\ \\
First we will solve for $\alpha$ and $\beta$ in the shear flow equation ($q_b(s)$). Since there is no $\eta$ component
of force, it is simplified. $V_{\zeta}=-F$ due to the sign conventions of shear flow (down is +).
\begin{align*}
    \alpha = \frac{V_{\eta}I_{\eta}-V_{\zeta}I_{\eta \zeta}}{I_{\eta}I_{\zeta}-I_{\eta \zeta}^2} =
    \frac{-V_{\zeta}I_{\eta \zeta}}{I_{\eta}I_{\zeta}-I_{\eta\zeta}^2}=-5.02E-8 \\
    \beta = \frac{V_{\zeta}I_{\zeta}-V_{\eta}I_{\eta \zeta}}{I_{\eta}I_{\zeta}-I_{\eta \zeta}^2} =
    \frac{V_{\zeta}I_{\zeta}}{I_{\eta}I_{\zeta}-I_{\eta\zeta}^2}=-5.92E-7
\end{align*} \\
Now we can set up and solve for $q_b$ in each section. We will split it into 3 sections and make 3 cuts---one in each
section at a convenient location. \\
\begin{figure}[H]
    \centering
    \includegraphics[width=0.85\linewidth]{figs/hw6fig4.jpg}
\end{figure}
\begin{center}
    \begin{tabular}{|c|c|c|}
        \hline
        I & II & III \\
        \hline
        $q_{bo}^{61}=0$ & $q_b^{12}=0$ & $q_b^{23}=0$ \\
        $q_{bi}^{16}=q_{bo}^{61}+\alpha B_1\eta_1+\beta B_1\zeta_1$ &
        $q_b^{25}=q_b^{12}+\alpha B_2\eta_2+\beta B_2\zeta_2$ &
        $q_b^{34}=q_b^{23}+\alpha B_3\eta_3+\beta B_3\zeta_3$ \\
        & $q_b^{56}=q_b^{25}+\alpha B_5\eta_5+\beta B_5\zeta_5$ &
        $q_b^{45}=q_b^{34}+\alpha B_4\eta_4+\beta B_4\zeta_4$ \\
        & $q_b^{16}=q_b^{12}+\alpha B_1\eta_1+\beta B_1\zeta_1$ &
        $q_b^{25}=q_b^{23}+\alpha B_2\eta_2+\beta B_2\zeta_2$ \\
        \hline
    \end{tabular}
\end{center}
Plug in and solve the previous table to get the $q_b$ in each boom: \\
\begin{center}
    \begin{tabular}{|c|c|c|}
        \hline
        I & II & III \\
        \hline
        $q_{bo}^{61}=0$ & $q_b^{12}=0$ & $q_b^{23}=0$ \\
        $q_{bi}^{16}=-0.0363$ & $q_b^{25}=-0.0242$ & $q_b^{34}=-0.0163$ \\
        & $q_b^{56}=0.0028$ & $q_b^{45}=-0.0015$ \\
        & $q_b^{16}=-0.0363$ & $q_b^{25}=-0.0242$ \\
        \hline
    \end{tabular}
\end{center}
Now set up the rotational equations for each section and solve for twist in terms of q of each cell:
\begin{align*}
    A_I=\frac{\pi R^2}{2}=24543.7; \quad A_{II}=2aR=200000; \quad A_{III}=\frac{2R+c}{2}b=76500 \\
    \delta_{61o}=\frac{2\pi R}{2t_s}=523.6; \quad \delta_{16}=\delta_{25}=\frac{2R}{t_w}=166.67; \quad
    \delta_{12}=\delta_{56}=\frac{a}{t_s}=533.33 \\
    \delta_{34}=\frac{c}{t_w}=116.67; \quad \delta_{45}=\frac{b}{t_s}=480; \quad
    \delta_{23}=\frac{\sqrt{b^2+(2R-c)^2}}{t_s}=490.3
\end{align*}
\subsubsection*{Cell I}
\begin{align*}
    \frac{d\theta}{dz}_I=\frac{1}{2A_IG}(q_I(\delta_{61o}+\delta_{16i})-q_{II}\delta_{16i}+q_{bo}^{61}\delta_{61o}
    +q_{bi}^{16}\delta_{16i})
\end{align*}
\subsubsection*{Cell II}
\begin{align*}
    \frac{d\theta}{dz}_{II}=\frac{1}{2A_{II}G}(-q_I\delta_{16i}+q_{II}(\delta_{12}+\delta_{25}+\delta_{56}+\delta_{61})
    -q_{III}\delta_{25}+q_b^{12}\delta_{12}+q_b^{25}\delta_{25}+q_b^{56}\delta_{56}-q_b^{16}\delta_{16})
\end{align*}
\subsubsection*{Cell III}
\begin{align*}
    \frac{d\theta}{dz}_{III}=\frac{1}{2A_{III}G}(-q_{II}\delta_{25}+q_{III}(\delta_{23}+\delta_{34}+\delta_{45}+\delta_{52})
    +q_b^{23}\delta_{23}+q_b^{34}\delta_{34}+q_b^{45}\delta_{45}-q_b^{25}\delta_{25})
\end{align*} \\
Set moments equal to get final equation:
\begin{align*}
    2A_Iq_I+2A_{II}q_{II}+2A_{III}q_{III}+q_b^{25}*(400-50)*2R+q_b^{34}*(760-50)*175=-q_{bi}^{16}*50*2R
\end{align*}
Use matlab to solve system of equations with the twists equal to each other and the moment equation:
\begin{align*}
    q_I=0.0153 \textrm{kN/mm};\quad
    q_{II}=0.0089 \textrm{kN/mm};\quad
    q_{III}=0.0075 \textrm{kN/mm}
\end{align*}
Determine shear flow in each section via below equation. (also done in matlab for time saving)
\begin{align*}
    q=q_R+q_b \\
    q_{61o}=0.0153 \\
    q_{16i}=-0.03 \\
    q_{12}=0.0089 \\
    q_{25}=-0.0228 \\
    q_{56}=0.0117 \\
    q_{23}=0.0075 \\
    q_{34}=-0.0088 \\
    q_{45}=0.006
\end{align*}
Lastly, divide by the shear flow in each section by its thickness to get the shear force.
\begin{paracol}{2}
    \begin{align*}
        \tau_{61o}=0.0204 \\
        \tau_{16i}=-0.02 \\
        \tau_{12}=0.0119 \\
        \tau_{25}=-0.0152 \\
        \tau_{56}=0.0156 \\
        \tau_{23}=0.01 \\
        \tau_{34}=-0.0058 \\
        \tau_{45}=0.008
    \end{align*}
    \switchcolumn
    \begin{align*}
        \fbox{$\tau_{61o}=20.4$ MPa} \\
        \fbox{$\tau_{16i}=-20.0$ MPa} \\
        \fbox{$\tau_{12}=11.9$ MPa} \\
        \fbox{$\tau_{25}=-15.2$ MPa} \\
        \fbox{$\tau_{56}=15.6$ MPa} \\
        \fbox{$\tau_{23}=10.0$ MPa} \\
        \fbox{$\tau_{34}=-5.8$ MPa} \\
        \fbox{$\tau_{45}=8.0$ MPa}
    \end{align*}
\end{paracol}
\begin{figure}[H]
    \centering
    \includegraphics[width=0.9\linewidth]{figs/hw6fig3.png}
\end{figure}

\pagebreak
\subsection*{(c)}
Find the twist angle at the wing tip. \\ \\
Using Matlab again, we will solve one of the twist equations (since all are equal). Evaluating $\frac{d\theta}{dz}_I$
gives a twist rate of $1.26E-4$ deg/mm. To find the twist at the tip, we multiply by the total wing length (L):
\begin{align*}
    \theta=2000*1.257E-4 \\
    \fbox{$\theta=0.25^{\circ}$}
\end{align*}

\end{document}