\documentclass[12 pt]{article}
\usepackage[utf8]{inputenc}
\usepackage{graphicx}
\usepackage{amsmath}
\usepackage[version=4]{mhchem}
\usepackage{siunitx}
\usepackage{longtable,tabularx}
\usepackage{float}
\usepackage[left=1in, right=1in]{geometry}

\title{AEEM5058 HW\#3}
\date{09.24.24}
\author{Slade Brooks \\ M13801712}

\begin{document}
\maketitle

\section*{Problem 1}
The same cross section considered in Homework \#1 is modified by introducing a longitudinal cut
at corner 1 spanning the entire length L of the beam (no material is removed). The section is
subject to a shear force F= 25 kN contained in the u-v plane with its line of action intersecting
the shear center (S.C.) and parallel to the u axis.
\begin{figure}[!hbtp]
    \centering
    \includegraphics[width=0.8\linewidth]{figs/hw3fig1.png}
\end{figure} \par

\pagebreak
\subsection*{a)}
Compute the shear flow around the cross section, plot its distribution as a function of the arc-length coordinate, s,
and provide the minimum and maximum values of the shear flow. In your answer show how you compute the shear flow. \\ \\
The shear flow is defined by:
\begin{align*}
    q(s)=q(0)+\alpha f(s) + \beta g(s)
\end{align*}
We will evaluate at the free end starting at $s_1$, so $q(0)=0$. We can define the shear forces based on the direction
of F. Due to the sign conventions, $V_{\eta}$ will be $-F$ and $V_{\zeta}$ will be 0. Adjusting the bounds of s from
HW\#1 and subtracting from the centroid to convert to the eta and zeta axes, we can define the components of the shear
flow equation. \\
\begin{tabular}{|c|c|c|c|}
    \hline
    s & $\eta(s)$ & $\zeta(s)$ & bounds \\ \hline
    $s_2$ & $-u_c$ & $R+s_2-v_c$ & $0 \rightarrow R$ \\
    $s_3$ & $-2R\sin{\frac{s_3}{2R}}-u_c$ & $2R\cos{\frac{s_3}{2R}}-v_c$ & $0 \rightarrow \pi R$ \\
    $s_4$ & $-\frac{3R}{2} - \frac{R}{2}\cos{\frac{2s_4}{R}}-u_c$ & $\frac{R}{2}\sin{\frac{2s_4}{R}}-v_c$ & $0 \rightarrow
    \frac{\pi R}{2}$
    \\
    $s_1$ & $-R\cos{\frac{s_1}{R}}-u_c$ & $-R\sin{\frac{s_1}{R}}-v_c$ & $0 \rightarrow \frac{3\pi R}{2}$ \\
    \hline
\end{tabular} \\ \\
From the previous homeworks:
\begin{align*}
    I_{\eta} = 119284851.3\textrm{mm}^4 \\
    I_{\zeta}=130805242.4\textrm{mm}^4\\
    I_{\eta \zeta}=-22862046.03\textrm{mm}^4
\end{align*} \\
$\alpha$ and $\beta$ can now be calculated. Since thickness is constant, it can be removed from f(s) and g(s) and multiplied
by $\alpha$ and $\beta$ directly.
\begin{align*}
    \alpha = \frac{V_{\eta}I_{\eta}-V_{\zeta}I_{\eta \zeta}}{I_{\eta}I_{\zeta}-I_{\eta \zeta}^2}t = -3.3617*10^{-4} \\
    \beta = \frac{V_{\zeta}I_{\zeta}-V_{\eta}I_{\eta \zeta}}{I_{\eta}I_{\zeta}-I_{\eta \zeta}^2}t = -6.4430*10^{-5}
\end{align*} \\
Now f(s) and g(s) can be calculated. \\
\begin{align*}
    f^{12}_{s_1}=\int_{0}^{s_1}(-R\cos{\frac{s}{R}}-u_c)ds=-u_cs_1-R^2\sin{\frac{s_1}{R}} \\
    f^{23}_{s_2}=f^{12}(s_1)+\int_{0}^{s_2}(-u_c)ds=f^{12}(s_1)-u_cs_2 \\
    f^{34}_{s_3}=f^{23}(s_2)+\int_{0}^{s_3}\left(-2R\sin{\frac{s}{2R}}-u_c\right)ds=f^{23}(s_2)-u_cs_3+4R^2\cos(\frac{s_3}{2R})-4R^2 \\
    f^{41}_{s_4}=f^{34}(s_3)+\int_{0}^{s_4}\left(\frac{-3R}{2}-\frac{R}{2}\cos{\frac{2s}{R}}-u_c\right)ds=f^{34}(s_3)-\frac{3Rs_4}{2}
    -u_cs_4-\frac{R^2}{4}\sin(\frac{2s_4}{R})
\end{align*}
\begin{align*}
    g^{12}_{s_1}=\int_{0}^{s_1}(-R\sin\frac{s}{R}-v_c)ds=-v_cs_1+R^2(\cos{\frac{s_1}{R}}-1) \\
    g^{23}_{s_2}=g^{12}(s_1)+\int_{0}^{s_2}(R+s-v_c)ds=g^{12}(s_1)+Rs_2-v_cs_2+\frac{s_2^2}{2} \\
    g^{34}_{s_3}=g^{23}(s_2)+\int_{0}^{s_3}(2R\cos{\frac{s}{2R}}-v_c)ds=g^{23}(s_2)-v_cs_3+4R^2\sin\frac{s_3}{2R} \\
    g^{41}_{s_4}=g^{34}(s_3)+\int_{0}^{s_4}(\frac{R}{2}\sin{\frac{2s}{R}}-v_c)ds=g^{34}(s_3)-v_cs_4-\frac{R^2}{4}(\cos{\frac{2s_4}{R}}-1)
\end{align*} \\ \\
Now we have representations for all of the parts of the shear flow equation. We will use Python to plot these equations
around the length of the cross section. We can also use python to evaluate the locations and magnitudes of the maximum
and minimum shear flow. \\
\begin{align*}
    \fbox{$q(s)_{max}=7.477$N/mm at s=266.6mm} \\
    \fbox{$q(s)_{min}=-51.448$N/mm at s=1301.1mm}
\end{align*}
\begin{figure}[H]
    \centering
    \includegraphics[width=0.8\linewidth]{figs/hw3fig2.png}
\end{figure}

\pagebreak
\subsection*{b)}
Sketch the distribution of the shear stress around the cross section, indicate its direction, and
provide the value of stress at significant points. \\ \\
The shear stress is simply the shear flow divided by the thickness $t$. The direction is given by the sign of the shear
flow at the point along the cross section. From the plot generated in part a, a sketch of the shear stress can be
created showing the distribution and values.
\begin{figure}[H]
    \centering
    \includegraphics[width=0.85\linewidth]{figs/hw3fig3.png}
\end{figure}
\begin{figure}[H]
    \centering
    \includegraphics[width=0.85\linewidth]{figs/hw3fig4.png}
\end{figure}

\end{document}