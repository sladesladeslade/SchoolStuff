\documentclass[12 pt]{article}
\usepackage[utf8]{inputenc}
\usepackage{graphicx}
\usepackage{amsmath}
\usepackage[version=4]{mhchem}
\usepackage{siunitx}
\usepackage{longtable,tabularx}
\usepackage{paracol}
\usepackage{float}
\usepackage[left=1in, right=1in]{geometry}

\title{AEEM5058 HW\#9}
\date{12.03.24}
\author{Slade Brooks \\ M13801712}

\begin{document}
\maketitle

\section*{Problem 1}
A panel of thickness $t = 3$ mm is clamped at one end and subject to an axial force $F= 9600$N at
the other end. The material has yield stress $\sigma_Y=251$ MPa. Airworthiness regulations require a
safety factor $n=1.25$.
\begin{figure}[H]
    \centering
    \includegraphics[width=0.7\linewidth]{figs/hw11fig1.png}
\end{figure}

\pagebreak
\subsection*{a)}
Show that the panel does not fail under the applied load. \\ \\
We will analyze each feature on its own. First the leftmost rectanglar section with a circular hole:

\subsection*{Big Hole}
Using the stress concentration pdf:
\begin{align*}
    \frac{d}{w}=\frac{15}{75}=0.2 \rightarrow K_t=2.5
\end{align*}
where
\begin{align*}
    K_t=\frac{\sigma_{max}}{\sigma_0}
\end{align*}
Then determine the nominal stress at the minimum cross sectional area and the max from that:
\begin{align*}
    \sigma_0=\frac{F}{A_{min}}=\frac{F}{t(w-d)}=\frac{9600}{3(75-15)}=53.33\text{N/mm$^2$} \rightarrow
    \sigma_{max} = 133.33\text{MPa}
\end{align*}

\subsection*{Fillets}
Same process with stress concentration chart:
\begin{align*}
    \frac{r}{d}=\frac{5}{50}=0.1,\quad \frac{D}{d}=\frac{75}{50}=1.5 \rightarrow K_t=2.1
\end{align*}
Then determine nominal stress and the max:
\begin{align*}
    \sigma_0=\frac{F}{A_{min}}=\frac{F}{td}=\frac{9600}{3\cdot50}=64\text{N/mm$^2$} \rightarrow
    \sigma_{max}= 134.4\text{MPa}
\end{align*}

\subsection*{Little Hole}
Using the stress concentration pdf:
\begin{align*}
    \frac{d}{w}=\frac{10}{50}=0.2 \rightarrow K_t=2.5
\end{align*}
Then determine the nominal stress at the minimum cross sectional area and the max from that:
\begin{align*}
    \sigma_0=\frac{F}{A_{min}}=\frac{F}{t(w-d)}=\frac{9600}{3(50-10)}=80\text{N/mm$^2$} \rightarrow
    \sigma_{max} = 200\text{MPa}
\end{align*}
\\ Then, analyze the yield stress + safety factor to determine if any violated it:
\begin{align*}
    \sigma=\frac{\sigma_Y}{n}=\frac{251}{1.25}=200.8\text{MPa}
\end{align*}
\fbox{Since all of the individual segments have stress lower than the yield with safety factor} \\
\fbox{(200.8MPa), the panel does
not fail.} \\

\subsection*{b)}
To reduce the weight of the panel the diameter of one of the holes can be increased without
causing failure. Identify which hole can be enlarged and propose a new diameter for it that
can reduce weight without compromising the integrity of the panel under the applied load.
Note- it is not necessary to find the largest possible hole diameter. \\ \\
It is clear from the results of part a that the small hole cannot be enlarged any meaningful amount, as its stress is
only 0.8MPa away from the maximum allowable. Thus, we will attempt to double the diameter of the large hole to save
weight. We can solve with the same process as used in part a:
\begin{align*}
    \frac{d}{w}=\frac{30}{75}=0.4 \rightarrow K_t=2.2625
\end{align*}
\begin{align*}
    \sigma_0=\frac{F}{A_{min}}=\frac{F}{t(w-d)}=\frac{9600}{3(75-30)}=71.11\text{N/mm$^2$} \rightarrow
    \sigma_{max} = 160.9\text{MPa}
\end{align*}
Since the result is still lower than the 200.8Mpa limit, \fbox{the large hole could be enlarged to 30mm safely.}

\pagebreak
\section*{Problem 2}
A bar of length $L = 700$ mm has a 15 mm x 45 mm rectangular cross section and is simply
supported at its ends. At the midspan point the bar has two circular notches of radius $r =3.75$ mm
as shown in Fig. 3. The bar is loaded by a constant force $F = 2$ kN and by a time-varying bending
moment, M, that oscillates between $\pm 300$ Nm. Both the force and moment are applied at the
center of the bar. Assuming that the bar is made of a material with ultimate strength $\sigma_u = 280$ MPa
and endurance limit $\sigma_e = 150$ MPa, determine if the bar fails under the Goodman criterion using a
safety factor of 1.2.
\begin{figure}[H]
    \centering
    \includegraphics[width=1.\linewidth]{figs/hw11fig2.png}
\end{figure}

\pagebreak
\subsection*{}
To determine the goodman criterion:
\begin{align*}
    \frac{k_{ctf}\sigma_{rev}}{\sigma_e}+\frac{\sigma_m}{\sigma_u}=\frac{1}{n} \\
    \frac{k_{ctf}\sigma_{rev}}{150}+\frac{\sigma_m}{280}=\frac{1}{1.2}
\end{align*}
First we need to determine the moment on the beam contributed by the applied force. For a simply supported beam with a
point force at the center:
\begin{align*}
    M_F=\frac{FL}{4}=\frac{2000\cdot700}{4}=350000\text{Nmm}
\end{align*}
The stress due to the moment is determined by:
\begin{align*}
    \sigma=\frac{My}{I};\quad y=37.5/2=18.75\text{mm}
\end{align*}
We will need to determine two moments of inertia to analyze the stress due to force and moment separately:
\begin{align*}
    I_{Y'}=\frac{b^3h}{12}=\frac{15^3\cdot37.5}{12}=10546.875\text{mm$^4$} \\
    I_{X}=\frac{bh^3}{3}=\frac{15\cdot37.5^3}{3}=263671.875\text{mm$^4$}
\end{align*}
\begin{align*}
    \sigma_F=\frac{M_Fy}{I_{Y'}}=\frac{350000\cdot18.75}{10546.875}=622.22\text{N/mm$^2$} \\
    \sigma_M=\frac{My}{I_X}=\frac{300000\cdot18.75}{263671.875}=21.33\text{N/mm$^2$}
\end{align*}
Then we can determine the max and min stress, and then determine the mean and reversible stresses for the goodman\includegraphics[scale=0.02]{figs/goodman.jpg}
criterion:
\begin{align*}
    \sigma_{max}=\sigma_F+\sigma_M=643.55\text{N/mm$^2$} \\
    \sigma_{min}=\sigma_F-\sigma_M=600.89\text{N/mm$^2$} \\
    \sigma_m=\frac{\sigma_{max}+\sigma_{min}}{2}=622.22\text{N/mm$^2$} \\
    \sigma_{rev}=\frac{\sigma_{max}-\sigma_{min}}{2}=21.33\text{N/mm$^2$}
\end{align*}
Lastly, we need to determine $k_{ctf}$ from the stress concentration charts:
\begin{align*}
    \frac{r}{d}=\frac{3.75}{37.5}=0.1;\quad\frac{D}{d}=\frac{45}{37.5}=1.2 \rightarrow k_{ctf}=2.0
\end{align*}
Then we will plug in and solve the goodman criterion and see what the actual safety factor would be:
\begin{align*}
    \frac{2\cdot21.33}{150}+\frac{622.22}{280}=\frac{1}{n} \rightarrow n=0.4
\end{align*}
Since the required safety factor is $n=1.2$, \fbox{the beam fails the goodman\includegraphics[scale=0.02]{figs/goodman.jpg} criterion.}

\pagebreak
\section*{Problem 3}
The fuselage of commercial jet aircraft has a diameter of 2.6 m. The aircraft is used typically for
an average of six short-haul passenger flights each day. During each flight, the cabin is pressurized
to 52 kPa. The fuselage skin, 0.9 mm in thickness, is made of an aluminum alloy with fracture
toughness $K_I=30$ MPa$\sqrt{\text{m}}$. The fatigue crack growth behavior is characterized by the relationship
da/dN$=2\times10^{-9}(\Delta\text{K})^3$ where da/dN is in m/cycle and $\Delta$K is in MPa$\sqrt{\text{m}}$. When a routine maintenance
inspection was conducted on the aircraft, the presence of a 0.4-mm long, through thickness crack
in the fuselage skin, oriented parallel to the longitudinal axis of the fuselage, was left undetected
due to human error. If the fuselage is designed such that the cabin pressure could be sustained
even with a longitudinal crack up to 10 cm length, find the safe fatigue life (in years) of the aircraft
from the time of the maintenance. \\ \\
First, we will need to determine the minimum and maximum stress on the fuselage. The minimum can be assumed to be 0
since the vehicle will not see higher pressure than cabin pressure. Furthermore, we will assume the maximum stress to
occur when the external pressure is 0 and determine the hoop stress at that point.
\begin{align*}
    \sigma_{max}=\sigma_h=\frac{P_{cabin}-P_{ext}}{t}R_0=\frac{0.052}{0.0009}(2.6/2)=75.11\text{MPa} \\
    \sigma_{min}=0
\end{align*}
Next, we will determine the Y of the material at the critical crack length:
\begin{align*}
    K_I=Y\sigma_{mean}\sqrt{\pi a_c} \rightarrow Y=\frac{K_I}{\frac{\sigma_{max}+\sigma_{min}}{2}\sqrt{\pi a_c}}=1.425
\end{align*}
Now we have all necessary values to solve for the residual life:
\begin{align*}
    N_f=\frac{2\left(a_c^{\frac{2-m}{2}}-a_i^{\frac{2-m}{2}}\right)}{(2-m)A[Y(\sigma_{max}-\sigma_{min})\sqrt{\pi}]^m}
    =\frac{2\left(0.1^{\frac{2-3}{2}}-0.0004^{\frac{2-3}{2}}\right)}{(2-3)2E-9[1.425(75.11)\sqrt{\pi}]^3}=6860.14\text{cycles} \\
    N_f=6860.14/6/365.25=3.13
\end{align*}
\fbox{$N_f=3.13$ years}

\end{document}