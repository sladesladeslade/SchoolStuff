\documentclass[12 pt]{article}
\usepackage[utf8]{inputenc}
\usepackage{graphicx}
\usepackage{amsmath}
\usepackage[version=4]{mhchem}
\usepackage{siunitx}
\usepackage{longtable,tabularx}
\usepackage{float}
\usepackage[left=1in, right=1in]{geometry}

\title{AEEM5058 HW\#5}
\date{10.08.24}
\author{Slade Brooks \\ M13801712}

\begin{document}
\maketitle

\section*{Problem 1}
A thin-walled cylinder of radius R and wall thickness t is subject to a torque M. In order to lower
the shear stress in the cylinder, designers are considering adding fins of width R/2 and the same
thickness as the cylinder as shown in Fig. 1.
\begin{figure}[!hbtp]
    \centering
    \includegraphics[width=0.5\linewidth]{figs/hw5fig1.png}
\end{figure} \par
\pagebreak

\subsection*{a)}
Find the expression of the shear stress in the cylinder, $\tau_c$, and the maximum shear stress in
the fins, $\tau_f$ as a function of the number of fins, $n$. Note that in Fig. 1 eight fins are show for
illustration purpose. \\ \\
First, we will determine the $\tau$ of each section, we will treat them as separate sections. $\tau_c$ will be a closed
section and $\tau_f$ will be an open section.
\begin{align*}
    \tau_c=\frac{M_c}{2\Sigma t} \\
    \frac{d\theta_c}{dx}=\frac{M_c}{4G\Sigma^2}\int{\frac{1}{t}ds} \\
    \Sigma = \pi R^2 \\
    \frac{d\theta_c}{dx}=\frac{M_c}{4G(\pi R^2)^2}\left(\frac{2\pi R}{t}\right)=\frac{M_c}{2\pi GtR^3} \\
    \tau_c=\frac{M_c}{2\pi tR^2}
\end{align*}
\begin{align*}
    \tau_{f_{max}}=\frac{M_f}{J}t \\
    \frac{d\theta_f}{dx}=\frac{M_f}{GJ} \\
    J=\frac{1}{3}\int_{L}{t^3(s)ds}=\frac{1}{3}\frac{Rt^3}{2}\cdot n=\frac{t^3R}{6}n \\
    \frac{d\theta_f}{dx}=\frac{6M_f}{Gt^3nR} \\
    \tau_{f_{max}}=\frac{6M_f}{t^2nR}
\end{align*}
We can state that the total moment $M$ is equal to the sum of the individual moments:
\begin{align*}
    M=M_c+M_f \\
    M_f=M-M_c
\end{align*}
We also impose that since the pieces are connected their rotations must be equivalent:
\begin{align*}
    \frac{d\theta_c}{dx}=\frac{d\theta_f}{dx} \\
    \frac{M_c}{2\pi GtR^3}=\frac{6M_f}{Gt^3nR} \\
    M_c=\frac{12\pi R^2}{nt^2}M_f=\frac{12\pi R^2}{nt^2}(M-M_c) \\
    M_c(1+\frac{12\pi R^2}{nt^2})=\frac{12\pi R^2}{nt^2}M \\
    M_c=\frac{12\pi R^2}{nt^2+12\pi R^2}M \\
    M_f=M-\frac{12\pi R^2}{nt^2+12\pi R^2}M \\
    M_f=\frac{nt^2}{nt^2+12\pi R^2}M
\end{align*}
Then we can plug back in and solve:
\begin{align*}
    \fbox{$\tau_c=\frac{6}{nt^3+12\pi tR^2}M$} \\
    \fbox{$\tau_{f_{max}}=\frac{6}{Rnt^2+12\pi R^3}M$}
\end{align*}

\pagebreak
\subsection*{b)}
Show that the ratio $\tau_c/\tau_f$ is independent of the number of fins.
\begin{align*}
    \tau_c/\tau_f=\frac{\tau_c}{\tau_{f_{max}}}=\frac{\frac{6}{nt^3+12\pi tR^2}M}{\frac{6}{Rnt^2+12\pi R^3}M}=\frac{R}{t}
\end{align*}
We can see that the ratio of the shear forces is only based on the radius and thickness of the shape, as $n$ disappears
when dividing the equations for the shear in each section.
\\ \\ \\ \\

\subsection*{c)}
Find the radius, $R'$, of a cylinder with the same wall thickness $t$ but without fins that leads
to the same $\tau_c$ as the cylinder with fins. \\ \\
We will set the previously determined $\tau_c$ equal to a new $\tau'$ and solve for the radius that makes them
equivalent.
\begin{align*}
    \tau'=\tau_c \rightarrow \frac{M}{2\pi tR'^2}=\frac{6}{nt^3+12\pi tR^2}M \\
    \fbox{$R'=\sqrt{\frac{nt^2+12\pi R^2}{12\pi }}$}
\end{align*}

\pagebreak
\subsection*{d)}
Show that the cylinder in (c) is always lighter than the cylinder with fins in (a). \\ \\
First, we will determine the cross sectional area of each shape based on the number of fins. The area of each cylinder
will be its diameter times the thickness. The area of the fins is the number of fins times the length of each times the
thickness.
\begin{align*}
    A'=\pi Dt=2\pi R't=t\sqrt{\frac{nt^2+12\pi R^2}{12\pi }} \\
    A=\pi Dt+n\frac{R}{2}t=2\pi Rt+\frac{nRt}{2}
\end{align*}
We will calculate the ratio between the two for ease of comparison. If $A'/A<1$, then the cylinder without fins will be
lighter (its area is some small percentage of the one with fins).
\begin{align*}
    \frac{A'}{A}=\frac{\sqrt{3\pi}\sqrt{nt^2+12\pi R^2}}{3\pi R(n+4\pi)}
\end{align*}
We can assume an infinitesimally small $t$ for simplicity and simplify the ratio.
\begin{align*}
    \lim_{t\rightarrow 0}\frac{A'}{A}=\frac{2}{n+4\pi}
\end{align*} 
We can see that for any number of fins (other than 0), the ratio will be less than 1 meaning $A'$ is smaller than $A$
and therefore the larger cylinder weighs less than the cylinder with fins.
\begin{align*}
    \lim_{n\rightarrow \infty}\frac{A'}{A}=0
\end{align*}
Following is a plot of $\frac{A'}{A}$. The x axis is $n$. It is clear that the ratio approaches 0 as $n$ approaches
infinity, and the ratio is never greater than 1, meaning the cylinder without fins must always be lighter.
\begin{figure}[H]
    \centering
    \includegraphics[width=0.75\linewidth]{figs/hw5fig2.png}
\end{figure} \par

\end{document}