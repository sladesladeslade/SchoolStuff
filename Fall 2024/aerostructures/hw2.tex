\documentclass[12 pt]{article}
\usepackage[utf8]{inputenc}
\usepackage{graphicx}
\usepackage{amsmath}
\usepackage[version=4]{mhchem}
\usepackage{siunitx}
\usepackage{longtable,tabularx}
\usepackage{float}
\usepackage[left=1in, right=1in]{geometry}

\title{AEEM5058 HW\#2}
\date{09.17.24}
\author{Slade Brooks \\ M13801712}

\begin{document}
\maketitle

\section*{Problem 1}
A cantilevered beam of length L = 1.5 m is simply supported at its end as shown in Fig. 1. The
beam has the same cross section as in HW\#1 and is subject to a uniformly distributed load p = 40
kN/m applied at node \#3.
\begin{figure}[!hbtp]
    \centering
    \includegraphics[width=0.8\linewidth]{figs/hw2fig1.png}
\end{figure} \par

\pagebreak
\subsection*{a)}
Determine the maximum compressive and tensile stresses and indicated on the cross
section the points where they occur. \\ \\
The largest stress will occur at the free end farthest from the wall support ($x=L$).
\begin{align*}
    P(x) = 40 \rightarrow V(x)=\int{P(x)}=40x \rightarrow M(x)=\int{V(x)} = 40\frac{x^2}{2}=\frac{40L^2}{2}
    = 45 \textrm{kNm}
\end{align*}
\begin{figure}[H]
    \centering
    \includegraphics[width=0.5\linewidth]{figs/hw2fig3.png}
\end{figure} \par
From HW\#1:
\begin{align*}
    I_{\eta} = 119284851.3\textrm{mm}^4 \\
    I_{\zeta}=130805242.4\textrm{mm}^4\\
    I_{\eta \zeta}=-22862046.03\textrm{mm}^4
\end{align*} \\ \\
\begin{align*}
    \tan{\beta}=\frac{M_{\eta}I_{\eta \zeta}-M_{\zeta}I_{\eta}}{M_{\zeta}I_{\eta \zeta}-M_{\eta}I_{\zeta}} \\
    M_{\eta}=45000\textrm{kNmm};\quad M_{\zeta}=0 \\
    \tan{\beta}=\frac{45000(-22862046.03)}{-45000(130805242.4)} \rightarrow \beta=9.9^{\circ}
\end{align*}
\begin{figure}[H]
    \centering
    \includegraphics[width=0.6\linewidth]{figs/hw2fig4.png}
\end{figure} \par
\begin{align*}
    \eta_1 = 102.759;\quad \zeta_1=400-95.925=304.075 \\
    \eta_2 = 102.759-R\sin{\beta}=68.37;\quad \zeta_2=-95.925-R\cos{\beta}=-292.95 \\
    \eta_1 = 102.759\textrm{mm};\quad \zeta_1=304.075\textrm{mm} \\
    \eta_2 = 68.37\textrm{mm};\quad \zeta_2=-292.95\textrm{mm}
\end{align*} \\ \\
Finally, we can calculate the stress at each of the maximum stress locations.
\begin{align*}
    \sigma_{xx}=-\frac{M_{\eta}(I_{\zeta}\zeta-I_{\eta \zeta}\eta)}{I_{\eta}I_{\zeta}-I_{\eta \zeta}^2}
    -\frac{M_{\zeta}(I_{\eta}\eta - I_{\eta \zeta}\zeta)}{I_{\eta}I_{\zeta}-I_{\eta \zeta}^2}
\end{align*} \\ \\
For each point, plug in the values and solve for $\sigma_{xx}$:
\begin{align*}
    \sigma_{x1}=-0.1216 \textrm{kN/mm}^2 \\
    \sigma_{x2}=0.1061 \textrm{kN/mm}^2 \\
    \fbox{$\sigma_{x1}=-121.6 \textrm{MPa (compression)}$} \\
    \fbox{$\sigma_{x2}=106.1 \textrm{MPa (tension)}$}
\end{align*}

\pagebreak
\subsection*{b)}
Determine the principal axes of the cross section, their corresponding second moments of
area, and sketch them on the cross section. Find the angle that one of the axis (either y or
z) form with one of the $\eta$, $\zeta$ axes. \\ \\
From HW\#1:
\begin{align*}
    I_{\eta} = 119284851.3\textrm{mm}^4 \\
    I_{\zeta}=130805242.4\textrm{mm}^4\\
    I_{\eta \zeta}=-22862046.03\textrm{mm}^4
\end{align*} \\ \\
Create matrix for principal axes:
\begin{align*}
    \begin{bmatrix}
        I_{\eta} & -I_{\eta \zeta} \\
        -I_{\eta \zeta} & I_{\zeta}
    \end{bmatrix} = \begin{bmatrix}
        119284851.3 & 22862046.03 \\
        22862046.03 & 130805242.4
    \end{bmatrix}
\end{align*} \\ \\
Determine the eigenvalues and eigenvectors via Matlab. The eigenvectors are the unit vectors in the direction of the
principal axes, and the eigenvalues are the moments of inertia for each axis.
\begin{align*}
    \begin{bmatrix}
        1.0147 \\
        1.4862
    \end{bmatrix}*10^8; \quad \begin{bmatrix}
        -0.7888 & 0.6147 \\
        0.6147 & 0.7888
    \end{bmatrix} \\ \\
    \fbox{$I_z = 101470000\textrm{mm}^4; \quad \vec{u_z}=-0.7888\hat{i}+0.6147\hat{j}$} \\
    \fbox{$I_y = 148620000\textrm{mm}^4; \quad \vec{u_y}=0.6147\hat{i}+0.7888\hat{j}$}
\end{align*} \\ \\
Determine the angle between the y axis and the $\eta$ axis:
\begin{align*}
    \theta = \tan^{-1}\left(\frac{0.7888}{0.6147}\right)=52.07^{\circ}
\end{align*} \\
\begin{figure}[H]
    \centering
    \includegraphics[width=0.75\linewidth]{figs/hw2fig2.png}
\end{figure} \par

\end{document}