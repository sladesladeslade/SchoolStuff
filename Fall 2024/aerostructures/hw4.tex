\documentclass[12 pt]{article}
\usepackage[utf8]{inputenc}
\usepackage{graphicx}
\usepackage{amsmath}
\usepackage[version=4]{mhchem}
\usepackage{siunitx}
\usepackage{longtable,tabularx}
\usepackage{float}
\usepackage[left=1in, right=1in]{geometry}

\title{AEEM5058 HW\#4}
\date{10.01.24}
\author{Slade Brooks \\ M13801712}

\begin{document}
\maketitle

\section*{Problem 1}
The cross section shown in Fig 1 is obtained from the one considered in previous homework by
replacing arc 41 with a straight element and maintaining a cut at node 1. The dimensions continue
to be R=200 mm and t =1.7 mm. Find the vertical coordinate $v_{S.C.}$ of the shear center (S.C.) relative
to the frame (u,v). You can use the expressions provided in the Appendix to aid your analysis.
\begin{figure}[!hbtp]
    \centering
    \includegraphics[width=0.5\linewidth]{figs/hw4fig1.png}
\end{figure} \par
\pagebreak
First we will apply an arbitrary shear force. Since we only need to find the $v$ location, we will apply a force in the
$V_\eta$ direction at an unknown $v$ coordinate $e$. For simplicity, we can use the 25kN force in the negative direction
from the previous
homework. The moment of this arbitrary force will be set equal to the moment of the shear flow. We can substitute in the
know functions for the shear flow to get:
\begin{align*}
    -eF=\alpha \int_{0}^{L}{f_{\eta}(s)h(s)ds} + \beta \int_{0}^{L}{g_{\eta}(s)h(s)ds}
\end{align*} \\ \\
Then, we can define the $h(s)$ for each section, which can be determined by the minimum distance. For the straight
segments, this will be 0 because they have no tangent points to the origin. Then, these can be plugged into the integral
for each section.
\begin{align*}
    h^{12}(s)=R \\
    h^{23}(s)=0 \\
    h^{34}(s)=2R \\
    h^{41}(s)=0
\end{align*} \\
\begin{align*}
    \int{f^{12}(s)h^{12}(s)}=\int_{-\pi}^{\pi/2}(R^2\sin{\theta_1}-u_cR\theta_1-u_cR\pi)(R)Rd\theta \\
    \int{f^{23}(s)h^{23}(s)}=0 \\
    \int{f^{34}(s)h^{34}(s)}=\int_{\pi/2}^{\pi}(-3R^2-\frac{1}{2}\pi u_cR-u_cR+4R^2\sin{\theta_3}-2u_cR\theta_3)(2R)Rd\theta \\
    \int{f^{41}(s)h^{41}(s)}=0
\end{align*}
\begin{align*}
    \int{g^{12}(s)h^{12}(s)}=\int_{-\pi}^{\pi/2}(-R^2\cos{\theta_1}-\theta_1Rv_c-R^2-v_c\pi R)(R)Rd\theta \\
    \int{g^{23}(s)h^{23}(s)}=0 \\
    \int{g^{34}(s)h^{34}(s)}=\int_{\pi/2}^{\pi}(-\frac{1}{2}v_c\pi R-v_cR+\frac{R^2}{2}-4R^2\cos{\theta_3}-2Rv_c\theta_3)(2R)Rd\theta \\
    \int{g^{41}(s)h^{41}(s)}=0
\end{align*} \\
Next, we will solve for $\alpha$ and $\beta$. They are slightly simplified because $V_{\zeta}=0$.
\begin{align*}
    I_{\eta}=I_u-Av_c^2=1.213*10^8 \\
    I_{\zeta}=I_v-Au_c^2=1.213*10^8 \\
    I_{\eta \zeta}=I_{uv}-Au_cv_c=-1.965*10^7
\end{align*}
\begin{align*}
    \alpha=t\frac{V_{\eta}I_{\eta}}{I_{\eta}I_{\zeta}-I_{\eta\zeta}^2}=-3.599*10^{-4} \\
    \beta=t\frac{-V_{\eta}I_{\eta\zeta}}{I_{eta}I_{\zeta}-I_{\eta\zeta}^2}=-5.831*10^{-5}
\end{align*} \\ \\
Lastly, we can solve for the distance. Plugging in the integrals (and solving with an online calculator) and known
values allows for $e$ to be solved for.
\begin{align*}
    e=\frac{\alpha \int_{0}^{L}{f_{\eta}(s)h(s)ds} + \beta \int_{0}^{L}{g_{\eta}(s)h(s)ds}}{-F} \\ \\
    \fbox{$e=462.6$ mm}
\end{align*}
\pagebreak
\section*{Problem 2}
Compute the vertical coordinate $v_{S.C.}$ of the shear center (S.C.) relative to the frame (u,v)
assuming that the section in Problem \#1 is closed by a weld line along corner 1 as shown in Fig.
2.
\begin{figure}[!hbtp]
    \centering
    \includegraphics[width=0.5\linewidth]{figs/hw4fig2.png}
\end{figure} \par
\pagebreak
First, for a closed section the same arbitrary force and location are applied, but the equation is:
\begin{align*}
    -eF=\int(q_o+q_B(s))h(s)ds
\end{align*} \\ \\
We already determined some equations in the previous problem, so we can divide those by h(s) to get $q_b(s)$.
\begin{align*}
    \int{f^{12}(s)h^{12}(s)}=\int_{-\pi}^{\pi/2}(R^2\sin{\theta_1}-u_cR\theta_1-u_cR\pi)Rd\theta \\
    \int{f^{23}(s)h^{23}(s)}=\int_{0}^{R}(R^2-\frac{3}{2}\pi u_cR-u_cs_2)ds \\
    \int{f^{34}(s)h^{34}(s)}=\int_{\pi/2}^{\pi}(-3R^2-\frac{1}{2}\pi u_cR-u_cR+4R^2\sin{\theta_3}-2u_cR\theta_3)Rd\theta \\
    \int{f^{41}(s)h^{41}(s)}=\int_{0}^{R}(-3R^2-\frac{5}{2}\pi u_cR-u_cR-(2R+u_c)s_4+\frac{s_4^2}{2})ds
\end{align*}
\begin{align*}
    \int{g^{12}(s)h^{12}(s)}=\int_{-\pi}^{\pi/2}(-R^2\cos{\theta_1}-\theta_1Rv_c-R^2-v_c\pi R)Rd\theta \\
    \int{g^{23}(s)h^{23}(s)}=\int_{0}^{R}(-R^2-\frac{3}{2}v_c\pi R+(R-v_c)s_2+\frac{s_2^2}{2})ds \\
    \int{g^{34}(s)h^{34}(s)}=\int_{\pi/2}^{\pi}(-\frac{1}{2}v_c\pi R-v_cR+\frac{R^2}{2}-4R^2\cos{\theta_3}-2Rv_c\theta_3)Rd\theta \\
    \int{g^{41}(s)h^{41}(s)}=\int_{0}^{R}(-\frac{5}{2}v_c\pi R-v_cR+\frac{9R^2}{2}-v_cs_4)ds
\end{align*} \\ \\
$\alpha$ and $\beta$ will be the same for the closed section. Then we can plug in and solve the integral (with an online
calculator) for $q_B$ in order to obtain $q_o$.
\begin{align*}
    q_B=\alpha \int_{0}^{L}{f_{\eta}(s)ds} + \beta \int_{0}^{L}{g_{\eta}(s)ds}=-42094.23 \\
    q_o=\frac{q_B}{L} \\
    L = 0.75(2\pi R)+R+0.25(4\pi R)+R=1970.79 \\
    q_o=21.36
\end{align*} \\ \\
Lastly, we can find the area enclosed by the shape $\Sigma$ and use that to determine the shear center location $e$. We
will solve with an online calculator in the same manner as problem 1.
\begin{align*}
    \Sigma=0.75\pi R^2+0.25\pi(2R)^2=219911.49 \\
    e=\frac{2\Sigma q_o+\int{q_B(s)h(s)ds}}{-F} = \fbox{$86.86$ mm}
\end{align*}

\end{document}